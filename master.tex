\documentclass[12pt, a4paper, UTF8, fontset=adobe, oneside]{ctexbook} % oneside 去掉所有空白页

\setmainfont{Times New Roman} % 英文用 Times New Roman 字体
\linespread{1.3} % 行距设置
\setcounter{secnumdepth}{3} % 层次为 3 以上的标题生成序号

%% 宏包
\usepackage{amsmath} % AMS数学宏包
\usepackage{fancyhdr} % 设置页眉页脚宏包
\usepackage{geometry} % 设置页边距宏包
\usepackage{xcolor} % 颜色宏包
\usepackage{hyperref} % 交叉引用宏包 colorlinks启用彩色模式 参考文献引用为紫红色
\usepackage[listings,breakable]{tcolorbox} % 彩色盒子宏包 代码宏包
\usepackage{enumitem} % 枚举设置宏包
\usepackage{tikz} % 画图宏包
\usepackage{booktabs} % 表格宏包
\usepackage{dsfont} % 数字粗体宏包

% 宏包设置
% 页眉页脚样式
\pagestyle{fancy} % 页面样式采用fancyhdr宏包中的fancy
\fancyhf{} % 去掉页眉
\cfoot{\thepage} % 页脚中间显示页码
\renewcommand{\headrulewidth}{0pt} % 去掉页眉的横线
% 页边距设置
\geometry{top = 2.54cm, bottom = 2.54cm, left = 3.18cm, right = 3.18cm}
% 章节样式设置
\CTEXsetup[name={第,章},number={\arabic{chapter}}]{chapter}
% 文档设置
\renewcommand\contentsname{目录} % 中文 目录
\renewcommand\bibname{参考文献} % 中文 参考文献
% 清华紫
\definecolor{THU}{RGB}{111, 23, 135}
% 交叉引用宏包设置
\hypersetup{colorlinks=true,linkcolor=THU,citecolor=THU}

% tcolorbox 样式设置
\newtcolorbox{redbox}[2][]{colback=yellow!10,colframe=red!75!black,coltitle=white,fonttitle=\bfseries,fontupper=\kaishu,title=#2,#1,breakable} % 红色
\newtcolorbox{RCbox}[2][]{colback=yellow!10,colframe=red!75!black,coltitle=white,fonttitle=\bfseries,fontupper=\kaishu,title=#2,#1,center
  title, center upper,breakable} % 红色居中
\newtcolorbox{magbox}[2][]{colback=yellow!10,colframe=magenta!75!black,coltitle=white,fonttitle=\bfseries,fontupper=\kaishu,title=#2,#1} % 紫红色
\newtcolorbox{THUbox}[2][]{colback=yellow!10,colframe=THU!75!black,coltitle=white,fonttitle=\bfseries,fontupper=\kaishu,title=#2,#1,breakable} % 紫罗兰色
\newtcolorbox{THUCbox}[2][]{colback=yellow!10,colframe=THU!75!black,coltitle=white,fonttitle=\bfseries,fontupper=\kaishu,title=#2,#1,center title,center upper,breakable} % 紫罗兰色 居中
\newtcolorbox{purbox}[2][]{colback=yellow!10,colframe=purple!75!black,coltitle=white,fonttitle=\bfseries,fontupper=\kaishu,title=#2,#1,center title,center upper} % 紫色
% \usetikzlibrary{calc,shapes.multipart,chains,arrows,positioning} % tikz library
% \tikzset{circarrow/.style={*->,shorten <=-2pt}}

% enumerate 样式设置
\setlist[enumerate]{label={\arabic*.},leftmargin=2.5em,align=left,topsep=0em,itemsep=-0.5em,labelsep=-0.5em,  before=\vspace{2pt},after=\vspace{2pt}}
\setlist[itemize]{leftmargin=2.5em,align=left,topsep=0em,itemsep=-0.5em,labelsep=-0.5em,  before=\vspace{2pt},after=\vspace{2pt}}

% 引用参考文献时提高位置
\newcommand{\citerb}[1]{\raisebox{1pt}{\cite{#1}}}

\begin{document}
\frontmatter
\begin{titlepage}
\begin{center}

\vspace*{5cm}
% Title
{\huge \bfseries 计算机视觉深度学习技术栈知识总结}\\[0.4cm]

\vspace{12cm}

{\large 江浩} \\[1cm]
{\large \today}

\end{center}
\end{titlepage}

{
\hypersetup{linkcolor=black} % 目录链接为黑色
\pagenumbering{Roman} % 页码编号为大写罗马数字
\tableofcontents % 目录
}

\mainmatter % 正文部分 重新编号
\pagenumbering{arabic} % 页码编号为阿拉伯数字

\part{数学}
\chapter{概率}

%\section{基础}
%
%\section{几何概率}
%
%\subsection{单位圆圆周任取两点的弦长}
%
%问题:单位圆的圆周上,任取两点,求两点之间弦长的期望。

% 模型性能即可达到全部图片训练模型性能的 \textbf{95\%},见表~\ref{tab:AL-Normal}。

%\begin{table}[htb]
%  \centering
%  \caption{主动学习(AL)模型性能与正常训练(Normal)模型性能对比}
%  \label{tab:AL-Normal}
%    \begin{tabular}{cccc}
%    \specialrule{0em}{10pt}{1pt}
%      \toprule[1.5pt]
%      {\heiti 数据集} & {\heiti AL 模型 mAP(35\%)} & {\heiti Normal 模型 mAP(100\%)} & AL/Normal \\ \midrule[1pt]
%      PASCOL VOC & 73.46 & 77.20 & 95.15\% \\
%      KITTI & 54.94 & 57.67 & 95.27\% \\
%      \bottomrule[1.5pt]
%    \end{tabular}
%\end{table}

%%% Local Variables:
%%% TeX-master: "../master"
%%% End:
\part{数据结构和算法}

%%% Local Variables:
%%% TeX-master: "../master"
%%% End:

\part{机器学习和深度学习基础}

\chapter{基本概念}

\section{激活函数}

激活函数的作用是引入非线性因素,提升神经网络的拟合能力。如果没有激活函数,无论神
经网络有多少层,输出总为输入的线性函数,因此网络的拟合能力将会十分有限。

\subsection{Sigmoid 函数}
\label{subsec:Sigmoid}

Sigmoid 函数的原函数和导数为:
\begin{align}
  \label{equ:sigmoid}
  \sigma(x) & = \frac{1}{1 + e^{-x}} \\
  \label{equ:sigmoid-d}
  \sigma'(x) & = \sigma(x) (1-\sigma(x))
\end{align}

Sigmoid 函数的图像如图~\ref{fig:sigmoid}~所示:

\begin{figure}[ht]
  \centering
  \includegraphics[width=0.8\textwidth]{images/机器学习和深度学习基础/基本概念/Sigmoid.png}
  \caption{Sigmoid 函数及其导数的图像}
  \label{fig:sigmoid}
\end{figure}

由图~\ref{fig:sigmoid}~中的蓝色曲线可知,Sigmoid 函数的形状与字母 S 类似,可以
将 $-\infty$ 到 $\infty$ 的输入单调映射到 0 到 1 之间。由图~\ref{fig:sigmoid}~中
的红色曲线可知,在输入值较大或较小时,Sigmoid 函数的导数趋近于 0,这个性质可能导
致网络出现~\ref{subsec:gradient-vanish-explosion}~小节提到的梯度消失问题,从而导
致网络无法正常训练。

\subsection{线性整流函数(ReLU)}

线性整流函数(Rectifier Linear Unit)是目前应用广泛的激活函数,其形式为:

\begin{equation}
  \label{equ:ReLU}
  \mathrm{ReLU}(x) = \left\{
    \begin{array}{lr}
      x & x > 0 \\
      0 & x \leq 0 \\
    \end{array}
  \right.
\end{equation}

ReLU 的图像很简单,如图~\ref{fig:ReLU}~所示。由图中可看出,ReLU 在 $x > 0$ 时对
应的导数始终为 1,不会出现梯度消失现象;但当 $x \leq 0$ 时梯度始终为 0,同样会导
致梯度消失。

\begin{figure}[ht]
  \centering
  \includegraphics[width=0.5\textwidth]{images/机器学习和深度学习基础/基本概念/ReLU.png}
  \caption{ReLU 的图像}
  \label{fig:ReLU}
\end{figure}

\section{损失函数}

\subsection{交叉熵损失函数}

\subsection{$L_1$ 损失函数}

\subsection{$L_2$ 损失函数}

\subsection{Smooth $L_1$ 损失函数}
Smooth L1 是 $L_1$ 和 $L_2$ 损失函数的组合,其形式为:
\begin{equation}
  \label{equ:SmoothL1}
  \mathrm{SL}_1(x) = \left\{
    \begin{array}{lr}
      |x| & |x| > \alpha \\
      \frac{1}{\alpha} x^2 & |x| \leq \alpha \\
    \end{array}
  \right.
\end{equation}

由上式可知,Smooth $L_1$ 在 $x$ 的绝对值较小时为 $L_2$ 函数,而在 $x$ 的绝对值较
大时为 $L_1$ 函数。R-CNN 系列文章中,回归 loss 均使用 Smooth $L_1$ 形
式\citerb{2015-Fast-RCNN, 2015-Faster-RCNN}。

\subsection{Focal loss}

Focal loss 是由文献\citerb{2017-RetinaNet}中提出,用于解决目标检测中正负样本不平
衡的问题。一般而言,只要样本存在类别不平衡,都可以尝试引入 focal loss 解决,相当
于进行了难样本挖掘(HEM)。

Focal loss 的形式和图像如公式~\ref{equ:focal-loss}~和图~\ref{fig:focal-loss}~所示:
\begin{equation}
  \label{equ:focal-loss}
  \mathrm{FL}(p_{\mathrm{t}}) = - \alpha_t (1-p_{\mathrm{t}})^{\gamma} \mathrm{log}(p_{\mathrm{t}})
\end{equation}

\begin{figure}[ht]
  \centering
  \includegraphics[width=0.8\textwidth]{images/机器学习和深度学习基础/基本概念/Focal-Loss.pdf}
  \caption{Focal loss 的图像($\alpha_t = 1$)}
  \label{fig:focal-loss}
\end{figure}

图~\ref{fig:focal-loss}~中,$\gamma = 0$ 的蓝色曲线即传统的交叉熵 loss,可以认为
是 focal loss 的一个特例。其他颜色的曲线分别表示 $\gamma$ 取不同数值时的 focal
loss。Focal loss 中两个参数的具体作用是:

\begin{itemize}
  \item $\alpha_t$:用于平衡不同类别样本的权重,一般所有类别对应的 $\alpha_t$ 之
    和为 1。
  \item $\gamma$:用于平衡难易样本的权重,$\gamma$ 越大,则简单样本的等效权重越
    小。例如 $\gamma = 2$ 时,$p = 0.9$ 的样本,focal loss 的值比交叉熵 loss 的
    值小 100 倍,而 $p = 0.968$ 的样本则小 1000 倍。
\end{itemize}

\section{反向传播算法}
考虑一个神经网络中的某个全连接层 $Y = XW + b$,设网络的 loss 为 $L$,根据链式法
则,有如下关系~\citerb{2017-FC-BP}:

\begin{align}
  \label{equ:bp-fc}
  \frac{\partial L}{\partial X} & = \frac{\partial L}{\partial Y} \frac{\partial Y}{\partial X} = \frac{\partial L}{\partial Y} W^{\mathrm{T}} \\
  \frac{\partial L}{\partial W} & = \frac{\partial L}{\partial Y} \frac{\partial Y}{\partial W} = X^{\mathrm{T}} \frac{\partial L}{\partial Y}\\
  \frac{\partial L}{\partial b} & = \frac{\partial L}{\partial Y} \frac{\partial Y}{\partial b} = \frac{\partial L}{\partial Y}
\end{align}

上面三个式子中,$\frac{\partial L}{\partial X}$ 是为了继续进行反向传播,因为上一
层参数导数的计算依赖该值;$\frac{\partial L}{\partial W}$ 和 $\frac{\partial
  L}{\partial b}$ 即更新参数时需要的梯度。前向计算完成后,逐层向后计算 loss 对每
一层输入($X$)和参数($W, \, b$)的梯度。

\subsection{梯度消失和梯度爆炸}
\label{subsec:gradient-vanish-explosion}
根据链式法则,对于深层神经网络,其浅层参数的梯度与之后所有层输出对输入梯度的乘积
相关。直观理解,如果梯度都小于 1,相乘后的值很小,导致梯度消失;反之如果梯度都大
于 1,相乘后的值很大,导致梯度爆炸。

梯度消失的一个典型原因是使用了 Sigmoid 函数,具体原因请见~\ref{subsec:Sigmoid}小节中的分析。

梯度爆炸的一个简单直接的解决方法是梯度裁剪(Gradient Clipping),即当梯度的值超过
一定阈值时,将其设为阈值的值,避免出现梯度过大的情况。

\section{归一化}

\subsection{批归一化}
\label{sub:BN}

批归一化(Batch Normalization,BN)

\chapter{卷积神经网络}

\section{卷积类型}

\subsection{分组卷积}

分组卷积(Group Convolution,GC)在最早的 AlexNet 中就已经应用,但当时应用的原因是 GPU 计算能力不足,只能
将标准卷积拆成 2 个分组卷积~\citerb{2012-AlexNet}。分组卷积和标准卷积的对比如
图 \ref{fig:normal-conv} 和图 \ref{fig:group-conv} 所示~\citerb{2019-Diff-Conv}:

\begin{figure}[ht]
  \centering
  \includegraphics[width=0.8\textwidth]{images/机器学习和深度学习基础/卷积神经网络/标准卷积.png}
  \caption{标准卷积}
  \label{fig:normal-conv}
\end{figure}

\begin{figure}[ht]
  \centering
  \includegraphics[width=0.8\textwidth]{images/机器学习和深度学习基础/卷积神经网络/分组卷积.png}
  \caption{分组卷积}
  \label{fig:group-conv}
\end{figure}

由以上两图可知,分组卷积的具体步骤是:

\begin{enumerate}
  \item 将输入拆分成 $g$ 组,每组数据的尺寸与输入相同,均为 $H_{\mathrm{in}}
    \times W_{\mathrm{in}}$,但每组的 channel 数均为输入的 $ 1/g $,即 channel
    数为 $ C_{\mathrm{in}} / g $。
  \item 对拆分后的 $g$ 组数据分别进行标准卷积,每一组数据对应的输出的 channel 数
    均为 $ C_{\mathrm{out}}/g $。
  \item 将 $g$ 组数据的输出结果进行 concat,得到 channel 数为 $C_{\mathrm{out}}$
    的输出。
\end{enumerate}

由以上分析可知,与标准卷积相比,分组卷积的计算量约为其 $ 1 / g^2 $,即组数越多,
计算量越小。

分层卷积(Depthwise Convolution,DW)是分组卷积的一种特例,指的是组数 $g$ 与 输
入数据的 channel 数 $C_{\mathrm{in}}$ 相同的卷积,即每组卷积的输入只有 1 个 channel。

\subsection{深度可分离卷积}

深度可分离卷积(Depthwise Separable Convolution,DSC)如图~\ref{fig:ds-conv}~所
示~\citerb{2019-Diff-Conv}:

\begin{figure}[ht]
  \centering
  \includegraphics[width=0.8\textwidth]{images/机器学习和深度学习基础/卷积神经网络/深度可分离卷积.png}
  \caption{深度可分离卷积}
  \label{fig:ds-conv}
\end{figure}

深度可分离卷积的步骤包括分层卷积和逐点卷积两部分,计算量为标准卷积计算量的 $ 1 /
C_{\mathrm{in}} + 1/k^2 $,$k$ 为分层卷积的卷积核大小。
当 channel 数 $C_{\mathrm{in}}$ 较大时,上式可近似为 $1 / k^2$~\citerb{2017-MobileNet-v1}。

\subsection{反卷积}

反卷积(Deconvolution),有时也称转置卷积(Transposed Convolution),是一种特殊的卷积
形式。反卷积的具体步骤是,先通过补 0 扩大输入图像的尺寸,再进行标准卷积。反卷积的
作用是将卷积后尺寸较小的 feature map 恢复到与输入大小相同的尺寸。 需要特别说明的
是,反卷积虽然可以恢复尺寸,\textbf{但不能保证恢复后的数值与输入相同}~\citerb{2016-Guide-Conv}。

\section{感受野}

感受野(Receptive Field,RF) 是指卷积神经网络中某一层的一个像素,会受到多少个输
入数据像素的影响。直观上理解,就是该像素能看到多大范围的输入层像
素~\citerb{2017-Guide-to-RF-cal}。具体的计算方法为:
\begin{align}
\label{equ:rf-cal}
j_{\mathrm {out}} & = j_{\mathrm{in}} \times s \\
r_{\mathrm {out}} & = r_{\mathrm{in}} + (k-1) \times j_{\mathrm{in}}
\end{align}

其中 $j$ 为 jump,相当于每一层的 stride;s 为卷积的 stride,k 为卷积的 kernel
size,r 即感受野。利用 \eqref{equ:rf-cal} 即可计算各层的感受野。

经典网络 VGG-16 各层感受野的详细计算可以参考~\citerb{2018-VGG-16-RF-cal}。

%%% Local Variables:
%%% TeX-master: "../master"
%%% End:

\part{编程语言}
\chapter{C++}
\section{面向对象编程}
\subsection{面向对象的三大基本特征}

\subsubsection{封装}
封装是指将对象抽象成具体的类,同时进行接口和实现的分离,外界只能通过接口与类的对
象交互,将内部的信息隐藏。

\subsubsection{继承}
继承是指子类继承父类的数据成员和函数,并根据需要进行扩展或重写,无需完全重写新类,
实现代码的复用。

\subsubsection{多态}
多态是指通过将父类的指针或引用指向子类对象,在运行时根据所指对象类型实现对应功能,
即同一个父类指针通过指向不同的对象,可以表现出不同形态。

\section{基础}
\subsection{关键字}
\subsubsection{static}
static 关键字主要应用于以下四种情况:

\begin{itemize}
  \item 静态局部变量
  \item 静态全局变量
  \item 静态数据成员
  \item 静态成员函数
\end{itemize}

\subsection{变量}
\subsubsection{作用域}
C++ 中,根据作用域的不同,可以将对象分为局部对象、全局对象和动态对象。

\begin{table}[htbp]
  \centering
  \caption{C++ 的变量作用域}
  \label{tab:cpp-variable-scope}
  \begin{tabular}{cccc}
    \specialrule{0em}{10pt}{1pt}
    \toprule[1.5pt]
    {\heiti 名称} & {\heiti 作用域} & {\heiti 相关关键字} & {\heiti 存储区域} \\
    \midrule[1pt]
    局部对象       & 函数/块作用域内    & 无         & 栈 \\
    文件内全局对象 & 所在文件           & static     & 静态存储区 \\
    全局对象       & 整个程序           & 无         & 静态存储区 \\
    动态对象       & 申请到释放内存期间 & new/delete & 堆 \\
    \bottomrule[1.5pt]
  \end{tabular}
\end{table}

\subsubsection{初始化}
C++ 中定义变量时如果没有指定初值,则会根据变量类型和位置进行对应的初始化。

\begin{itemize}
  \item 内置类型:非静态局部变量将不被初始化,拥有未定义的值;静态局部变量和全局
    变量进行值初始化,初值为 0。
  \item 类类型:与位置无关,统一调用默认构造函数生成对象,如果类不支持默认初始化,
    要求显式提供初值,将会出现编译错误。
\end{itemize}

\chapter{Python}


%%% Local Variables:
%%% TeX-master: "../master"
%%% End:

\part{目标检测}
\chapter{基本概念}

\section{指标}

\subsection{交并比}
交并比(Intersection over Union,IoU)

\section{非极大值抑制}

非极大值抑制(Non maximum suppression,NMS)是一种后处理方法,其作用是保证一个待
检测物体只有一个检测框与之对应,更直观地理解其实是极大值保留,即只保留(置信度)
是极大值的检测框。图~\ref{fig:nms}~给出了一个具体示例\citerb{2018-NMS}:

\begin{figure}[ht]
  \centering
  \includegraphics[width=0.8\textwidth]{images/目标检测/NMS.png}
  \caption{NMS 示例}
  \label{fig:nms}
\end{figure}

\subsection{传统 NMS}

传统 NMS 算法的具体流程是,先根据分数对所有检测框进行排序,然后从分高到分低遍历所
有检测框,如果高分检测框与低分检测框的 IoU 大于一定阈值,则将低分检测框删掉,后续
遍历时不再处理,如图~\ref{fig:nms-algo}~中的红框。

\begin{figure}[ht]
  \centering
  \includegraphics[width=0.5\textwidth]{images/目标检测/Soft-NMS.pdf}
  \caption{NMS 和 Soft NMS 算法流程}
  \label{fig:nms-algo}
\end{figure}

\subsection{Soft NMS}

传统 NMS 算法需要选择合适的两个检测框的 IoU 阈值,否则阈值太低会导致误检,而阈值
太高又会导致漏检。Soft NMS 的基本思想是根据两个检测框的 IoU,降低低分检测框的分
数,但并不直接将其删除,具体形式如图~\ref{fig:nms-algo}~中的绿框~\citerb{2017-Soft-NMS}。

\subsection{Softer NMS}
文献~\citerb{2018-Softer-NMS}~中在 Soft NMS 的基础上提出了 Softer NMS,具体流程
如图~\ref{fig:softer-nms}~所示:

\begin{figure}[ht]
  \centering
  \includegraphics[width=0.5\textwidth]{images/目标检测/Softer-NMS.pdf}
  \caption{Softer NMS 算法流程}
  \label{fig:softer-nms}
\end{figure}

由图~\ref{fig:softer-nms}~可知,Softer NMS 在 Soft NMS 基础上,加入了 var voting
的部分,其具体步骤是:

\begin{enumerate}
  \item 对任一检测框,计算所有分数低于该框的检测框与其的 IoU。
  \item 根据 IoU 和每个检测框的不确定度 $\sigma$,修正该检测框的位置,其中 IoU
    越大,$ \sigma $ 越小,则相应的权重越高。
\end{enumerate}

\chapter{两阶段方法}

\chapter{单阶段方法}

\section{YOLO 系列}
\label{sec:YOLO}

\subsection{YOLO v1}
\label{subsec:YOLOv1}
YOLO v1 将检测问题全部用回归问题描述,没有任何分类的部分。

\subsubsection{检测方法}
YOLO v1 的检测方法如下:

\begin{enumerate}
  \item 将整张图片划分为 $S \times S$ 个网格,每个网格预测 $ B $ 个检测框和 $ C $
  个条件概率 $ \mathrm{Pr}(\mathrm{Class_i}|\mathrm{Object}) $。
  \item 每个 bounding box 共包括 5 个值 $x, y, w, h$ 和 confidence。其中 $x, y$为中
  心位置,$w, h$ 为长和宽(相对图片而言),confidence 为$\mathrm{Pr}(\mathrm{Object}) \times
  \mathrm{IoU}^{\mathrm{truth}}_{\mathrm{pred}}$,inference 时直接将 confidence
  乘以该网格对应的条件概率得到最终的分数。因此网络最终的输出维度为 $ S \times S
  \times (B \times 5 + C) $。
\end{enumerate}


\subsubsection{Backbone}
YOLO v1 的 backbone 没有采用改造的经典分类网络,而是采用结构为 24 个卷积层 + 2 个
全连接层的自己设计的网络。训练时,先将前 20 个卷积层 + 平均池化层组成的网络
在 ImageNet 数据集上进行预训练,输入尺寸为 $224 \times 224$,在检测时将尺寸扩大
为 $448 \times 448$。

\subsubsection{Loss 函数}

YOLO v1 使用的 loss 函数为:
\begin{align}
  \label{equ:yolo-v1-loss}
  \begin{split}
    L = & \, \lambda_{\mathrm{coord}} \sum_{i=0}^{S^2} \sum_{j=0}^{B} \mathds{1}_{ij}^{\mathrm{obj}} \left [ \left (x_i - \hat{x}_i \right )^2 + \left (y_i - \hat{y}_i \right )^2 \right ] + \\
    & \, \lambda_{\mathrm{coord}} \sum_{i=0}^{S^2} \sum_{j=0}^{B} \mathds{1}_{ij}^{\mathrm{obj}} \left [ \left(\sqrt{w_i} - \sqrt{\hat{w}_i} \right)^2 + \left (\sqrt{h_i} - \sqrt{\hat{h}_i} \right )^2 \right ] + \\
    & \, \sum_{i=0}^{S^2} \sum_{j=0}^{B} \mathds{1}_{ij}^{\mathrm{obj}} \left( C_i - \hat{C}_i \right)^2 + \\
    & \, \lambda_{\mathrm{noobj}} \sum_{i=0}^{S^2} \sum_{j=0}^{B} \mathds{1}_{ij}^{\mathrm{noobj}} \left( C_i - \hat{C}_i \right)^2 + \\
    & \, \sum_{i=0}^{S^2} \mathds{1}_{i}^{\mathrm{obj}} \sum_{c \in \mathrm{classes}} \left( p_i(c) - \hat{p}_i(c) \right)^2
  \end{split}
\end{align}

\subsection{YOLO v2}
\label{subsec:YOLOv2}

\subsection{YOLO v3}
\label{subsec:YOLOv3}

\section{SSD 系列}
\label{sec:SSD}

\subsection{SSD}
\label{subsec:SSD}

\subsection{DSSD}
\label{subsec:DSSD}

\chapter{Anchor Free 方法}


%%% Local Variables:
%%% TeX-master: "../master"
%%% End:


\bibliographystyle{thubib}
\bibliography{refs}
\end{document}

%%% Local Variables:
%%% TeX-master: t
%%% End: