\documentclass[12pt, a4paper, UTF8, fontset=adobe, oneside]{ctexbook} % oneside 去掉所有空白页

\setmainfont{Times New Roman} % 英文用 Times New Roman 字体
\linespread{1.3} % 行距设置
\setcounter{secnumdepth}{3} % 层次为 3 以上的标题生成序号

%% 宏包
\usepackage{amsmath} % AMS数学宏包
\usepackage{fancyhdr} % 设置页眉页脚宏包
\usepackage{geometry} % 设置页边距宏包
\usepackage{xcolor} % 颜色宏包
\usepackage{hyperref} % 交叉引用宏包 colorlinks启用彩色模式 参考文献引用为紫红色
\usepackage[listings,breakable]{tcolorbox} % 彩色盒子宏包 代码宏包
\usepackage{enumitem} % 枚举设置宏包
\usepackage{tikz} % 画图宏包
\usepackage{booktabs} % 表格宏包
\usepackage{dsfont} % 数字粗体宏包

% 宏包设置
% 页眉页脚样式
\pagestyle{fancy} % 页面样式采用fancyhdr宏包中的fancy
\fancyhf{} % 去掉页眉
\cfoot{\thepage} % 页脚中间显示页码
\renewcommand{\headrulewidth}{0pt} % 去掉页眉的横线
% 页边距设置
\geometry{top = 2.54cm, bottom = 2.54cm, left = 3.18cm, right = 3.18cm}
% 章节样式设置
\CTEXsetup[name={第,章},number={\arabic{chapter}}]{chapter}
% 文档设置
\renewcommand\contentsname{目录} % 中文 目录
\renewcommand\bibname{参考文献} % 中文 参考文献
% 清华紫
\definecolor{THU}{RGB}{111, 23, 135}
% 交叉引用宏包设置
\hypersetup{colorlinks=true,linkcolor=THU,citecolor=THU}

% tcolorbox 样式设置
\newtcolorbox{redbox}[2][]{colback=yellow!10,colframe=red!75!black,coltitle=white,fonttitle=\bfseries,fontupper=\kaishu,title=#2,#1,breakable} % 红色
\newtcolorbox{RCbox}[2][]{colback=yellow!10,colframe=red!75!black,coltitle=white,fonttitle=\bfseries,fontupper=\kaishu,title=#2,#1,center
  title, center upper,breakable} % 红色居中
\newtcolorbox{magbox}[2][]{colback=yellow!10,colframe=magenta!75!black,coltitle=white,fonttitle=\bfseries,fontupper=\kaishu,title=#2,#1} % 紫红色
\newtcolorbox{THUbox}[2][]{colback=yellow!10,colframe=THU!75!black,coltitle=white,fonttitle=\bfseries,fontupper=\kaishu,title=#2,#1,breakable} % 紫罗兰色
\newtcolorbox{THUCbox}[2][]{colback=yellow!10,colframe=THU!75!black,coltitle=white,fonttitle=\bfseries,fontupper=\kaishu,title=#2,#1,center title,center upper,breakable} % 紫罗兰色 居中
\newtcolorbox{purbox}[2][]{colback=yellow!10,colframe=purple!75!black,coltitle=white,fonttitle=\bfseries,fontupper=\kaishu,title=#2,#1,center title,center upper} % 紫色
% \usetikzlibrary{calc,shapes.multipart,chains,arrows,positioning} % tikz library
% \tikzset{circarrow/.style={*->,shorten <=-2pt}}

% enumerate 样式设置
\setlist[enumerate]{label={\arabic*.},leftmargin=2.5em,align=left,topsep=0em,itemsep=-0.5em,labelsep=-0.5em,  before=\vspace{2pt},after=\vspace{2pt}}
\setlist[itemize]{leftmargin=2.5em,align=left,topsep=0em,itemsep=-0.5em,labelsep=-0.5em,  before=\vspace{2pt},after=\vspace{2pt}}

% 引用参考文献时提高位置
\newcommand{\citerb}[1]{\raisebox{1pt}{\cite{#1}}}

\begin{document}
\frontmatter
\begin{titlepage}
\begin{center}

\vspace*{5cm}
% Title
{\huge \bfseries 计算机视觉深度学习}\\[0.4cm]
{\huge \bfseries 技术栈知识总结}\\[0.4cm]

\vspace{12cm}

{\large 江浩} \\[1cm]
{\large \today}

\end{center}
\end{titlepage}

{
\hypersetup{linkcolor=black} % 目录链接为黑色
\pagenumbering{Roman} % 页码编号为大写罗马数字
\tableofcontents % 目录
}

\mainmatter % 正文部分 重新编号
\pagenumbering{arabic} % 页码编号为阿拉伯数字

\part{数学}
\chapter{概率}

\chapter{其他}

%\section{基础}
%\section{几何概率}
%\subsection{单位圆圆周任取两点的弦长}
%问题:单位圆的圆周上,任取两点,求两点之间弦长的期望。

%\begin{table}[htb]
%  \centering
%  \caption{主动学习(AL)模型性能与正常训练(Normal)模型性能对比}
%  \label{tab:AL-Normal}
%    \begin{tabular}{cccc}
%    \specialrule{0em}{10pt}{1pt}
%      \toprule[1.5pt]
%      {\heiti 数据集} & {\heiti AL 模型 mAP(35\%)} & {\heiti Normal 模型 mAP(100\%)} & AL/Normal \\ \midrule[1pt]
%      PASCOL VOC & 73.46 & 77.20 & 95.15\% \\
%      KITTI & 54.94 & 57.67 & 95.27\% \\
%      \bottomrule[1.5pt]
%    \end{tabular}
%\end{table}

%%% Local Variables:
%%% TeX-master: "../master"
%%% End:
\part{数据结构和算法}

\chapter{数据结构}

\section{栈}
栈是先进后出的基本数据结构,实际上程序在递归调用时都需要使用栈,但这种情况一般由
编译器隐式完成。

\subsection{括号题目}
括号是典型的满足先进后出特点的一种结构,所以和括号相关的题目一般都可以利用栈来解
决,例如
\href{https://leetcode.com/problems/minimum-add-to-make-parentheses-valid/}{LC
  921},
\href{https://leetcode.com/problems/minimum-remove-to-make-valid-parentheses/}{LC
  1249},
\href{https://leetcode.com/problems/longest-valid-parentheses/}{LC
  32},
\href{https://leetcode.com/problems/score-of-parentheses/}{LC
  856},
\href{https://leetcode.com/problems/reverse-substrings-between-each-pair-of-parentheses/}{LC 1190}~等。

\subsection{计算器题目}

\subsection{文本解析题目}

\href{https://leetcode.com/problems/decode-string/}{LC 394}。

\section{队列}

\section{优先队列(堆)}
优先队列的适用场合是在插入/删除频繁的数组中获得最小值,一个典型的实现是堆。优先
队列也是在难题中应用广泛的一种数据结构。

\section{链表}
链表相关问题一般都需要新建一个哨兵节点(Dummy node),哨兵节点的下一节点为当前链
表头节点,以简化边界处理。

\subsection{翻转}
\begin{itemize}
  \item
    \href{https://leetcode.com/problems/reverse-linked-list}{翻转整个链表(LC
    206)}。
  \item
    \href{https://leetcode.com/problems/reverse-linked-list-ii}{翻转部分链表(LC
    92)}。
  \item
    \href{https://leetcode.com/problems/reverse-nodes-in-k-group/}{每 k 个节点翻
      转(LC 25)}
\end{itemize}

\subsection{删除节点}
\begin{itemize}
  \item
    \href{https://leetcode.com/problems/reverse-linked-list}{删除有重复值的节
      点,只保留第一个(LC 83)}。
  \item
    \href{https://leetcode.com/problems/reverse-linked-list-ii}{删除有重复值的
      所有节点(LC 82)}。
\end{itemize}


\section{哈希表}

\section{树}

\subsection{一般二叉树}
\paragraph{树的遍历}
树的遍历包括深度优先和广度优先两种,其中深度优先遍历又包括前序(中、左、右)、中
序(左、中、右)、后序(左、右、中)三种。需要同时掌握递归和非递归两种写法。

\begin{itemize}
\item 树的 DFS 遍历:
  \href{https://leetcode.com/problems/binary-tree-preorder-traversal}
  {前序遍历(LC 144)},
  \href{https://leetcode.com/problems/binary-tree-inorder-traversal}
  {中序遍历(LC 94)}和
  \href{https://leetcode.com/problems/binary-tree-postorder-traversal}
  {后序遍历(LC 145)}。

\item 树的 BFS 遍历:
  \href{https://leetcode.com/problems/binary-tree-postorder-traversal}
  {层次遍历(LC 102)},相关问题
  \href{https://leetcode.com/problems/binary-tree-zigzag-level-order-traversal}
  {LC 103},
  \href{https://leetcode.com/problems/binary-tree-right-side-view}
  {LC 199}。

\item 根据遍历结果重构树:
  \href{https://leetcode.com/problems/construct-binary-tree-from-preorder-and-inorder-traversal}
  {前序+中序重构(LC 105)},
  \href{https://leetcode.com/problems/construct-binary-tree-from-inorder-and-postorder-traversal}
  {中序+后序重构(LC 106)},
  \href{https://leetcode.com/problems/construct-binary-tree-from-preorder-and-postorder-traversal}
  {前序+后序重构(LC 889)}。
\end{itemize}

\paragraph{典型问题}
树的典型问题一般可以利用递归解决,其中参数一般与父节点及祖先相关,返回值一般与子
节点及后代相关。
\begin{itemize}
\item 最低公共祖先(Lowest Common Ancestors,LCA)问
  题:
  包括
  \href{https://leetcode.com/problems/lowest-common-ancestor-of-a-binary-tree}
  {二叉树任意两节点(LC 236)},
  \href{https://leetcode.com/problems/lowest-common-ancestor-of-deepest-leaves}
  {二叉树最深节点(LC 1123)}以及
  \href{https://leetcode.com/problems/lowest-common-ancestor-of-a-binary-search-tree}
  {BST 两节点(LC 235)}。
  如果树中节点有指向父节点的指针,则可将 LCA 问题转化为两个链表的第一个公共节点问题。
\end{itemize}

\subsection{二叉搜索树}

二叉搜索树(Binary Search Tree,BST)是一种特殊的二叉树,整棵树满足父节点大于左
子节点且小于右子节点。

\paragraph{BST 的主要 API}
\begin{itemize}
  \item 查找
    (\href{https://leetcode.com/problems/search-in-a-binary-search-tree}{LC
      700})。
  \item 插入
    (\href{https://leetcode.com/problems/insert-into-a-binary-search-tree}{LC
      701})。
  \item 删除(\href{https://leetcode.com/problems/delete-node-in-a-bst}{LC 450})。
  \item 后继。包括给定根节点和值查找后继
    (\href{https://leetcode.com/problems/inorder-successor-in-bst}{LC 285}),
    和给定当前节点(带父指针)查找后继
    (\href{https://leetcode.com/problems/inorder-successor-in-bst-ii}{LC 510})。
  \item 前驱。与后继类似。
\end{itemize}

\paragraph{BST 的遍历性质}
\begin{itemize}
\item BST 中序遍历的性质:BST 中序遍历的结果为从小到大的有序数组,利用这个性质可
  以解决
  \href{https://leetcode.com/problems/validate-binary-search-tree}
  {LC 98},
  \href{https://leetcode.com/problems/binary-search-tree-iterator}
  {LC 173},
  \href{https://leetcode.com/problems/kth-smallest-element-in-a-bst}
  {LC 230}。
\item BST 前序、后序性质:根据 BST 前序或后序遍历结果可以直接重构 BST,
  \href{https://leetcode.com/problems/construct-binary-search-tree-from-preorder-traversal}
  {根据前序遍历结果重构 BST(LC 1008)},LC 上暂时还没有根据后序遍历结果重构 BST
  的题目。
\end{itemize}

\section{图}

\subsection{拓扑排序}
拓扑排序是指将一个有向无环图(Directed Acyclic Graph,DAG)G 中的所有顶点排成一个
线性序列,使得对任意一对顶点 $u, v$,若边 $(u, v) \in G$,则在序列中 $u$ 出现
在 $v$ 之前。相关题目包括
\href{https://leetcode.com/problems/course-schedule/}{LC 207},
\href{https://leetcode.com/problems/course-schedule-ii/}{LC 210},
\href{https://leetcode.com/problems/parallel-courses}{LC 1136},
\href{https://leetcode.com/problems/sequence-reconstruction}{LC 444},
\href{https://leetcode.com/problems/alien-dictionary}{LC 269},
\href{https://leetcode.com/problems/sort-items-by-groups-respecting-dependencies/}{LC 1203}。

\subsection{并查集}
并查集(Union Find,UF)可以解决图的动态连通性问题,这类问题也可以用深度优先搜索
(DFS)解决。并查集的 API 主要包括查找,判断连通性和合并,典型问题:

\begin{itemize}
  \item 
    \href{https://leetcode.com/problems/friend-circles}{图的连通分量数(LC 547)}
  \item
    \href{https://leetcode.com/problems/redundant-connection}{删除一条边使图变树
      (LC 684)}
\end{itemize}

\section{前缀树}

\chapter{算法}
\section{遍历}
\subsection{深度优先搜索}
% LC 39. Combination Sum
深度优先搜索(Depth First Search,DFS)的应用非常广泛,基于递归的方法相当于程序
实现 DFS + 回溯,此外在树和图的遍历中也有普遍应用。

\subsection{广度优先搜索}
广度优先搜索(Breath First Search,BFS)一般用在求最少次数/最短路径等问题中。

最少次数:
\begin{itemize}
  \item
    \href{https://leetcode.com/problems/remove-invalid-parentheses/}{LeetCode 301. Remove Invalid Parentheses}
\end{itemize}

最短路径:
\begin{itemize}
  \item
    \href{https://leetcode.com/problems/shortest-path-in-binary-matrix/}{LeetCode
      1091. Shortest Path in Binary Matrix}:八方向连通最短路径。
\end{itemize}

\subsection{最优优先搜索}
最优优先搜索是 DFS 和 BFS 的推广,一般通过维护最优堆进行搜索,在搜索时更新最优堆。
DFS 和 BFS 都可以看成是最优优先搜索的特例。相关问题:

\begin{itemize}
  \item
    \href{https://leetcode.com/problems/swim-in-rising-water/}{LeetCode 778. Swim in Rising Water}
  \item
    \href{https://leetcode.com/problems/path-with-maximum-minimum-value}{LeetCode
      1102. Path With Maximum Minimum Value}:与 LeetCode 778 相同。
  \item
    \href{https://leetcode.com/problems/trapping-rain-water-ii/}{LeetCode 407.
      Trapping Rain Water II}:维护最小堆,依次出堆判断四方向位置并入堆。
\end{itemize}


\section{并查集}

\section{动态规划}

\subsection{一维动态规划}
% LC 91. Decode ways

\subsection{二维动态规划}
二维动态规划的一个典型场景是字符串的公共/回文子串和编辑距离问题:
\begin{itemize}
  \item
    \href{https://leetcode.com/problems/longest-common-subsequence/}{LeetCode 1143. Longest Common Subsequence}
  \item
    \href{https://leetcode.com/problems/longest-palindromic-subsequence/}{LeetCode
      516. Longest Palindromic Subsequence}:可转化为求字符串与其自身倒序的 LCS,
    即转化为 LeetCode 1143。
  \item
    \href{https://leetcode.com/problems/palindrome-removal/}{LeetCode 1246. Palindrome Removal(Premium)}
  \item
    \href{https://leetcode.com/problems/edit-distance/}{LeetCode 72. Edit Distance}
  \item
    \href{https://leetcode.com/problems/one-edit-distance}{LeetCode 161. One
      Edit Distance}:与 LeetCode 72 类似,只需要判断是否为 1。
  \item
    \href{https://leetcode.com/problems/minimum-ascii-delete-sum-for-two-strings/}{LeetCode
      712. Minimum ASCII Delete Sum for Two Strings}:LeetCode 72 升级版,只允许
    删除操作,同时代价变为字符的 ASCII 值。
  \item
    \href{https://leetcode.com/problems/regular-expression-matching/}{LeetCode
      10. Regular Expression Matching}:分析 s[i:] 和 t[j:] 的匹配,单独考虑字符
    为 * 的情况。
\end{itemize}

\section{贪心}
贪心可以看做一种特殊的动态规划,即局部最优解一定是全局最优解,因此贪心算法不断选
择局部最优解以构造全局最优解。

\subsection{区间选择问题}
区间选择问题是贪心算法的经典应用场景,每个区间的权重均相同时才能应用贪心算法,若
不同则需要使用动态规划求解。

\begin{itemize}
  \item
    \href{https://leetcode.com/problems/meeting-rooms-ii}{LeetCode 253. Meeting
      Rooms II}
  \item
    \href{https://leetcode.com/problems/non-overlapping-intervals/}{LeetCode
      435. Non-overlapping Intervals}
  \item
    \href{https://leetcode.com/problems/minimum-number-of-arrows-to-burst-balloons/}{LeetCode
      452. Minimum Number of Arrows to Burst Balloons}
\end{itemize}

最小生成树 Kruskal 算法:
\begin{itemize}
  \item
    \href{https://leetcode.com/problems/optimize-water-distribution-in-a-village}{LeetCode
      1168. Optimize Water Distribution in a Village}:well 可转化为 0 号节点到
    所有节点的边,然后应用 Kruskal 算法,利用并查集判断是否连通。
\end{itemize}

\section{二分查找}
二分查找的典型应用场景包括:

\begin{itemize}
  \item 在有序序列(数组、矩阵等)中查找某个值。
  \item 在一定范围内查找满足某个条件的最大/最小值。
\end{itemize}

\subsection{有序序列查找}
二分查找的基本题型:

二分查找的变形题型:
\begin{itemize}
  \item
    \href{https://leetcode.com/problems/search-in-rotated-sorted-array/}{LeetCode
      33. Search in Rotated Sorted Array}:判断 left/right 和 mid 值的关系,得到
    有序部分,判断 target 是否在有序范围,重新查找。
\end{itemize}

\subsection{范围查找}
以下题目可以直接写出二分的判断条件:

\begin{itemize}
  \label{lc:bs-range-general}
  \item
    \href{https://leetcode.com/problems/split-array-largest-sum/}{LeetCode 410. Split Array Largest Sum}
  \item
    \href{https://leetcode.com/problems/koko-eating-bananas/}{LeetCode 875. Koko Eating Bananas}
  \item
    \href{https://leetcode.com/problems/capacity-to-ship-packages-within-d-days/}{LeetCode
      1011. Capacity To Ship Packages Within D Days}
  \item
    \href{https://leetcode.com/problems/divide-chocolate}{LeetCode 1231. Divide Chocolate}
\end{itemize}

以下题目需要结合一定的数学推导写判断条件:

\begin{itemize}
  \label{lc:bs-range-math}
  \item
    \href{https://leetcode.com/problems/nth-magical-number/}{LeetCode 878. Nth
      Magical Number}:容斥原理。
  \item
    \href{https://leetcode.com/problems/ugly-number-iii/}{LeetCode 1201. Ugly
      Number III}:LeetCode 878 升级版,两个数变为三个数。

\end{itemize}

\section{滑动窗口}

滑动窗口有时也叫双指针,一般是用 l 和 r 两个变量记录起始、结束位置,根据条件更新
结果或调整 l 和 r 的值。

两指针异向滑动,一般用于解决回文问题:
\begin{itemize}
  \item
    \href{https://leetcode.com/problems/longest-palindromic-substring/}{LeetCode
    5. Longest Palindromic Substring}
\end{itemize}

两指针同向滑动,典型问题是求满足条件的最长子串:
\begin{itemize}
  \item
    \href{https://leetcode.com/problems/longest-substring-without-repeating-characters/submissions/}{LeetCode
      3. Longest Substring Without Repeating Characters}
  \item
    \href{https://leetcode.com/problems/max-consecutive-ones-iii/}{LeetCode 1004. Max Consecutive Ones III}
  \item
    \href{https://leetcode.com/problems/longest-repeating-character-replacement/}{LeetCode
      424. Longest Repeating Character Replacement}:LeetCode 1004 升级版,0/1 变为 26 个大写字母。
  \item
    \href{https://leetcode.com/problems/longest-substring-with-at-most-two-distinct-characters}{LeetCode
      159. Longest Substring with At Most Two Distinct Characters(Premium)}
  \item
    \href{https://leetcode.com/problems/longest-substring-with-at-most-k-distinct-characters}{LeetCode
      340. Longest Substring with At Most K Distinct Characters(Premium)}:LeetCode 159
    升级版,2 个不同字母变为 k 个不同字母。
  \item
    \href{https://leetcode.com/problems/subarrays-with-k-different-integers}{LeetCode
    992. Subarrays with K Different Integers}:要求的是正好 k 个,可以转化为最多
  k 个 - 最多 k-1 个,类似 LeetCode 340。
  \item
    \href{https://leetcode.com/problems/get-equal-substrings-within-budget/}{LeetCode
      1208. Get Equal Substrings Within Budget}:和不超过 k 的最长子数组。
  \item
    \href{https://leetcode.com/problems/minimum-size-subarray-sum/}{LeetCode
      209. Minimum Size Subarray Sum}:和不小于 k 的最短子数组。
\end{itemize}

\section{前缀和}
前缀和(Prefix Sum)可以认为是线段树的简化版,在数组不频繁更新的情况下,可以利用
前缀和在 O(1) 时间,O(n) 空间的条件下获得数组在某一范围内的和。

前缀和 + 哈希相关题目:
\begin{itemize}
  \item
    \href{https://leetcode.com/problems/subarray-sum-equals-k/}{LeetCode 560. Subarray Sum Equals K}
  \item
    \href{https://leetcode.com/problems/count-number-of-nice-subarrays/}{LeetCode
      1248. Count Number of Nice Subarrays}
\end{itemize}

% 待分类
% 排列
% 31. Next Permutation
%
% DFS/BFS
% 1245. Tree Diameter

%%% Local Variables:
%%% TeX-master: "../master"
%%% End:

\part{机器学习和深度学习基础}

\chapter{基本概念}

\section{激活函数}

激活函数的作用是为神经网络引入非线性因素,提升网络的拟合能力。如果没有激活函数,
无论神经网络有多少层,输出总为输入的线性函数,因此网络的拟合能力将会十分有限。

\subsection{Sigmoid 函数}
\label{subsec:Sigmoid}

Sigmoid 函数及其导数的形式和图像如下:
\begin{align}
  \label{equ:sigmoid}
  \sigma(x) & = \frac{1}{1 + e^{-x}} \\
  \label{equ:sigmoid-d}
  \sigma'(x) & = \sigma(x) (1-\sigma(x))
\end{align}

\begin{figure}[ht]
  \centering
  \includegraphics[width=0.8\textwidth]{images/机器学习和深度学习基础/基本概念/Sigmoid.png}
  \caption{Sigmoid 函数及其导数的图像}
  \label{fig:sigmoid}
\end{figure}

\begin{itemize}
\item 图~\ref{fig:sigmoid}~中蓝色曲线为 Sigmoid 函数,其形状与字母 S 类似,可以
  将 $-\infty$ 到 $\infty$ 的输入单调映射到 0 到 1 之间。由
\item 图~\ref{fig:sigmoid}~中红色曲线为 Sigmoid 函数的导数,在输入值较大或较小
  时,Sigmoid 函数的导数趋近于 0,这个性质可能引发\hyperref[subsec:gradient-vanish-explosion]{梯度消失}问题,导致网络无法正常训练。
\end{itemize}

\subsection{线性整流函数(ReLU)}

线性整流函数(Rectifier Linear Unit,ReLU)是目前应用广泛的激活函数,其形式为:
\begin{equation}
  \label{equ:ReLU}
  \mathrm{ReLU}(x) = \left\{
    \begin{array}{lr}
      x & x > 0 \\
      0 & x \leq 0 \\
    \end{array}
  \right.
\end{equation}

ReLU 的图像很简单,如图~\ref{fig:ReLU}~所示。由图中可看出,ReLU 在 $x > 0$ 时对
应的导数始终为 1,不会导致梯度消失;但当 $x \leq 0$ 时梯度始终为 0,同样会导致梯度消失。

\begin{figure}[ht]
  \centering
  \includegraphics[width=0.5\textwidth]{images/机器学习和深度学习基础/基本概念/ReLU.png}
  \caption{ReLU 的图像}
  \label{fig:ReLU}
\end{figure}

\section{损失函数}

\subsection{交叉熵损失函数}

\subsection{Softmax 损失函数}

\subsection{$L_1$ 损失函数}

\subsection{$L_2$ 损失函数}

\subsection{Smooth $L_1$ 损失函数}
Smooth $L_1$ 损失函数是 $L_1$ 和 $L_2$ 损失函数的组合,其形式为:
\begin{equation}
  \label{equ:SmoothL1}
  \mathrm{SL}_1(x) = \left\{
    \begin{array}{lr}
      |x| & |x| > \alpha \\
      \frac{1}{\alpha} x^2 & |x| \leq \alpha \\
    \end{array}
  \right.
\end{equation}

由上式可知,Smooth $L_1$ 损失函数在 $x$ 绝对值较小时为 $L_2$ 函数,而在 $x$ 绝对
值较大时为 $L_1$ 函数。在目标检测的\hyperref[sec:R-CNN]{两阶段方法}中,回归 loss
一般使用 Smooth $L_1$ 损失函数。

\subsection{Focal loss}

文献\citerb{2017-RetinaNet}提出 focal loss,用于解决目标检测中正负样本不平衡的问
题,该 loss 在存在样本不平衡的其他问题中也有不错的表现。Focal loss 通过降低简单样
本的权重,在一定程度上相当于难样本挖掘(Hard Negative Mining,HEM)。

Focal loss 的形式和图像如公式~\ref{equ:focal-loss}~和图~\ref{fig:focal-loss}~所示:
\begin{equation}
  \label{equ:focal-loss}
  \mathrm{FL}(p_{\mathrm{t}}) = - \alpha_t (1-p_{\mathrm{t}})^{\gamma} \mathrm{log}(p_{\mathrm{t}})
\end{equation}

\begin{figure}[ht]
  \centering
  \includegraphics[width=0.8\textwidth]{images/机器学习和深度学习基础/基本概念/Focal-Loss.pdf}
  \caption{Focal loss 的图像($\alpha_t = 1$)}
  \label{fig:focal-loss}
\end{figure}

图~\ref{fig:focal-loss}~中,$\gamma = 0$ 的蓝色曲线即传统的交叉熵 loss,可以认为
是 focal loss 的一个特例。其他颜色的曲线分别表示 $\gamma$ 取不同数值时的 focal
loss。Focal loss 中两个参数的具体作用是:

\begin{itemize}
  \item $\alpha_t$:用于平衡不同类别样本的权重,一般正负样本对应的 $\alpha_t$ 之
    和为 1。
  \item $\gamma$:用于平衡难易样本的权重,$\gamma$ 越大,简单样本的权重越小。例
    如 $\gamma = 2$ 时,预测值 $p = 0.9$ 的样本,focal loss 的值要比交叉熵 loss
    的值小 100 倍。
\end{itemize}

\section{优化方法}
\label{sec:opt}

\subsection{梯度下降(GD)和随机梯度下降(SGD)}
梯度下降方法属于一阶优化算法,利用函数的一阶导数信息,即梯度是函数值上升最快的方
向,因此选择负梯度方向即函数值下降最快的方向更新参数:
\begin{equation}
  \boldsymbol{\theta}_t = \boldsymbol{\theta}_{t-1} - \eta \cdot \nabla_{\boldsymbol{\theta}}J(\boldsymbol{\theta})
\end{equation}

根据上式中梯度计算时使用样本数的不同,梯度下降法又可分为:
\begin{itemize}
  \item 批梯度下降:使用全部样本计算梯度。
  \item 随机梯度下降:使用单个样本计算梯度。
  \item Mini-batch 梯度下降:使用部分样本(例如 8、16、32 个样本)计算梯度。目前
    应用最为广泛,有时也被称为随机梯度下降。
\end{itemize}

\subsection{带动量的随机梯度下降}
带动量的随机梯度下降的作用是训练时,梯度方向相同时加速,梯度改变方向时减速:
\begin{align}
  v_t & = \gamma v_{t-1} + \eta \cdot \nabla_{\boldsymbol{\theta}}J(\boldsymbol{\theta}) \\ 
  \boldsymbol{\theta}_{t} & = \boldsymbol{\theta}_{t-1} - v_t
\end{align}

\subsection{Nesterov 方法}
Nesterov 方法是对带动量的随机梯度下降的改进,计算梯度时不用 $\theta$ 而用
$\theta - \gamma_{t-1}$:
\begin{align}
  v_t & = \gamma v_{t-1} + \eta \cdot \nabla_{\boldsymbol{\theta}}J(\boldsymbol{\theta}-\gamma v_{t-1}) \\ 
  \boldsymbol{\theta}_{t} & = \boldsymbol{\theta}_{t-1} - v_t
\end{align}

\subsection{Adagrad 方法}
Adagrad 是对随机梯度下降的改进,将参数对应梯度除以历史梯度累积和的开方:
\begin{equation}
  \boldsymbol{\theta}_{t} = \boldsymbol{\theta}_{t-1} - \frac{\eta \cdot \nabla_{\boldsymbol{\theta}}J(\boldsymbol{\theta})}{G_t + \epsilon}
\end{equation}

\subsection{Adadelta 方法}
Adagrad 方法中,分母上的历史梯度累计和会越来越大,导致当前更新值很小。Adadelta
方法将历史梯度值的计算改为类似 Nesterov 的形式:
\begin{align}
  g_t & = \nabla_{\boldsymbol{\theta}}J(\boldsymbol{\theta_t}) \\
  E[g^2]_t & = \gamma E[g^2]_{t-1} + (1-\gamma)g_t^2 \\
  \boldsymbol{\theta}_{t} & = \boldsymbol{\theta}_{t-1} - \frac{\eta \cdot g_t}{E[g^2]_t+\epsilon}
\end{align}

RMSProp 方法和 Adadelta 方法类似。

\subsection{Adam 方法}


\section{反向传播算法}
神经网络的训练实际上是一个不断调整网络参数,从而减小损失函数的过程。对网络参数的
优化需要先得到损失函数 $L$ 对神经网络中各参数 $\omega$ 的导数,利用该值计算参
数 $\omega$ 的调整值。反向传播算法利用链式法则,由后向前逐层计算 $L$ 对各参
数$\omega$ 的导数,是神经网络训练的一个关键环节。

考虑一个神经网络中的某个全连接层 $Y = XW + b$,设网络的 loss 为 $L$,根据链式法
则,有如下关系~\citerb{2017-FC-BP}:

\begin{align}
  \label{equ:bp-fc}
  \frac{\partial L}{\partial X} & = \frac{\partial L}{\partial Y} \frac{\partial Y}{\partial X} = \frac{\partial L}{\partial Y} W^{\mathrm{T}} \\
  \frac{\partial L}{\partial W} & = \frac{\partial L}{\partial Y} \frac{\partial Y}{\partial W} = X^{\mathrm{T}} \frac{\partial L}{\partial Y}\\
  \frac{\partial L}{\partial b} & = \frac{\partial L}{\partial Y} \frac{\partial Y}{\partial b} = \frac{\partial L}{\partial Y}
\end{align}

上面三个式子中,$\frac{\partial L}{\partial X}$ 是为了继续进行反向传播,因为上一
层参数导数的计算依赖该值;$\frac{\partial L}{\partial W}$ 和 $\frac{\partial
  L}{\partial b}$ 即更新参数时需要的梯度。前向计算完成后,逐层向后计算 loss 对每
一层输入($X$)和参数($W, \, b$)的梯度。

\subsection{梯度消失和梯度爆炸}
\label{subsec:gradient-vanish-explosion}
根据链式法则,对于深层神经网络,浅层参数的梯度与之后所有层输出对输入梯度的乘积
相关。直观理解,如果梯度都小于 1,相乘后的值很小,会导致梯度消失;反之如果梯度都大
于 1,相乘后的值很大,会导致梯度爆炸。

梯度消失的一个典型原因是激活层使用了 Sigmoid 函数,具体原因请
见~\ref{subsec:Sigmoid}小节中的分析。

梯度爆炸的一个简单的解决方法是梯度裁剪(Gradient Clipping),即设定一个阈值,当梯
度值大于阈值时,使其等于阈值,这样即可避免梯度过大。

\section{归一化}

\subsection{批归一化(BN)}
\label{sub:BN}

批归一化(Batch Normalization,BN)是文献\citerb{2015-BN}中提出的一种归一化方法,
BN 的作用包括:

\begin{enumerate}
  \item 降低深层神经网络的训练难度。训练时可以使用更大的学习率,同时不需要特别谨
    慎地对网络参数进行初始化。
  \item 防止过拟合。BN 一定程度上相当于引入了正则项,网络中加入 BN 后可以去
    掉 dropout 层,\hyperref[subsec:YOLOv2]{YOLOv2} 就是一个典型的例子。
  \item 加快神经网络训练速度。加入 BN 后网络的训练时间减少 10 倍以上。
\end{enumerate}

文献\citerb{2015-BN}中提出 BN 有效的原因是其减少了神经网络的内部协同变化
(Internal Covariate Shift,ICS),但有文献论证 BN 并不能减少 ICS,其提升性能的原
因是使 loss 变得更平滑\citerb{2018-BN-help-opt}。

\paragraph{BN 的形式}
BN 包括归一化和线性变换两部分。其中归一化是为了减少上层网络参数变化导致对输入的影
响;线性变换部分是为了增加 BN 层的表达能力,如果没有线性变换部分,BN 层的输出将始
终为均值为 0,方差为 1 的分布,无法体现上层网络参数变化对其的影响。
\begin{align}
  \label{equ:BN}
  \mu_{\mathcal{B}} & = \frac{1}{m} \sum_{i=1}^{m} x_i \\
  \sigma_{\mathcal{B}}^2 & = \frac{1}{m} \sum_{i=1}^{m} (x_i-\mu_{\mathcal{B}})^2 \\
  \hat{x}_i & = \frac{x_i - \mu_{\mathcal{B}}}{\sqrt{\sigma_{\mathcal{B}}^2 + \epsilon}} \\
  y_i & = \gamma \hat{x}_i + \beta
\end{align}

\begin{itemize}
\item 上式中 $m$ 为一个 batch 的样本数,即期望和方差计算的都是一个 batch 的统计量。
  实际应用中,训练时可以采用滑动平均的方法计算均值和方差,即同时考虑历史 batch 和
  当前 batch 的统计量,例如 Caffe 中的
  \href{https://bit.ly/2JBY7aw}{BN层}。测试时,\textbf{使用训练时保存的均值和方
    差},不需要每个 batch 都重新计算。
  \item BN 层中有两个需要学习的参数,即线性变换的参数 $\gamma$ 和 $\beta$,这两
    个参数的初值分别为 1 和 0。
\end{itemize}

\paragraph{CNN 中的 BN}
\begin{itemize}
  \item 卷积层之后 BN 均值和方差的计算方法:计算的是一个 batch 中 $m$ 个样本对应的
    同一个 feature map 上所有点的均值和方差。如果一个 batch 对应的卷积层的输出维度
    为 $m \times C \times H \times W$,则只计算通道数 $C$ 个均值和方差,不同
    的 feature map 利用对应的均值和方差进行归一化。
  \item BN 的位置:文献~\citerb{2015-BN}~中给出的结构是\textbf{卷积层-BN-ReLU},
    即 BN 层加在卷积层之后,ReLU 之前。
  \item 卷积层的偏置 $b$:由于 BN 中包含减均值的操作,因此可以省略卷积层的偏置。
  \item 测试时将卷积层和 BN 层进行合并:由于卷积层和 BN 层都是线性变换,且测试
    时 BN 层使用的均值和方差均为训练时保存的结果,因此可以将二者合并以提升计算速
    度,合并后参数的计算方法可以参考文献\citerb{2017-DSSD}。
\end{itemize}

\subsection{组归一化(GN)}
\label{sub:GN}

文献\citerb{2018-GN}中提出了组归一化(Group Normalization,GN)方法,主要解决 BN
在 batch 样本较少时性能不佳的问题。

\paragraph{GN 的形式}
不同归一化方法的对比如图~\ref{fig:all-norm}~所示:
\begin{figure}[ht]
  \centering
  \includegraphics[width=0.8\textwidth]{images/机器学习和深度学习基础/基本概念/GN.pdf}
  \caption{不同归一化方法对比}
  \label{fig:all-norm}
\end{figure}

\begin{itemize}
  \item BN:一个 batch 内不同样本,对同一个通道的 feature map 做归一化,共有通道
    数 $C$ 个均值和方差。
  \item LN:同一样本,对所有通道的 feature map 做归一化,共有 batch 样本数 $N$
    个均值和方差。
  \item GN:同一样本,对按组划分通道的 feature map 做归一化,共有 batch 样本数乘
    以组数 $N \times G$ 个均值和方差。一般 $G$ 取 32,LN 可看做 GN 中 $G=1$ 的特
    例。
\end{itemize}

\chapter{卷积神经网络}

\section{卷积类型}

\subsection{分组卷积}

分组卷积(Group Convolution,GC)在 AlexNet 中就已经有应用,但当时的背景是 GPU 计
算能力不足,只能将标准卷积拆成 2 个分组卷积~\citerb{2012-AlexNet}。分组卷积和标准
卷积的对比如图 \ref{fig:normal-conv} 和图 \ref{fig:group-conv} 所
示~\citerb{2019-Diff-Conv}:

\begin{figure}[ht]
  \centering
  \includegraphics[width=0.8\textwidth]{images/机器学习和深度学习基础/卷积神经网络/标准卷积.png}
  \caption{标准卷积}
  \label{fig:normal-conv}
\end{figure}

\begin{figure}[ht]
  \centering
  \includegraphics[width=0.8\textwidth]{images/机器学习和深度学习基础/卷积神经网络/分组卷积.png}
  \caption{分组卷积}
  \label{fig:group-conv}
\end{figure}

分组卷积的具体步骤是:

\begin{enumerate}
  \item 将输入按通道拆分成 $g$ 组,每组数据的尺寸与输入相同,均为 $H_{\mathrm{in}}
    \times W_{\mathrm{in}}$,通道数均为输入的 $ 1/g $,即 $
    C_{\mathrm{g}} = C_{\mathrm{in}} / g $。
  \item 分别对拆分后的 $g$ 组数据进行标准卷积,每一组数据对应的输出的通道数
    均为 $ C_{\mathrm{out}}/g $。
  \item 将 $g$ 组数据的输出结果进行拼接,得到通道数为 $C_{\mathrm{out}}$
    的输出。
\end{enumerate}

分组卷积可以节省计算量,因此在轻量型网络中得到广泛应用。参数相同的情况下,分组卷
积的计算量约为标准卷积的 $ 1 / g^2 $,组数越多,计算量越小。

分层卷积(Depthwise Convolution,DW)是分组卷积的一种特例,指的是组数 $g$ 与 输
入数据通道数 $C_{\mathrm{in}}$ 相同的卷积,即每组卷积的输入只有 1 个通道。

\subsection{深度可分离卷积}

深度可分离卷积(Depthwise Separable Convolution,DSC)如图~\ref{fig:ds-conv}~所
示~\citerb{2019-Diff-Conv}:

\begin{figure}[ht]
  \centering
  \includegraphics[width=0.8\textwidth]{images/机器学习和深度学习基础/卷积神经网络/深度可分离卷积.png}
  \caption{深度可分离卷积}
  \label{fig:ds-conv}
\end{figure}

深度可分离卷积的步骤包括分层卷积和逐点卷积两部分,计算量为标准卷积计算量的 $ 1 /
C_{\mathrm{in}} + 1/k^2 $,$k$ 为分层卷积的卷积核大小。
当通道数 $C_{\mathrm{in}}$ 较大时,上式可近似为 $1 / k^2$~\citerb{2017-MobileNet-v1}。

\subsection{反卷积}
\label{subsec:deconv}

反卷积(Deconvolution),有时也称转置卷积(Transposed Convolution),是一种特殊的卷积
形式。反卷积的具体步骤是,先通过补 0 扩大输入图像的尺寸,再进行标准卷积。反卷积的
作用是将卷积后尺寸较小的 feature map 恢复到与输入大小相同的尺寸。 需要特别说明的
是,反卷积虽然可以恢复尺寸,\textbf{但不能保证恢复后的数值与输入相同}~\citerb{2016-Guide-Conv}。

\section{感受野}

感受野(Receptive Field,RF) 是指卷积神经网络中某一层的一个像素,会受到多少个输
入数据像素的影响。直观上理解,就是该像素能看到多大范围的输入层像
素~\citerb{2017-Guide-to-RF-cal}。具体的计算方法为:
\begin{align}
\label{equ:rf-cal}
j_{\mathrm {out}} & = j_{\mathrm{in}} \times s \\
r_{\mathrm {out}} & = r_{\mathrm{in}} + (k-1) \times j_{\mathrm{in}}
\end{align}

其中 $j$ 为 jump,相当于每一层的 stride;s 为卷积的 stride,k 为卷积的 kernel
size,r 即感受野。利用 \eqref{equ:rf-cal} 即可计算各层的感受野。

经典网络 VGG-16 各层感受野的详细计算可以参考~\citerb{2018-VGG-16-RF-cal}。

%%% Local Variables:
%%% TeX-master: "../master"
%%% End:

\part{编程语言}
\chapter{C++}

\section{基础}
\subsection{基本类型}
\subsubsection{指针和引用的区别}
\begin{enumerate}
  \item 可否为空:有空指针(C++11 标准推荐\texttt{nullptr}),但没有空引用,引用必须初始化。
  \item 可否改变指向的对象:指针可以改变,引用初始化之后不能改变。
\end{enumerate}

\subsection{关键字}
\subsubsection{static}
static 关键字主要用于以下四种情况:

\begin{itemize}
  \item 静态局部变量
  \item 静态全局变量
  \item 静态数据成员
  \item 静态成员函数
\end{itemize}

\paragraph{静态局部变量}
静态局部变量指在函数中利用 static 关键字声明的局部变量。

\begin{itemize}
  \item 初始化:程序首次执行到对象声明时进行初始化,后续不再初始化。如果没有显式
    初始化,将进行值初始化,内置类型变量初值为 0。
  \item 存储位置:存储在全局数据区,而非像函数内的局部变量存储在栈内。
  \item 作用域:虽然存储在全局数据区,但作用域仍然为局部作用域,只能在对应的函数
    内使用。
\end{itemize}

\paragraph{静态全局变量}
静态全局变量指在函数外利用 static 关键字声明的变量。

\begin{itemize}
  \item 初始化:程序执行前初始化。如果没有显式初始化,将进行值初始化,内置类型变量初值为 0。
  \item 存储位置:存储在全局数据区。
  \item 作用域:只能在声明静态全局变量的文件中使用,其他文件中不能使用,但可以定
    义相同名称的变量。
\end{itemize}

C++ 中,通常通过在匿名命名空间中声明变量实现静态全局变量,不再使用 static 关键字。

\paragraph{静态数据成员}
静态数据成员指类中利用 static 关键字声明的数据成员。

\begin{itemize}
  \item 初始化:只能在类内声明,类外定义并初始化。如果没有显式初始化,将进行值初始
    化,内置类型变量初值为 0。
  \item 存储位置:存储在全局数据区,类的所有对象共享同一个静态数据成员。
  \item 作用域:程序执行的整个过程。
\end{itemize}

\paragraph{静态成员函数}
静态成员函数指类中利用 static 关键字声明的成员函数。静态成员函数只能访问类的静态
数据成员和调用其他静态成员函数,不能访问非静态成员和调用非静态成员函数,也没有
this 指针。

\subsection{变量}
\subsubsection{作用域}
C++ 中,根据作用域的不同,可以将对象分为局部对象、全局对象和动态对象。

\begin{table}[htbp]
  \centering
  \caption{C++ 的变量作用域}\label{tab:cpp-variable-scope}
  \begin{tabular}{cccc}
    \specialrule{0em}{10pt}{1pt}
    \toprule[1.5pt]
    {\heiti{名称}} & {\heiti{作用域}} & {\heiti{相关关键字}} & {\heiti{存储区域}} \\
    \midrule[1pt]
    局部对象       & 函数/块作用域内    & 无         & 栈 \\
    静态全局对象   & 所在文件           & static     & 静态存储区 \\
    全局对象       & 整个程序           & 无         & 静态存储区 \\
    动态对象       & 申请到释放内存期间 & new/delete & 堆 \\
    \bottomrule[1.5pt]
  \end{tabular}
\end{table}

\subsubsection{初始化}
C++ 中定义变量时如果没有指定初值,则会根据变量类型和位置进行对应的初始化。

\begin{itemize}
  \item 内置类型:非静态局部变量将不被初始化,拥有未定义的值;静态局部变量和全局
    变量进行值初始化,初值为 0。
  \item 类类型:与位置无关,统一调用默认构造函数生成对象,如果类不支持默认初始化,
    要求显式提供初值,将会出现编译错误。
\end{itemize}

\section{面向对象编程}
\subsection{面向对象编程的三大基本特征}

\subsubsection{封装}
封装是指将对象抽象成具体的类,同时进行接口和实现的分离,外界只能通过接口与类的对
象交互,将内部的信息隐藏。

\subsubsection{继承}
继承是指子类继承父类的数据成员和函数,并根据需要进行扩展或重写,无需完全重写新类,
实现代码的复用。

\subsubsection{多态}
多态是指通过将父类的指针或引用指向子类对象,在运行时根据所指对象类型实现对应功能,
即同一个父类指针通过指向不同的对象,可以表现出不同形态。


\chapter{Python}


%%% Local Variables:
%%% TeX-master: "../master"
%%% End:

\part{目标检测}
\chapter{基本概念}

\section{指标}

\subsection{交并比}
交并比(Intersection over Union,IoU)

\section{非极大值抑制(NMS)}

非极大值抑制(Non maximum suppression,NMS)是一种后处理方法,其作用是保证一个待
检测物体只有一个检测框与之对应,更直观地理解其实是极大值保留,即只保留(置信度)
是极大值的检测框。图~\ref{fig:nms}~给出了一个具体示例\citerb{2018-NMS}:

\begin{figure}[ht]
  \centering
  \includegraphics[width=0.8\textwidth]{images/目标检测/NMS.png}
  \caption{NMS 示例}
  \label{fig:nms}
\end{figure}

\subsection{传统 NMS}

传统 NMS 算法的具体流程是,先根据分数对所有检测框进行排序,然后从分高到分低遍历所
有检测框,如果高分检测框与低分检测框的 IoU 大于一定阈值,则将低分检测框删掉,后续
遍历时不再处理,如图~\ref{fig:nms-algo}~中的红框。

\begin{figure}[ht]
  \centering
  \includegraphics[width=0.5\textwidth]{images/目标检测/Soft-NMS.pdf}
  \caption{NMS 和 Soft NMS 算法流程}
  \label{fig:nms-algo}
\end{figure}

\subsection{Soft NMS}

传统 NMS 算法需要选择合适的两个检测框的 IoU 阈值,否则阈值太低会导致误检,而阈值
太高又会导致漏检。Soft NMS 的基本思想是根据两个检测框的 IoU,降低低分检测框的分
数,但并不直接将其删除,具体形式如图~\ref{fig:nms-algo}~中的绿框~\citerb{2017-Soft-NMS}。

\subsection{Softer NMS}
文献~\citerb{2018-Softer-NMS}~中在 Soft NMS 的基础上提出了 Softer NMS,具体流程
如图~\ref{fig:softer-nms}~所示:

\begin{figure}[ht]
  \centering
  \includegraphics[width=0.5\textwidth]{images/目标检测/Softer-NMS.pdf}
  \caption{Softer NMS 算法流程}
  \label{fig:softer-nms}
\end{figure}

由图~\ref{fig:softer-nms}~可知,Softer NMS 在 Soft NMS 基础上,加入了 var voting
的部分,其具体步骤是:

\begin{enumerate}
  \item 对任一检测框,计算所有分数低于该框的检测框与其的 IoU。
  \item 根据 IoU 和每个检测框的不确定度 $\sigma$,修正该检测框的位置,其中 IoU
    越大,$ \sigma $ 越小,则相应的权重越高。
\end{enumerate}

\chapter{两阶段方法}
\section{R-CNN 系列}
\label{sec:R-CNN}

\subsection{Fast R-CNN}
\label{subsec:Fast-R-CNN}

Fast R-CNN 在训练和测试时,只需要利用 CNN 将整张图片前传一次,不需要像 R-CNN 中
将所有 proposal 对应的图片区域先 resize 再进行前传,因此可以大大提升训练和测试速
度\citerb{2015-Fast-RCNN}。

\paragraph{网络结构} 

Fast R-CNN 的网络结构如~\ref{fig:Fast-RCNN}~所示,前面是统一的 CNN backbone,对于
每个 proposal,将其投射到 feature map 再经过 RoI pooling 层得到尺寸为 $h \times
w$ 的 RoI,再经过两个全连接层得到 RoI feature,再经过全连接层 + Softmax 层的分类
器得到分类分数,同时 RoI feature 也会经过全连接层得到回归变换值。

\begin{figure}[ht]
  \centering
  \includegraphics[width=0.8\textwidth]{images/目标检测/Fast-RCNN.pdf}
  \caption{Fast R-CNN 网络结构}
  \label{fig:Fast-RCNN}
\end{figure}

\paragraph{RoI Pooling 层}

RoI 即 Region of Interest,中文为感兴趣区域。

\paragraph{Loss 函数}

\paragraph{正负样本}

\paragraph{训练}

\subsection{Faster R-CNN}
\label{subsec:Faster-R-CNN}

\chapter{单阶段方法}

\section{YOLO 系列}
\label{sec:YOLO}

\subsection{YOLO v1}
\label{subsec:YOLOv1}
YOLO v1 是目标检测单阶段方法的开创性工作之一,文中将检测问题刻画为回归问题,没有
任何分类的部分\citerb{2015-YOLO-v1}。

\paragraph{检测方法}

\begin{enumerate}
  \item 将整张图片划分为 $S \times S$ 个网格,每个网格预测 $ B $ 个检测框和 $ C $
  个条件概率 $ \mathrm{Pr}(\mathrm{Class_i}|\mathrm{Object}) $。
  \item 每个检测框共包括 5 个值 $x, y, w, h$ 和 confidence。其中 $x, y$为中心位
  置,$w, h$ 为长和宽(相对图片而言),confidence 为$\mathrm{Pr}(\mathrm{Object})
  \times \mathrm{IoU}^{\mathrm{truth}}_{\mathrm{pred}}$,inference 时直接
  将 confidence乘以该网格对应的条件概率得到最终的分数。网络的输出维度为 $ S
  \times S \times (B \times 5 + C) $。
\end{enumerate}

\paragraph{Backbone}

YOLO v1 的 backbone 没有采用改造后的经典分类网络,而是采用结构为 24 个卷积层 + 2
个全连接层的自定义网络。训练时,先将前 20 个卷积层 + 平均池化层组成的网络
在 ImageNet 数据集上进行预训练,输入尺寸为 $224 \times 224$,在检测时将输入尺寸扩
大为 $448 \times 448$。与此同时,为避免过拟合,第一个全连接层后加了一个 $p=0.5$
的 dropout 层。

\paragraph{Loss 函数}

\begin{align}
  \label{equ:yolo-v1-loss}
  \begin{split}
    L = & \, \lambda_{\mathrm{coord}} \sum_{i=0}^{S^2} \sum_{j=0}^{B} \mathds{1}_{ij}^{\mathrm{obj}} \left [ \left (x_i - \hat{x}_i \right )^2 + \left (y_i - \hat{y}_i \right )^2 \right ] \\
    & \, + \lambda_{\mathrm{coord}} \sum_{i=0}^{S^2} \sum_{j=0}^{B} \mathds{1}_{ij}^{\mathrm{obj}} \left [ \left(\sqrt{w_i} - \sqrt{\hat{w}_i} \right)^2 + \left (\sqrt{h_i} - \sqrt{\hat{h}_i} \right )^2 \right ]  \\
    & \, + \sum_{i=0}^{S^2} \sum_{j=0}^{B} \mathds{1}_{ij}^{\mathrm{obj}} \left( C_i - \hat{C}_i \right)^2  \\
    & \, + \lambda_{\mathrm{noobj}} \sum_{i=0}^{S^2} \sum_{j=0}^{B} \mathds{1}_{ij}^{\mathrm{noobj}} \left( C_i - \hat{C}_i \right)^2  \\
    & \, + \sum_{i=0}^{S^2} \mathds{1}_{i}^{\mathrm{obj}} \sum_{c \in \mathrm{classes}} \left( p_i(c) - \hat{p}_i(c) \right)^2
  \end{split}
\end{align}

\begin{enumerate}
  \item Loss 函数共包括三部分:前两项是第一部分,是检测框位置的 loss;中间两项是
    第二部分,是检测框分数的 loss;最后一项是第三部分,是网格类别概率的 loss。所
    有的 loss 均为 $L_2$ 形式。
  \item 检测框位置 loss 权重 $\lambda_{\mathrm{coord}}$ 提高到 5,同时将不含 GT 的
    网格中检测框分数 loss 的权重 $\lambda_{\mathrm{noobj}}$ 降为 0.5。
  \item 只有与 GT IoU 最大的检测框才会产生检测框位置 loss,只有网格中包含 GT 中
    心才产生网格类别概率 loss。
\end{enumerate}

\subsection{YOLO v2}
\label{subsec:YOLOv2}
YOLO v2 在 YOLO v1 的基础上做了一系列改进\citerb{2016-YOLO-v2}:

\begin{enumerate}
  \item 引入 BN:所有卷积层都加入 BN 层,由于 BN 层有正则化作用,因此可以去掉
    dropout 层。
  \item 改变预训练尺寸:YOLO v1 中预训练时图片输入尺寸为 $224 \times 224$,而检
    测时图片的输入尺寸为 $448 \times 448$,v2 中将预训练时的输入尺寸也改为 $448
    \times 448$,但网络实际的输入尺寸为 $416 \times 416$。
  \item 引入 anchor:参考 Faster R-CNN \cite{2015-Faster-RCNN},YOLO v2 中同样引
    入 anchor 的概念,即网络的输出为 anchor 到 GT 的变换,同时每个 anchor 分别预
    测类别概率和前景分数,而不是一个网格只预测一个类别概率,这样就可以避免 v1 中
    每个网格只能预测一个类别的问题。
  \item Anchor 尺寸聚类:利用 k-means 算法将 GT 聚类,得到 k 个 anchor 尺寸的先
    验。聚类时距离定义为 $d = 1 - \mathrm{IoU}(\mathrm{box}, \mathrm{centroid})$。
  \item 预测相对位置:YOLO v2 输出的检测框的坐标是相对网格的偏差,而非与
    Faster R-CNN 中相同的变换,这样可以将 anchor 的中心限制在该网格内。在网络输
    出后加入 sigmoid 函数即可实现将任意输入压缩至 0 到 1 之间。
  \item 特征融合:将尺寸为 $26 \times 26$ 的 feature map 与尺寸为 $13 \times 13$
    的 feature map 进行融合,具体方法是将 $26 \times 26 \times 512$ 的 feature
    map 变换为 $13 \times 13 \times 2048$,再和 $13 \times 13$ 的 feature map 拼
    接。
  \item 多尺度训练:训练时采用多尺度训练,即图片输入尺寸为 320 到 608 之间,步长
    为 32 的随机数。
  \item 新的 backbone:采用新的 backbone Darknet-19,包含 19 个卷积层和 5 个最大
    池化层。
\end{enumerate}

\subsection{YOLO v3}
\label{subsec:YOLOv3}
YOLO v3 在 YOLO v2 的基础上做了部分改进\citerb{2018-YOLO-v3}:

\begin{enumerate}
  \item Anchor loss 的计算:每一个 GT 分配与其 IoU 最大的 anchor,该 anchor 对应的
    前景分数为 1,如果不是 IoU 最大的 anchor 且与某个 GT 的 IoU 大于 0.5,则
    该 anchor 会被忽略,即不产生前景分数 loss。同时,没有 GT 匹配的 anchor 也不产生
    检测框位置和类别 loss,只计算前景分数 loss。
  \item Loss 类型:将 Softmax loss 改为多个 sigmoid loss,一个 anchor 可以同时对
    应多个类别。
  \item 多尺度特征融合:采用类似 FPN\cite{2016-FPN} 的结构,将深层小尺寸 feature
    map 进行上采样后,与浅层大尺寸 feature map 相加,作为新的 feature map,文
    中共有 3 个尺度。
  \item Anchor 按尺度分配:YOLO v3 中共有 9 种尺寸的 anchor,根据大小分配到不同的
    尺度。
  \item 新的 backbone:采用新的 backbone Darknet-53,包含 53 个卷积层,top-5 精
    度与 ResNet-152 相当,速度快 1 倍。
\end{enumerate}


\section{SSD 系列}
\label{sec:SSD}

\subsection{SSD}
\label{subsec:SSD}

\subsection{DSSD}
\label{subsec:DSSD}

\chapter{Anchor Free 方法}


%%% Local Variables:
%%% TeX-master: "../master"
%%% End:


\bibliographystyle{thubib}
\bibliography{refs}
\end{document}

%%% Local Variables:
%%% TeX-master: t
%%% End: