\part{目标检测}
\chapter{基本概念}

\section{指标}

\subsection{交并比(IoU)}
交并比(Intersection over Union,IoU)是衡量两个矩形位置接近程度的一种指标,其定
义为:
\begin{equation}
  \label{equ:IoU}
  \mathrm{IoU} = \frac{\mathrm{Intersection}(A, B)}{\mathrm{Union}(A, B)}
\end{equation}

IoU 的取值在 0 到 1 之间,越接近 1,说明两个矩形位置越接近。

\subsection{mAP}
mAP 全称为 Mean Average Precision。

\section{非极大值抑制(NMS)}

非极大值抑制(Non maximum suppression,NMS)是一种后处理方法,其作用是删除多余的
检测框,保证一个待检测物体只有一个检测框与之对应。一种更直观地理解是极大值保留,
即多个检测框重叠较大时,只保留置信度最大的检测框。图~\ref{fig:nms}~给出了 NMS 的一
个具体示例\citerb{2018-NMS}:

\begin{figure}[ht]
  \centering
  \includegraphics[width=0.8\textwidth]{images/目标检测/NMS.pdf}
  \caption{NMS 示例}\label{fig:nms}
\end{figure}

\subsection{传统 NMS}

传统 NMS 算法的具体流程是,先根据分数对所有检测框排序,高分在前,低分在后。然后将
当前分数最高的检测框加入最终结果,从头往后遍历剩余检测框,如果当前检测框与低分检
测框的 IoU 大于一定阈值,则将低分检测框删掉,如此循环直到处理完所有框(加入最终
结果或删除),具体流程如图~\ref{fig:nms-algo}~中的红框所示。

\begin{figure}[ht]
  \centering
  \includegraphics[width=0.5\textwidth]{images/目标检测/Soft-NMS.pdf}
  \caption{NMS 和 Soft NMS 算法流程}\label{fig:nms-algo}
\end{figure}

\subsection{Soft NMS}\label{subsec:soft-nms}

传统 NMS 算法需要选择合适 IoU 阈值,阈值太低会导致误检,而阈值太高又会导致漏
检。Soft NMS 的基本思想是,两个检测框重叠较大时,降低低分检测框的分数,但并不直接
将其删除。算法的具体流程如图~\ref{fig:nms-algo}~中的绿框所示\citerb{2017-Soft-NMS},
其中 $f(iou(M, b_i))$ 的形式为:
\begin{equation}
f(iou(M, b_i)) = e^{-\frac{\mathrm{iou}{(M, b_i)}^2}{\sigma}}
\end{equation}

即两个框的 IoU 越大,低分框的分数降低得越多。

\subsection{Softer NMS}
文献\citerb{2018-Softer-NMS}~中在 Soft NMS 的基础上提出了 Softer NMS,具体流程
如图~\ref{fig:softer-nms}~所示:

\begin{figure}[ht]
  \centering
  \includegraphics[width=0.5\textwidth]{images/目标检测/Softer-NMS.pdf}
  \caption{Softer NMS 算法流程}\label{fig:softer-nms}
\end{figure}

由图~\ref{fig:softer-nms}~可知,Softer NMS 在 Soft NMS 基础上,加入了 var voting
的部分,其具体步骤是:

\begin{enumerate}
  \item 对任一检测框,计算所有分数低于该框的检测框与该框的 IoU。
  \item 根据 IoU 和每个检测框的不确定度 $\sigma$,修正该检测框的位置,其中 IoU
    越大,$\sigma$ 越小,权重越高。
\end{enumerate}

\chapter{两阶段方法}
\section{R-CNN 系列}\label{sec:R-CNN}

\subsection{Fast R-CNN}\label{subsec:Fast-R-CNN}

Fast R-CNN 在训练和测试时,只需要利用 CNN 将整张图片前传一次,不需要像 R-CNN 中将
所有 proposal 对应的图片区域先缩放到固定尺寸再逐个前传,因此可以大大提升训练和测
试速度\citerb{2015-Fast-RCNN}。

\paragraph{网络结构}
Fast R-CNN 网络的最开始是统一的 CNN backbone,利用 backbone 得到 feature map 后,
将每个 proposal 对应的 feature map 区域,经过 RoI pooling 层得到尺寸为 $h \times
w$ 的 RoI,再经过两个全连接层得到 RoI 对应的 feature,之后再分别经过全连接层 +
Softmax 层的分类器得到分类分数(共 $C+1$ 类,多一个背景类),以及经过全连接层得
到$t_x, t_y, t_w, t_h$四个回归变换值。

\paragraph{RoI Pooling 层}
RoI 的全称是 Region of Interest,中文为感兴趣区域。RoI Pooling 层的作用是将大小不
一的 proposal 统一转化为 $h \times w$ 的尺寸,作为后续全连接层的输入。这是因为全
连接层的输入必须是固定的尺寸,因此 RoI Pooling 对任意尺寸的输入都产生相同尺寸
的输出。

\paragraph{损失函数}
Fast R-CNN 使用分类加回归的多任务 loss:
\begin{align}
  L & = L_{\mathrm{cls}}(p, u) + \lambda [u \geq 1] L_{\mathrm{loc}}(t^u, v) \\
    & = -\log{p_u} + \sum_{i \in {x, y, w, h}} \mathrm{smooth}_{L_1}(t_i^u - v_i)
\end{align}

上式中计算的是正负样本(RoI)的分类 loss 和正样本的回归 loss。其中回归真值 $t^u$
的计算方法为\citerb{2013-RCNN}:
\begin{align}
  t_x & = (G_x - P_x) / P_w \\
  t_y & = (G_y - P_y) / P_h \\
  t_w & = \log (G_w/P_w) \\
  t_h & = \log (G_h/P_h)
\end{align}

上式中,$G$ 和 $P$ 分别表示 GT 和 Proposal(即 RoI),因此 $t$ 相当于两者之间的一
种变换关系,Fast R-CNN 网络回归部分输出的 $v$ 与 $t$ 一样,\textbf{也是变换关系,
  而非直接预测检测框的位置}。回归部分采用 Smooth $L_1$ loss 的原因,是因为
与 $L_2$ loss 相比,该 loss 对偏差较大点的敏感性低,因此不容易出现梯度爆炸,
网络更容易训练。

\paragraph{正负样本}
Proposal 正负样本的判断标准为:

\begin{itemize}
  \item 正样本:与任一 GT 的 IoU $ \geq $ 0.5。
  \item 负样本:与所有 GT 的最大 IoU 在 $ [0.1, 0.5] $ 区间,不包含 0.5。
\end{itemize}

负样本的 IoU 下限取为 0.1 相当于做了难样本挖掘(HEM),因
为 IoU 小于 0.1 的 proposal 可以认为是简单样本,训练时只用较难的样本。训练时每
个 batch 包含 2 张图,每张图取 64 个 proposal,相当于每
个 batch 有 128个 proposal,其中正负样本比例为 1:3。

\subsection{Faster R-CNN}\label{subsec:Faster-R-CNN}

Faster R-CNN 是目标检测领域最经典的论文之一。文中提出了候选区域提取网络(Region
Proposal Network,RPN),可以直接利用 CNN 生成 proposal 以替代 selective search
等方法;同时通过共享 backbone,将 RPN 和 Fast R-CNN 合并成一个统一的 end-to-end
的网络,相比 Fast R-CNN 大幅提高了训练和测试速度。

\paragraph{RPN 网络结构}
RPN 网络最开始是 CNN backbone,在 backbone 之后接一个 $3 \times 3$ 的卷积层得
到 feature map,然后分别用一个 $1 \times 1$ 的卷积层 + Softmax 层的分类器得到分类
分数(\textbf{只是二分类},即前景类和背景类),同时再用一个 $1 \times 1$ 的卷积层得
到 anchor 到 GT 的变换($t_x, t_y, t_w, t_h$四个值)。Faster R-CNN 最终的网络结构
是两个共享 backbone 的 RPN 和 Fast R-CNN 的组合。

\paragraph{RPN 网络损失函数}
RPN 网络的 loss 和 Fast R-CNN 非常类似,都包括分类和回归两项。二者的区别
是:Fast R-CNN 一般为多分类,用 softmax loss;而 RPN 网络为二分类,可以直接用交叉
熵 loss。

\paragraph{RPN 网络正负样本}
RPN 网络 anchor 的正负样本判断标准为:

\begin{itemize}
  \item 正样本:与任一 GT 的 IoU $ \geq $ 0.7,以及与某个 GT 的 IoU 最大的 anchor。
  \item 负样本:与所有 GT 的最大 IoU $ \leq $ 0.3。
  \item 忽略样本:与所有 GT 的最大 IoU 在 $(0.3, 0.7)$ 区间。
\end{itemize}

训练时每个 batch 包含 1 张图的 256 个 anchor,其中正负样本比例为 1:1。如果正样本
数不足 128,用负样本填充,保证一个 batch 的 anchor 数为 256。

\paragraph{RPN 后处理}
由于 RPN 生成的 proposal 有大量的重复区域,因此需要先进行后处理才能作为 Fast
R-CNN 的输入,后处理的步骤为:

\begin{enumerate}
\item 取 RPN 网络生成的分数最高的 top $N_{\mathrm{pre}}$ 个 proposal。
\item 给定一个 IoU 阈值(例如 0.7),对第 1 步的所有 proposal 做 NMS。
\item 取第 2 步 NMS 之后分数最高的 top $N_{\mathrm{post}}$ 个 proposal。
\end{enumerate}

需要强调的是,后处理得到 proposal 后,后续训练就不再需要 RPN 的分数了。Proposal
正负样本的判断与 RPN 分数无关,\textbf{只与其与 GT 的 IoU 有关}。

\paragraph{训练方法}

Faster R-CNN 的训练方法包括 4 步交替训练法和近似联合训练法,其中 4 步交替训练法的
具体步骤如下:

\begin{enumerate}
  \item Backbone 采用 pretrain model 权重初始化,训练 RPN。
  \item Backbone 采用 pretrain model 权重初始化,利用第 1 步 RPN 生成的 proposal,
    训练 Fast R-CNN。
  \item Backbone 采用第 2 步得到的权重初始化,固定 backbone,只 fine tune RPN 新
    增的层。
  \item Backbone 采用第 2 步得到的权重初始化,固定 backbone,利用第 3 步生成
    的 proposal,只 fine tune Fast R-CNN 新增的层。
\end{enumerate}

通过以上四步即可得到 RPN 和 Fast R-CNN 共享 backbone 的 Faster R-CNN 网络。

\section{FPN 及其改进}\label{sec:FPN}

\subsection{FPN}\label{subsec:FPN}
FPN 的核心思想是融合不同尺度的 feature map 得到 feature 金字塔,再进行多尺度预
测\citerb{2016-FPN}。FPN 及不同方法的对比如图~\ref{fig:FPN}~所示:

\begin{figure}[ht]
  \centering
  \includegraphics[width=0.8\textwidth]{images/目标检测/FPN.pdf}
  \caption{FPN 与其他方法对比}\label{fig:FPN}
\end{figure}

\begin{itemize}
  \item 图 (a) 为传统的图像金字塔方法,该方法的问题是速度很慢,因为要进行 $n$ 张
    图片的前传。
  \item 图 (b) 为 Faster R-CNN 采用的方法,从单个 feature map 进行预测,未充分利用
    不同 feature map 的信息。
  \item 图 (c) 为 SSD 采用的方法,从多个 feature map 分别预测,未进行 feature map
    信息的融合。
  \item 图 (d) 为 FPN 采用的方法,先融合不同 feature map 信息,再进行多尺度预测。
\end{itemize}

\paragraph{FPN 不同 feature map 融合 block}
FPN 中不同 feature map 融合的 block 是将深层小尺寸 feature map 进行 2 倍上采样,
同时将其上一层 feature map 经过 $1 \times 1$ 的卷积,使其与上采样后的 feature
map 的通道数相同,再将二者进行 element-wise 相加,最后经过一个 $3 \times 3$
的卷积得到融合后的 feature map。

\paragraph{FPN 检测 head}
FPN 中所有尺寸预测的检测 head 均共享权重,且输出的通道数均为 256,同时 head 中没
有非线性层。

\paragraph{FPN 中的 RPN}

\begin{itemize}
  \item FPN 中的 RPN 从 5 个尺度分别预测,与 SSD 类似,15 个预设尺寸的anchor 分别放
    在 5 个尺度的 feature map,每层 feature map 上有 3 个与其尺寸匹配的 anchor。
  \item Anchor 的正负样本标准与 Faster R-CNN 相同,GT 不再分配到不同尺度,而是直
    接和所有 anchor 计算 IoU,相当于 GT 和 anchor 进行匹配,这样即隐式地将 GT 分
    配到不同尺度。
\end{itemize}

\paragraph{FPN 中的 Fast R-CNN}

FPN 中的 Fast R-CNN 与原始 Fast R-CNN 相比,最大的区别是 FPN 根据 proposal 的尺寸
将其分配到不同 scale,计算方法为 $ k = \lfloor k_0 + \\log_2 (
\sqrt{wh}/224 ) \rfloor$,即根据 proposal 的面积将其分配到不同尺度的 feature map,
面积越大的 RoI 所分配的 feature map 越深,反之越浅。

\subsection{FPN 的改进}
FPN 的改进大致可分为以下几方面:
\begin{enumerate}
  \item Feature map 融合方式
  \item RoI 分配/融合方式
\end{enumerate}

\subsubsection{Feature map 融合方式}
\paragraph{PANet}
PANet 在 FPN top-down 连接的基础上加入了 bottom-up 连接,将较大尺寸的 feature
map 经过 $3 \times 3$ stride 为 2 的卷积,与较小尺寸的 feature
map 做 element-wise 相加,再经过 $3 \times 3$ 的卷积得到对应的 feature
map\cite{2018-PANet}。

\paragraph{AugFPN}
AugFPN 对 FPN 进行了以下改进\citerb{2019-AugFPN}:
\begin{itemize}
  \item AugFPN 先将原始的 feature map $C_i$ 经过 $1 \times 1$ 卷积得到 $M_i$,
    再经过 $3 \times 3$ 卷积与上采样后的 $P_{i-1}$ 相加得到 $P_i$。
  \item AugFPN 中的 $C_5$ 会先缩放到不同尺度,再经过 $1 \times 1$ 卷积后上采样到
    相同尺寸,之后做自适应空间融合得到 $M_6$,即求不同尺度 feature map 不同位置的
    加权和,$M_6$ 再与 $M_5$ 相加后得到 $P_5$。
\end{itemize}

\paragraph{MFPN}
MFPN 同时采用了三种 FPN 结构\citerb{2019-MFPN}:

\begin{itemize}
  \item Top-down FPN:类似原始 FPN,区别是在最深层之后加入了全局平均池化层
    (Global Average Pooling,GAP),将结果与最深层相加,其余与原始 FPN 类似。
  \item Bottom-up FPN:与原始 FPN 相反,先从最浅层开始搭金字塔,该层和其上下两层
    都参与生成金字塔中对应的 feature map。
  \item Fusing-splitting FPN:先将前两层、后两层分别融合,再将融合后的两层融合,
    然后再将浅层上采样,深层下采样,得到最终的四个 feature map。
\end{itemize}

MFPN 将以上三种 FPN 得到的 feature map 相加,作为最终的 feature map。

\subsubsection{RoI 分配/融合方式}
\paragraph{PANet}
PANet 将 RoI 投射到多尺度 feature map 进行 RoI Align 得到同一 RoI 的不同尺
度 feature,再将不同尺度 feature 分别过权重共享的 fc 层后进行拼接,得到检
测 head的 feature\cite{2018-PANet}。

\paragraph{AugFPN}
AugFPN 将 RoI 同时投射到多个 feature map 做 RoI Align,将得到的 feature concat
后经过两个 $1 \times 1$ 卷积 + Sigmoid 得到 $N \times C$ 的权重,将不同 RoI
feature 做加权和得到 $C \times h \times w$ 的融合后 RoI feature\cite{2019-AugFPN}。

\section{Faster R-CNN 的改进}\label{sec:faster-improve}

\subsection{RoI Pooling 层的改进}
\subsubsection{Mask R-CNN}
Mask R-CNN 的主题是实例分割,但论文中提出了 RoI Align 层,该层可看做是更精细的 RoI
Pooling 层,应用后对检测模型的性能也有提升\citerb{2017-Mask-RCNN}。

\paragraph{RoI Pooling 和 RoI Align 的对比}

\begin{itemize}
  \item RoI Pooling:计算时有两次取整:第一次是计算 RoI 在 feature map 上的坐标
    时,将原坐标除以 stride 得到的小数取整;第二次是将 $H \times W$ 的 RoI 转化为 $h
    \times w$ 的尺寸时,将 $H/h$ 和 $W/w$ 的结果取整。
  \item RoI Align:RoI Pooling 中的两次计算都不取整而直接保留小数,采用双线性插值
    计算对应像素的值,具体如图~\ref{fig:roi-align}~所示。
\end{itemize}

\begin{figure}[ht]
  \centering
  \includegraphics[width=0.5\textwidth]{images/目标检测/RoI-Align.png}
  \caption{RoI Align 示意图}\label{fig:roi-align}
\end{figure}

\subsection{检测 head 的改进}
Faster R-CNN 中的检测 head 指的是 Fast R-CNN 中每个 RoI 之后接的网络,包括两个全
连接层,以及后面分类和回归分别对应的两个 $1 \times 1$ 的卷积层。本小节主要讨论
对 Faster R-CNN 检测 head 改进的相关工作。

\subsubsection{R-FCN}
文献\citerb{2016-R-FCN}中将 Faster R-CNN 中每个 RoI 后面接的检测 head 的全连接层,
替换为全卷积网络 + Position Sensitive RoI Pooling (PS-RoI) 层 + 全局平均池化层,模
型速度有 2.5 到 20 倍的提升。

\paragraph{网络结构}

\begin{itemize}
  \item 在 Backbone 得到的 2048 通道的 feature map 之后接一个 $1 \times 1$ 的卷积降
    为 1024 通道的 feature map,称为 FM-1024。
  \item 分类 head:FM-1024 后接 $(C+1)k^2$ 通道的 $1 \times 1$ 卷积,得
    到$(C+1)k^2$ 通道的 feature map,再接 PS-RoI Pooling 层得到 $(C+1) \times
    k^2$ 的 feature,再接平均池化得到 $(C+1)$ 维向量,最后接 Softmax 得到$(C+1)$
    个分类分数。
  \item 回归 head:与分类 head 类似,FM-1024 后接 $4$ 通道的 $1 \times 1$
    卷积,得到$4$ 通道的 feature map,再接 PS-RoI Pooling 层得到 $4
    \times k \times k$ 的 feature,再接平均池化得到 $4$ 维向量,分别对应 $x, y,
    w, h$ 的变换值。
\end{itemize}

R-FCN 中,PS-RoI Pooling 所用的 feature map 与 RPN 所用的 feature map \textbf{并不是同一
个}。

\paragraph{PS-RoI Pooling 层}
PS-RoI Pooling 和 RoI Pooling 的过程十分类似,区别在于 PS-RoI 在做 pooling 时,是
根据 pooling 区域在 RoI 中的位置,选择对应通道 feature map 的对应位置做 pooling,
而非对所有 feature map 的所有位置都做 pooling。

\paragraph{损失函数和 label}
与 \hyperref[subsec:Fast-R-CNN]{Fast R-CNN} 相同。

\subsubsection{Light-Head R-CNN}
Light-Head R-CNN 首先分析了影响 Faster R-CNN 和 R-FCN 模型速度的原因\citerb{2017-Light-head}:

\begin{itemize}
  \item Faster R-CNN:计算量主要集中在每个 RoI sub-network 的两个全连接层。
  \item R-FCN:计算量主要集中在 PS-RoI Pooling 层需要的 $(C+1+4)k^2$ 个 feature
    map 之前的卷积,因为 PS-RoI Pooling 需要的 feature map 数量很多。
\end{itemize}

Light-Head R-CNN 的检测 head 综合借鉴了 Faster R-CNN 和 R-FCN 的模型,用通道数较
少的 feature + pooling + 一个全连接的轻量 head 提高模型速度。三种模型的结构对比如
图~\ref{fig:light-head}~所示:

\begin{figure}[ht]
  \centering
  \includegraphics[width=0.8\textwidth]{images/目标检测/Light-Head-R-CNN.pdf}
  \caption{Faster R-CNN/R-FCN/Light-Head R-CNN 结构对比}\label{fig:light-head}
\end{figure}

\paragraph{Light-head R-CNN 中 PS-RoI pooling 的 feature map}
通过在 $C_5$ feature map 后加分离卷积得到 PS-RoI pooling 需要的 feature map,即
加入两个对称的组合卷积分支:

\begin{itemize}
  \item $k \times 1 \times C_{\mathrm{in}} \times C_{\mathrm{mid}}$ + $1 \times k
    \times C_{\mathrm{mid}} \times C_{\mathrm{out}}$
  \item $1 \times k \times C_{\mathrm{in}} \times C_{\mathrm{mid}}$ + $k \times 1
    \times C_{\mathrm{mid}} \times C_{\mathrm{out}}$
\end{itemize}

文中卷积核的尺寸非常大,参数 $k=15$,但输出通道数相比 R-FCN 减少很多,只有 $10
\times k \times k$,其中 $k$ 为 PS-RoI Pooling 的尺寸,而 R-FCN 中则需要 $(C+1)
\times k \times k$ 个通道的 feature map。

\paragraph{R-CNN 子网络}
将 Faster R-CNN 中的两个全连接层简化为一个输出为 2048 的全连接层,同时将分类的全
连接层的输出减少为 4 维,即所有类别共享输出结果,不再单独输出每个类别的结果。

\subsubsection{Double-Head R-CNN}
Double-head R-CNN 首先分析了两种获得 RoI feature 的方法\citerb{2019-Double-head}:

\begin{itemize}
  \item RoI pooling 后接两个全连接层,得到维度为 $C_{\mathrm{out}}$ 的 feature。
  \item RoI pooling 后接两个卷积层,得到通道数为 $C_{\mathrm{out}}$ 的 feature
    map,再做全局平均池化,得到维度为 $C_{\mathrm{out}}$ 的 feature。
\end{itemize}

文中经过实验对比分析,发现全连接层得到的 feature 更适合分类,而卷积层得到
的 feature 更适合回归。

\paragraph{损失函数}
Double-head R-CNN 采用 end-to-end 的联合训练,其损失函数为:
\begin{equation}
  \label{equ:double-head-loss}
  \mathcal{L} = \mathcal{L}^{\mathrm{fc}} + \mathcal{L}^{\mathrm{conv}} + \mathcal{L}^{\mathrm{rpn}}
\end{equation}

其中 $\mathcal{L}^{\mathrm{fc}}, \mathcal{L}^{\mathrm{conv}},
\mathcal{L}^{\mathrm{rpn}}$ 分别表示全连接 head、卷积 head 和 RPN 的 loss,两个
head 的 loss 又分别包含分类和回归两部分:
\begin{align}
  \label{equ:double-head-fc-loss}
  \mathcal{L}^{\mathrm{fc}} & = \lambda^{\mathrm{fc}}L_{\mathrm{cls}}^{\mathrm{fc}} + (1-\lambda^{\mathrm{fc}})L_{\mathrm{reg}}^{\mathrm{fc}} \\
  \label{equ:double-head-conv-loss}
  \mathcal{L}^{\mathrm{conv}} & = \lambda^{\mathrm{conv}}L_{\mathrm{cls}}^{\mathrm{conv}} + (1-\lambda^{\mathrm{conv}})L_{\mathrm{reg}}^{\mathrm{conv}}
\end{align}

\paragraph{权重最优值}
文中经过寻优,得到的权重最优值为 $\lambda^{\mathrm{fc}} = 0.7,
\lambda^{\mathrm{\mathrm{conv}}} = 0.0$,注意卷积 head 的 feature \textbf{完全不学分类,只
学回归时最优}。

\paragraph{分类分数}
文中对比了四种不同的利用全连接和卷积 feature 分类分数计算最终分类分数的方法,分别
是$s^{\mathrm{fc}}, (s^{\mathrm{fc}} + s^{\mathrm{conv}})/2,
\\max(s^{\mathrm{fc}}, s^{\mathrm{conv}}), s^{\mathrm{fc}} +
s^{\mathrm{conv}}(1-s^{\mathrm{fc}})$,四种方法对应的模型性能 差别不大,最后一种
方法性能最好。

\subsubsection{Grid R-CNN}
Grid R-CNN 中借鉴了 CornerNet 的思路,改进了 Faster R-CNN 中的回归 head,利
用 FCN 输出的 $N \times N$ 个heatmap 确定检测框的位置\citerb{2018-Grid-R-CNN}。

\paragraph{Grid R-CNN 网络结构}
如图~\ref{fig:Grid-RCNN}~所示:

\begin{figure}[ht]
  \centering
  \includegraphics[width=0.9\textwidth]{images/目标检测/Grid-R-CNN.pdf}
  \caption{Grid R-CNN 网络结构}\label{fig:Grid-RCNN}
\end{figure}

Grid R-CNN 对每个 RoI 输出 $N \times N$ 个点的 feature map,再利用这些点计算检测
框的位置。回归 head 的具体结构是:

\begin{itemize}
  \item $14 \times 14$ 的 RoI feature 后接 8 个 $3 \times 3$ 分离卷积,再接两个反卷
    积,得到 $N \times N \times 56 \times 56$ 的 feature map,分别对应 RoI 检测框
    的 $N \times N$ 个点。
  \item 一阶特征融合模块:将上一步得到的 feature map 再接 3 个 $5 \times 5$ 的卷
    积,考虑与其 $L_1$ 距离为 1 的相邻点对应的 feature map,做 element-wise 和。
  \item 二阶特征融合模块:与一阶类似,在一阶融合得到的 feature map 上继续将
    $L_1$ 距离为 2 的相邻点对应的 feature map 做 element-wise 和。
\end{itemize}

\paragraph{扩张区域映射}
RoI 可能不包括 GT 的角点,在做映射时使用如下公式:
\begin{align}
  \begin{split}
    I_x' = & P_x + \frac{4H_{x} - w_o}{2w_o} w_p  \\
    I_y' = & P_y + \frac{4H_{y} - h_o}{2h_o} h_p
  \end{split}
\end{align}

相当于将映射对应的区域扩大到了原 RoI 的 4 倍。

\paragraph{损失函数和正负样本}
heatmap 的损失函数为交叉熵 loss,GT 角点及其上下左右四个点组成的十字型为正样本,其
余点为负样本。

\paragraph{检测框位置的确定}
采用每条边上三个点的加权平均确定位置,权重为回归 head 输出的分数。

\subsection{RPN 的改进}
\subsubsection{GA-RPN}\label{subsubsec:GA-RPN}
文献\citerb{2019-GA-RPN}中对 RPN 网络进行了改进,提出 GA-RPN,anchor 数量仅为 baseline
的 10\%,同时 recall 比 baseline 高 9.1 个点。

\paragraph{网络结构}

\begin{itemize}
  \item 分类分支:用于预测 anchor 可能位置。在原始 feature map 上加 $1 \times 1$
    的卷积再接 sigmoid,输出尺寸为 $W \times H \times 1$,得到 [0, 1] 之间的值。
  \item 回归分支:用于预测不同位置 anchor 的形状。在原始 feature map 上加 $1
    \times 1$的卷积,输出尺寸为 $W \times H \times 2$,不直接输出尺寸,需要经过
    以下变换 $w = \sigma \cdot s \cdot e^{dw}$。
  \item Feature map 变换分支:原始 feature map 后接 deformable 卷积,offset 由
    回归分支输出后再接 $1 \times 1$ 的卷积得到。
\end{itemize}

\paragraph{损失函数}
分类分支为 focal loss,回归分支为如下 Smooth $L_1$ loss:
\begin{equation}
  \mathcal{L}_{\mathrm{shape}} = \mathcal{L}_1 \left( 1 - \\min\left( \frac{w}{w_g}, \frac{w_g}{w} \right) \right) + \mathcal{L}_1 \left( 1 - \\min \left( \frac{h}{h_g}, \frac{h_g}{h} \right) \right)
\end{equation}

\paragraph{正负样本}
设 GT 投射到 feature map 对应的框为 $b_p^l =[x_p^l, y_p^l, w_p^l, h_p^l]$,文中设置了
两个参数$\sigma_1 = 0.2, \sigma_2 = 0.5$。

\begin{itemize}
  \item 有效区域:$b_e^l=[x_p^l, y_p^l, \epsilon_e x_p^l, \epsilon_e y_p^l]$,其
    中 label 为 1。
  \item 忽略区域:$b_i^l=[x_p^l, y_p^l, \epsilon_i x_p^l, \epsilon_i y_p^l] - b_e^l$。
  \item 其余区域:label 为 0。
\end{itemize}

\paragraph{回归 label}
由于 GA-RPN 中 anchor 的 $w$ 和 $h$ 均为网络输出量,因此无法根据 IoU 最大的判据找到
匹配的 GT。文中的做法是利用预设的 $w$ 和 $h$ 寻找匹配的 GT,再利用匹配的 GT 的 $w$ 和
$h$ 作为回归分支的 label。

\subsection{级联方法的改进}

\subsubsection{Cascade R-CNN}\label{subsub:cascade-rcnn}

Cascade 方法是指利用多阶段检测器进行检测的方法。其实所有的两阶段方法都属于广义的
Cascade 方法,因为两阶段方法一般都分为生成 proposal 的第一阶段和精修 proposal 的
第二阶段。

Cascade R-CNN 拓展了 Faster R-CNN 的两阶段,在之后又接了两个检测 head,同时训练时
后级 head 的正样本判据更为严格,提升了模型性能\citerb{2017-Cascade-RCNN}。

\paragraph{简单提高正负样本 IoU 阈值的问题}
Fast 和 Faster R-CNN 中,RoI 的正负样本判据均为与 GT 的 IoU 大于 0.5,但 0.5 实际
上是一个相当松的阈值,然而简单地提高正负样本的 IoU 阈值会导致以下两个问题:

\begin{itemize}
  \item 提高阈值后正样本数量急剧减少,会导致严重的过拟合。
  \item 检测器对阈值附近的样本的修正效果最好,高阈值训练的模型在低质量的输入条件
    下,同样无法提高性能。
\end{itemize}

\paragraph{Cascade R-CNN 网络结构}
Cascade R-CNN 与 Faster R-CNN 的对比如图~\ref{fig:Cascade-RCNN}~所示:

\begin{figure}[ht]
  \centering
  \includegraphics[width=0.8\textwidth]{images/目标检测/Cascade-R-CNN.pdf}
  \caption{Cascade R-CNN 与 Faster R-CNN 的对比}\label{fig:Cascade-RCNN}
\end{figure}

Cascade R-CNN(上图 (d))与 Faster R-CNN(上图 (a))相比,多了两个检测 head,即 H2 和 H3。训练时,H1、
H2、H3 对应的正负样本 IoU 阈值分别取为 0.5、0.6 和 0.7,由于每个 head 输出的结果
相比输入都会有改善,相当于逐级微调得到最终的高质量检测框。

\chapter{单阶段方法}

\section{YOLO 系列}\label{sec:YOLO}

\subsection{YOLO v1}\label{subsec:YOLOv1}
YOLO v1 是目标检测单阶段方法的开创性工作之一,文中将检测问题转化为一个回归问题,而非
两阶段文章中的分类和回归两方面问题\citerb{2015-YOLO-v1}。

\paragraph{检测方法}

\begin{enumerate}
  \item 将整张图片划分为 $S \times S$ 个网格,每个网格预测 $ B $ 个检测框和 $ C $
    个条件概率 $ \Pr(\mathrm{Class}_i|\mathrm{Object}) $。
  \item 每个检测框共包括 5 个值: $x, y, w, h$ 和 confidence。其中 $x, y$为中心位
    置,$w, h$ 为长和宽(相对图片长宽的归一值
    ),confidence 为$\Pr(\mathrm{Object}) \times
    \mathrm{IoU}^{\mathrm{truth}}_{\mathrm{pred}}$,
    检测框的 confidence 乘该网格对应的条件概率为最终的检测得分。网络的输出维度为 $ S
    \times S \times (B \times 5 + C) $。
\end{enumerate}

\paragraph{Backbone}

YOLO v1 的 backbone 采用作者自己设计的结构为 24 个卷积层 + 2 个全连接层的自定义网
络。训练时,先将前 20 个卷积层 + 平均池化层组成的网络在 ImageNet 数据集上进行预训
练,输入尺寸为 $224 \times 224$。检测时将图片的输入尺寸扩大为 $448 \times 448$。
为避免过拟合,第一个全连接层后加了一个 $p=0.5$的 dropout 层。

\paragraph{损失函数}

\begin{align}
  \label{equ:yolo-v1-loss}
  \begin{split}
    L = & \, \lambda_{\mathrm{coord}} \sum_{i=0}^{S^2} \sum_{j=0}^{B} \mathds{1}_{ij}^{\mathrm{obj}} \left [ {\left (x_i - \hat{x}_i \right )}^2 + {\left (y_i - \hat{y}_i \right )}^2 \right ] \\
    & \, + \lambda_{\mathrm{coord}} \sum_{i=0}^{S^2} \sum_{j=0}^{B} \mathds{1}_{ij}^{\mathrm{obj}} \left [ {\left(\sqrt{w_i} - \sqrt{\hat{w}_i} \right)}^2 + {\left (\sqrt{h_i} - \sqrt{\hat{h}_i} \right )}^2 \right ]  \\
    & \, + \sum_{i=0}^{S^2} \sum_{j=0}^{B} \mathds{1}_{ij}^{\mathrm{obj}} {\left( C_i - \hat{C}_i \right)}^2  \\
    & \, + \lambda_{\mathrm{noobj}} \sum_{i=0}^{S^2} \sum_{j=0}^{B} \mathds{1}_{ij}^{\mathrm{noobj}} {\left( C_i - \hat{C}_i \right)}^2  \\
    & \, + \sum_{i=0}^{S^2} \mathds{1}_{i}^{\mathrm{obj}} \sum_{c \in \mathrm{classes}} {\left( p_i(c) - \hat{p}_i(c) \right)}^2
  \end{split}
\end{align}

\begin{enumerate}
  \item 损失函数可分为三部分:第一部分包括前两项,是检测框位置的 loss;第二部分包
    括中间两项,是检测框分数的 loss;第三部分包括最后一项,是网格类别概率的 loss。所
    有 loss 均为 $L_2$ 形式。
  \item 检测框位置的 loss 权重 $\lambda_{\mathrm{coord}}$ 提高到 5,同时将不含 GT 的
    网格中检测框分数的 loss 权重 $\lambda_{\mathrm{noobj}}$ 降为 0.5。
  \item 只有与 GT IoU 最大的检测框才会产生检测框位置 loss,只有网格中包含 GT 中
    心才产生网格类别概率 loss。
\end{enumerate}

\subsection{YOLO v2}\label{subsec:YOLOv2}
YOLO v2 在 YOLO v1 的基础上做了一系列改进\citerb{2016-YOLO-v2}:

\begin{enumerate}
  \item 引入 BN:所有卷积层之后都加入 BN 层,由于 BN 层有正则化作用,因此可以去掉
    YOLOv1 中全连接层之后的 dropout 层。
  \item 改变预训练尺寸:YOLO v1 中预训练时图片输入尺寸为 $224 \times 224$,而检
    测时图片的输入尺寸为 $448 \times 448$,v2 中将预训练时的输入尺寸也改为 $448
    \times 448$,但网络实际的输入尺寸为 $416 \times 416$。
  \item 引入 anchor:参考 Faster R-CNN~\cite{2015-Faster-RCNN},YOLO v2 中同样引
    入 anchor 的概念,即网络的输出为 anchor 到 GT 的变换,同时每个 anchor 分别预
    测类别概率和前景分数,而不是一个网格只预测一个类别概率,这样就可以避免 v1 中
    每个网格只能预测一个类别的问题。
  \item Anchor 尺寸聚类:利用 k-means 算法将 GT 聚类,得到 k 个 anchor 尺寸的先
    验。聚类时距离定义为 $d = 1 - \mathrm{IoU}(\mathrm{box}, \mathrm{centroid})$。
  \item 预测相对位置:YOLO v2 输出的检测框的坐标是相对网格的偏差,而非与
    Faster R-CNN 中相同的变换,这样可以将 anchor 的中心限制在该网格内。在网络输
    出后加入 sigmoid 函数即可实现将任意输入压缩至 0 到 1 的区间。
  \item 特征融合:将尺寸为 $26 \times 26$ 的 feature map 与尺寸为 $13 \times 13$
    的 feature map 进行融合,具体方法是将 $26 \times 26 \times 512$ 的 feature
    map 变换为 $13 \times 13 \times 2048$,再和 $13 \times 13$ 的 feature map 拼
    接。
  \item 多尺度训练:训练时采用多尺度训练,即图片输入尺寸为 320 到 608 之间,步长
    为 32 的随机数。
  \item 新的 backbone:采用新的 backbone Darknet-19,包含 19 个卷积层和 5 个最大
    池化层。
\end{enumerate}

\subsection{YOLO v3}\label{subsec:YOLOv3}
YOLO v3 在 YOLO v2 的基础上做了部分改进\citerb{2018-YOLO-v3}:

\begin{enumerate}
  \item Anchor loss 的计算:每一个 GT 分配与其 IoU 最大的 anchor,该 anchor 对应的
    前景分数为 1,如果不是 IoU 最大的 anchor 且与某个 GT 的 IoU 大于 0.5,则
    该 anchor 会被忽略,即不参与计算前景分数 loss。同时,没有 GT 匹配的 anchor 也不产生
    检测框位置和类别 loss,只计算前景分数 loss。
  \item Loss 类型:将 Softmax loss 改为多个 sigmoid loss,一个 anchor 可以同时对
    应多个类别。
  \item 多尺度特征融合:采用类似 FPN\cite{2016-FPN} 的结构,将深层小尺寸 feature
    map 进行上采样后,与浅层大尺寸 feature map 相加,作为新的 feature map,文
    中共有 3 个尺度。
  \item Anchor 按尺度分配:YOLO v3 中共有 9 种尺寸的 anchor,根据大小分配到不同的
    尺度。
  \item 新的 backbone:采用新的 backbone Darknet-53,包含 53 个卷积层,其 top-5
    精度与 ResNet-152 相当,速度快 1 倍。
\end{enumerate}

\section{SSD 系列}\label{sec:SSD}

\subsection{SSD}\label{subsec:SSD}

SSD 是目标检测单阶段方法的经典工作之一,直观上可以将 SSD 看成多分类 + 多 feature
map 预测的 RPN\cite{2015-SSD}。

\paragraph{网络结构}
SSD 网络的 backbone 采用 VGG-16,其后接了若干卷积层。检测 head 是一个 $3 \times
3$ 的卷积层,分别接在不同尺寸的 feature map 后面进行多尺度预测。

\paragraph{损失函数}
与 Fast R-CNN 相同。

\paragraph{正负样本}
SSD 中引入了和 anchor 类似的 default box,其正负样本的标准为:

\begin{itemize}
  \item 正样本:包括两类,第一类是某个 GT 对应的 IoU 最大的 default box,第二类是
    与所有 GT 的最大 IoU $ \geq 0.5 $。
  \item 负样本:与 GT 的最大 IoU $ < 0.5 $。
\end{itemize}

\paragraph{Default box 尺寸}
SSD 中不同尺寸 feature map 上的 default box 采用了不同的预设尺
寸,这部分原文和源代码不一致,基本原则是尺寸大的浅层 feature map 上的 default
box 尺寸小,反之尺寸大,具体请参
考\href{https://github.com/weiliu89/caffe/blob/ssd/examples/ssd/ssd_pascal.py}{源码}。

\paragraph{训练}
SSD 训练时的正负样本比为 1:3,在训练时使用了难样本挖掘(HEM),即只选择 loss 最高
的负样本进行计算。

\subsection{SSD 的改进}
\subsubsection{FPN 思想}
\paragraph{DSSD}
DSSD 是 SSD 的扩展版,思想与 \hyperref[sec:FPN]{FPN} 类似,同样对不同尺度的
feature map 进行融合以提高模型性能\citerb{2017-DSSD}。DSSD 的特征融合模块包括:

\begin{itemize}
  \item 深层 feature map:先经过 $2 \times 2$ 的反卷积,再接 $3 \times 3$ 的卷积
    和 BN,输出通道数为 512。
  \item 浅层 feature map:先经过一个标准的 $3 \times 3$ 卷积-BN-ReLU,再接 $3
    \times 3$ 的卷积和 BN,输出通道数同样为 512。
  \item 融合:将深层和浅层 feature map 做 element-wise 乘积,再接 ReLU,得到融合
    后的 feature map。
\end{itemize}

DSSD 对 SSD 的预测模块也进行了改进。SSD 中,预测模块只有 $3 \times 3$ 卷积,直接
接在对应的 feature map 上,DSSD 中对这一部分也进行了改进:

\begin{itemize}
  \item 原始 feature map 先经过 3 个 $1 \times 1$ 的卷积,输出通道数分别为 256,
    256 和 1024,得到 feature map 1。
  \item 原始 feature map 直接经过 1 个输出通道数为 1024 的 $1 \times 1$ 的卷积,
    得到 feature map 2。
  \item 将 feature map 1 和 feature map 2 做 element-wise 求和,得到最终的
    feature map,之后再接 $3 \times 3$ 的卷积分别进行分类和回归。
\end{itemize}

\paragraph{RON}
RON 借鉴 FPN 思想融合不同尺度的 feature map,此外还预测 feature map 上每个点对应
的 objectness score,仅在 objectness 分数高于一定阈值的区域进行检测,从而减轻单阶
段检测器的正负样本不平衡问题\citerb{2017-RON}。

\paragraph{NETNet}
NETNet 借鉴了 \hyperref[subsec:FPN]{FPN} 的思路对 SSD 进行改进,
同时引入 Neighbor Erasing Module(NEM) 和 Neighbor Transferring Module(NTM)融
合不同 scale feature map 信息\citerb{2020-NETNet}。

NETNet 中每层融合的 feature map 与本层及前后两层相关,下层上采样 + 本层 + 上层下
采样后接$1 \times 1$ 卷积后逐元素相加。

NEM 的具体结构为:
\begin{itemize}
  \item 上层 $p_{s+1}$ 经过上采样、$1 \times 1$卷积、sigmoid 得到 $g_{s+1}^s$。
  \item $g_{s+1}^s$ 与 $p_s$ 相乘得到 $p_{es}$。
  \item $p_s$ 与 $p_{es}$ 相减得到 $\tilde{p}_s$。
\end{itemize}

NTM 将 NEM 得到的 $p_{es}$ 经过下采样和 $1 \times 1$ 卷积,再与 $p_{s+1}$ 相加得
到 $\tilde{p}_{s+1}$。

\subsubsection{双向 FPN}
\paragraph{BPN}
BPN 在 FPN top-down 的基础上又加入了 bottom-up 的路径,整体结构类似 PANet,同时也
应用了 Cascade R-CNN 的思想,路径越深对应的 IoU 阈值越高,其整体结构如
图~\ref{fig:BPN}~所示\citerb{2018-BPN}:

\begin{figure}[ht]
  \centering
  \includegraphics[width=0.9\textwidth]{images/目标检测/BPN.pdf}
  \caption{BPN 的结构}\label{fig:BPN}
\end{figure}

\paragraph{LSN}
LSN 在 SSD 基础上加入双向 FPN,其结构如图~\ref{fig:LSN}~所示\citerb{2019-LSN}:

\begin{figure}[ht]
  \centering
  \includegraphics[width=0.8\textwidth]{images/目标检测/LSN.pdf}
  \caption{LSN 的结构}\label{fig:LSN}
\end{figure}

\begin{itemize}
  \item 新加入了 scratch 网络,先将原始图片下采样,再经过 6 个卷积,取四个尺寸的
    中间结果分别经过 $1 \times 1$ 的卷积,用于特征融合。
  \item 双向 FPN,包括由浅及深和由深及浅两部分,分别对应 bottom-up 和 top-down
    子网络。
\end{itemize}

\subsubsection{组合 FPN}
\paragraph{PFPNet}
PFPNet 从同一个 feature map 上 pooling 得到不同尺寸的 feature map,再分别组合得
到用于预测的 feature map,其结构如图~\ref{fig:PFPNet}~所示\citerb{2018-PFPNet}:

\begin{figure}[ht]
  \centering
  \includegraphics[width=0.8\textwidth]{images/目标检测/PFPNet.pdf}
  \caption{PFPNet 的结构}\label{fig:PFPNet}
\end{figure}

\begin{itemize}
\item Backbone 得到的 feature map 经过不同 stride 的 pooling 得到不同尺寸的
  feature map。
\item 不同尺寸 feature map 经过上/下采样后 concat,再经过卷积得到用于预测的
  feature map。
\end{itemize}

\subsubsection{网络新模块}
\paragraph{RefineDet}
RefineDet 借鉴了 Faster R-CNN、SSD 和 FPN 等论文的思路,综合吸收了单阶段和两阶段
目标检测的优点\citerb{2017-RefineDet}。

RefineDet 的网络结构包括 ARM、ODM 和 TCB 三个模块。
\begin{itemize}
  \item ARM:类似 Faster R-CNN 中的 RPN,将 anchor 做二分类和粗调。ARM 分数低于
    0.01 的负样本 anchor 训练时和测试时均直接舍弃,不再传给 ODM。
  \item ODM:类似 SSD,将粗调的 anchor 做全分类和精修。
  \item TCB:类似 FPN,但结构略复杂,底层 feature map 的操作由上采样改为反卷积,
    上层 feature map 的操作由 $1 \times 1$ 卷积改为 $3 \times 3$ 卷积 + ReLU +
    $3 \times 3$ 卷积。
\end{itemize}

RefineDet 的损失函数包括 ARM 和 ODM 两个模块的分别对应 RPN 和 Fast R-CNN 的 loss,
总 loss 为二者之和。

\paragraph{RFBNet}
RFBNet 借鉴 Inception V4 中的 inception 模块思想,同时加入了空洞卷积\citerb{2017-RFBNet}。

\paragraph{DES}
DES 借鉴了 \hyperref[subsec:SENet]{SENet} 的思想调整不同 feature map channel 的
权重,同时加入了弱监督的 segmentation 分支提升模型性能\citerb{2017-DES}。

图~\ref{fig:des-seg}~为 DES 中的 Segmentation 分支结构:

\begin{figure}[ht]
  \centering
  \includegraphics[width=0.8\textwidth]{images/目标检测/DES.pdf}
  \caption{DES 中的 segmentation 分支}\label{fig:des-seg}
\end{figure}

Segmentation 分支中间结果 $G(X)$ 经过卷积和 sigmoid 与原始 feature map 相乘得到改
进后用于 detection 的 feature map,同时中间结果经过卷积和 softmax 得到的 mask 还
会根据弱监督的分割信息计算分割 loss。弱监督的分割信息是指将 GT 检测框内全部视为对
应类别,如果有区域属于多个 GT 则类别为面积最小的 GT 对应的类别。

\paragraph{STDN}
STDN 以 DenseNet 为 Backbone,引入 scale transfer module,将$C \times r^2 \times
H \times W$ 的 feature map 变为 $C \times rH \times rW$ 的形式,即将 $r^2$ 个
channel 的值挪到一个 channel 上,拼成一个 $r^2$ 的正方形,将原 $H \times
W$ 的 feature map 的尺寸变为 $rH \times rW$。

\paragraph{LFIP}
LFIP 借鉴了传统的图像金字塔方法,将下采样的图像经过轻量级网络与原始 feature map
作 attention,同时将不同 level 的信息进行融合。LFIP 的结构如图~\ref{fig:LFIP}~所
示\citerb{2019-LFIP}:

\begin{figure}[ht]
  \centering
  \includegraphics[width=0.9\textwidth]{images/目标检测/LFIP.pdf}
  \caption{LFIP 网络结构}\label{fig:LFIP}
\end{figure}

\begin{itemize}
  \item 原图像下采样后的各个图像都经过 4 个卷积层的轻量网络与原始 feature map 作 attention。
  \item 浅层 feature map 与深层 feature map 相加做融合。
\end{itemize}

\paragraph{EFGRNet}
EFGRNet 的结构如图~\ref{fig:EFGRNet}~所示\citerb{2019-EFGRNet}:

\begin{figure}[ht]
  \centering
  \includegraphics[width=0.9\textwidth]{images/目标检测/EFGRNet.pdf}
  \caption{EFGRNet 网络结构}\label{fig:EFGRNet}
\end{figure}

EFGRNet 综合借鉴了多篇论文的思路:
\begin{itemize}
  \item 空洞卷积:RFBNet
  \item 级联结构和 deformable 卷积:HSD
  \item 下采样原始图片过简单卷积网络:LFIP
\end{itemize}

\section{RetinaNet 系列}
\subsection{RetinaNet}\label{sub:RetinaNet}

RetinaNet 是单阶段目标检测里程碑式的论文,综合吸收了 Faster R-CNN、SSD 和 FPN 中
的精华,同时提出新颖的 focal loss,精度超过了当时的两阶段模型,一举扭转了当时比较
流行的目标检测模型双阶段准、单阶段快的观点\citerb{2017-RetinaNet}。

\paragraph{Focal loss 的应用}
RetinaNet 网络的分类 loss 采用 \hyperref[subsec:focal]{focal loss},其中 $\alpha
= 0.25, \gamma = 2$ 时效果最好。由于正样本数量很少,因此分类器的初始化也做了相应
调整,将最后一个卷积层的偏置初值设为 $-\log((1 - \pi)/\pi)$,其中 $\pi$ 为正样
本概率,文中取为 0.01。使用 focal loss 后,所有 anchor 都参与计算 loss,不再需要
采样 anchor 以及控制正负样本比例。

\paragraph{Anchor 参数}
RetinaNet 同样采用了 FPN 结构,分别从 $P_3$ 到 $P_7$ 五个尺度进行预测,对应的
anchor 面积为 $32^2$ 到 $512^2$。每个尺度上有三个长宽比例和三种不同尺寸 anchor
的组合,共 9 种 anchor。

\paragraph{正负样本}
正负样本判断标准如下:
\begin{itemize}
  \item 正样本:与任一 GT 的 IoU $\geq$ 0.5。
  \item 负样本:与所有 GT 的最大 IoU $ < $ 0.4。
  \item 忽略样本:与所有 GT 的最大 IoU 在 $[0.4, 0.5]$,不包含 0.5。
\end{itemize}

\paragraph{分类 head}
在 FPN 得到的通道数为 $C$ 的 feature map 上,先接 4 个通道数为 $C$ 的
$3 \times 3$ 的包含 ReLU 的卷积,再接 1 个 channel 数为 $KA$ 的 $3 \times 3$ 的
卷积,其中 $K$ 为类别数,$A$ 为 anchor 数,最后再接 sigmoid 层得到最终的分数。所
有尺度分类 head 均共享参数。

\paragraph{回归 head}
回归 head 与分类 head 的结构基本相同,但最后不接 sigmoid,因为回归部分需要输出负
值。回归 head 所有类别共享相同的输出,而非每个类别单独输出。所有尺度回归 head 同
样共享参数。回归 head 和分类 head 虽然结构类似,但并不共享参数。

\paragraph{Inference}
Inference 时 FPN 每个尺度的预测结果取 0.05 作为分数阈值,然后取分数最高的前 1000
个结果,再进行阈值为 0.5 的 NMS,得到最终结果。

\subsection{RetinaNet 的改进}

\subsubsection{正负样本}
\paragraph{ATSS}
ATSS 发现将 FCOS\cite{2019-FCOS} 中的正负样本判据应用于 RetinaNet 后,二者性
能十分接近,借鉴 FCOS 的思路提出 ATSS 方法作为正负样本判据\citerb{2019-ATSS}。

\paragraph{RetinaNet 和 FCOS 的区别}
\begin{enumerate}
  \item anchor 形式:RetinaNet 为 bbox,FCOS 为 feature map 上的点。
  \item 每个位置的 anchor 数:RetinaNet 为 $k$ 个,FCOS 为 1 个。
  \item 正负样本定义:RetinaNet 使用 GT 和 anchor 的 IoU 值判断,FCOS 原始论文中,
    GT 内的点都算正样本,通过 FPN 和最小面积 GT 匹配减少模糊正样本数。
  \item 回归初始值:RetinaNet 为 anchor,FCOS 为 feature map 上点的坐标。
\end{enumerate}

\paragraph{ATSS 方法}
\begin{enumerate}
  \item FPN 每层 feature map 选出离 GT 中心 $L_2$ 距离最近的 $k$ 个 anchor 作为
    候选集合,其中 $k$ 为超参数。
  \item 计算候选集合中所有 anchor 与 GT 的 IoU,得到候选集合所有 anchor 与 GT IoU 的均值 $m_g$ 和标准差 $v_g$。
  \item IoU 阈值大于 $t_g = m_g + v_g$ 且 anchor 中心位于 GT 内的作为正样本,否
    则为负样本。如果一个 anchor 对应多个 GT,取 IoU 最大的 GT。
\end{enumerate}

\section{EfficientDet}
EfficientDet 利用 EfficientNet 作为 backbone,同时引入 BiFPN 结构,是当时 SOTA
的检测方法。

\subsection{BiFPN}
BiFPN 的结构如图~\ref{fig:BiFPN}~在 PANet 的基础上做了以下改进:
\begin{enumerate}
  \item 去掉只有一个输出的节点,同时加入同级输入到输出的连接。
  \item 加入重复的 BiFPN 模块进行特征融合。
  \item 不同 feature map 采用加权融合,而非简单地直接相加。
  \item 融合时的卷积使用深度可分离卷积代替传统卷积。
\end{enumerate}

\begin{figure}[ht]
  \centering
  \includegraphics[width=\textwidth]{images/目标检测/BiFPN.pdf}
  \caption{BiFPN 与其他 FPN 结构的对比}\label{fig:BiFPN}
\end{figure}

\subsection{EfficientDet}
EfficientDet 的结构如图~\ref{fig:EfficientDet}~所示:

\begin{figure}[ht]
  \centering
  \includegraphics[width=\textwidth]{images/目标检测/EfficientDet.pdf}
  \caption{EfficientDet 结构}\label{fig:EfficientDet}
\end{figure}

EfficientDet D0-D7 采用 EfficientNet D0-D6 作为 backbone,D7 仍然采
用 D6 作为 backbone,但输入尺寸更大,不同 level 对应的 BiFPN 宽度、深度、head 深
度和输入尺寸用如下公式计算:

\begin{enumerate}
\item BiFPN 宽度:$W = 64 \cdot 1.35^{\phi}$
\item BiFPN 深度:$D = 3 + \phi$
\item Head 深度:$D = 3 + \lfloor \phi \rfloor$
\item 输入尺寸:$R = 512 + 128 \phi$
\end{enumerate}

\section{LapNet}
LapNet 的思路与传统的检测网络有明显区别,其特点是体现了良好的速度和精度的均衡,
如图~\ref{fig:LapNet}~所示:

\begin{figure}[ht]
  \centering
  \includegraphics[width=0.6\textwidth]{images/目标检测/LapNet.pdf}
  \caption{LapNet 与其他检测方法对比}\label{fig:LapNet}
\end{figure}

\paragraph{网络结构}
主体网络仍然借鉴 FPN 的思路,即采用 bottom-up + 横向连接 + top-down 的结构得到信
息融合的不同尺寸的 feature map。但 LapNet 并不在这些不同尺寸的feature map 上直接
加检测 head,而是用卷积 + 上采样得到与最大尺寸相同的多个 feature map,将它们拼在
一起后,再接检测 head。其输出与 RetinaNet 类似,但回归部分是的维度为$4 K \times A$,
而非 RetinaNet 的 $4A$,因为 LapNet 每个类别都有各自的 anchor,所以每个类别都需要
单独输出。

\paragraph{PONO map}
PONO(Per-Object Normalized Overlap)是 LapNet 中的核心概念,其具体的计算方法为:

\begin{itemize}
  \item 计算 anchor 对应的 GT 及 overlap 值:遍历 feature map 上的所有点,计算
    每个点对应的 anchor 与所有 GT 的 IoU,得到 IoU 最大值及对应的 GT。
  \item 计算 PONO map:对任一 GT,计算所有对应该 GT 点的最大值,以此为基准对该 GT
    对应的所有点的值做标准化,标准化后最大值点变为 1,其他点的值为该点原值/基值。
\end{itemize}

\paragraph{损失函数}
LapNet 的损失函数包括回归、分类和正则化三部分 loss:

\begin{itemize}
  \item 回归:采用 IoU $L_2$ Loss,即 $\mathcal{L}_{\mathrm{loc}} =
    \left \| 1 - \mathrm{IoU}(B_{\mathrm{GT}, B_{\mathrm{DT}}}) \right\|$,且只
    在 PONO map 大于 0.5 的点计算损失。
  \item 分类:采用交叉熵 loss:
    $\mathcal{L}_{\mathrm{cls}}=\mathrm{BCE}(P, \hat{P})$,其中 $\hat{P}$ 的值在
    PONO map 与该点 DT 与 GT 的 IoU 乘积大于 0.5 时为 1,其他情况为 0。
  \item 正则化:对不同类别、不同 anchor 回归和分类 loss 权重的正则项。
\end{itemize}

总 loss 形式为:
\begin{align}
  \mathcal{L} = & \frac{\lambda_{\mathrm{loc}}}{N^+}\sum_c \sum_a \lambda_{\mathrm{loc}}^{c,a} \sum_{i, j}\mathcal{L}_{\mathrm{loc}}(c,a,i,j) \, + \\
                & \frac{\lambda_{\mathrm{cls}}}{N}\sum_c \sum_a \lambda_{\mathrm{cls}}^{c,a} \sum_{i, j}\mathcal{L}_{\mathrm{cls}}(c,a,i,j) \, + \\
                & \log\frac{1}{\lambda_{\mathrm{cls}}} + \log\frac{1}{\lambda_{\mathrm{loc}}} + \frac{1}{KA}\sum_{c}\sum_{a}\left( \log\frac{1}{\lambda_{\mathrm{cls}}^{c,a}} + \log\frac{1}{\lambda_{\mathrm{loc}}^{c,a}} \right)
\end{align}

其中 $N^+$ 为 anchor 正样本数,$N$ 为总 pixel 数。

\chapter{Anchor Free 方法}
自从 Faster R-CNN 引入 anchor 概念后,几乎所有的目标检测文章中都应用了这一方
法,但基于 anchor 的方法有以下几个问题\citerb{2019-FCOS}:

\begin{enumerate}
  \item 模型性能对 anchor 的尺寸、比例等参数很敏感,需要大量调参。
  \item Anchor 为预设的固定尺寸,难以解决尺度变化问题,尤其是小物体的检测。
  \item 绝大多数 anchor 都是负样本,导致严重的正负样本不平衡。
  \item 为保证模型性能需要大量 anchor,计算量和存储需求高。
\end{enumerate}

为克服上述缺点,Anchor free 检测方法应运而生,成为目标检测的一个新方向。

\section{关键点方法}

\subsection{CornerNet}\label{sec:CornerNet}
CornerNet\cite{2018-CornerNet} 是 anchor free 方法的代表作之一,与传统目标检测
预测目标框位置和长/宽不同,CornerNet 直接预测左上/右下角点及配对关系。

\paragraph{网络结构}
CornerNet 的网络结构和预测模块的结构如图~\ref{fig:CornerNet-network}~和图~\ref{fig:CornerNet-prediction}~所示:

\begin{figure}[ht]
  \centering
  \includegraphics[width=0.8\textwidth]{images/目标检测/CornerNet-网络结构.pdf}
  \caption{CornerNet 网络结构}\label{fig:CornerNet-network}
\end{figure}

\begin{figure}[ht]
  \centering
  \includegraphics[width=0.8\textwidth]{images/目标检测/CornerNet-预测模块.pdf}
  \caption{CornerNet 的预测模块}\label{fig:CornerNet-prediction}
\end{figure}

网络的三个输出分别为:

\begin{itemize}
  \item 角点 heatmap:尺寸为原图下采样 stride 等于 n 的 heatmap,每个类
    别 1 个,heatmap上的值在 $[0, 1]$ 之间。
  \item 角点配对 embedding:用于左上/右下间角点的配对,要求两个角点的 embedding
    值小于一定阈值,所有类别共享相同的 embedding。
  \item 角点精修 offset:输出角点坐标的修正量,所有类别共享相同的 offset。
\end{itemize}

\paragraph{Corner Pooling 层}
Corner Pooling 层是 CornerNet 中为了提取 feature map 上左上和右下点提出的一种新
的池化层。以左上点的 corner pooling 为例,从该点出发,分别向右/向下找到本行/本列
最大值,将二者相加作为该点的值。实际计算时,可以由右向左、由下向上分别计算,这样
只需保留当前最大值。

\paragraph{损失函数}
CornerNet 的损失函数包括三个输出对应的损失函数:
\begin{equation}\label{eq:cornernet-loss}
  L = L_{\det} + \gamma L_{\mathrm{off}} + (\alpha L_{\mathrm{pull}} + \beta L_{\mathrm{push}})
\end{equation}

\begin{itemize}
  \item heatmap 损失函数 $L_{\det}$:类似 focal loss 形式,但 focal loss
    中 label 只有 0 和 1 两种,而 CornerNet 中 label 是 $[0, 1]$ 区间的连续值:
    \begin{equation}
      \label{equ:cornernet-det-loss}
      L_{\det} = -\frac{1}{N} \sum_{c=1}^{C} \sum_{i=1}^{H} \sum_{j=1}^{W}
      \left\{
        \begin{array}{lr}
           {(1-p_{cij})}^{\alpha}\log\,(p_{cij}) & y_{cij} = 1 \\
           {(1-y_{cij})}^{\beta} {(1-p_{cij})}^{\alpha} \log\,(p_{cij}) & y_{cij} < 1 \\
        \end{array}
      \right.
    \end{equation}
    上式中 $y_{cij}$ 的值在 GT 左上/右下角点为 1,但并非除角点外均为 0,而是存在一
    个衰减系数 $e^{-\frac{9(x^2+y^2)}{2\sigma^2}}$,其中 $\sigma$ 的确定方法为:
    在角点附近该范围以内的点组成的方框与 GT 的 IoU 均大于 0.3。
  \item Offset 损失函数 $L_{\mathrm{off}}$:stride 为 n 时,输入上的点 $(x, y)$
    对应的值为 $( \lfloor \frac{x}{n} \rfloor, \lfloor \frac{y}{n} \rfloor)$,
    offset 部分的真值 $\mathbf{o}_k = \left( \frac{x}{n} - \lfloor \frac{x}{n}
      \rfloor, \frac{y}{n} - \lfloor \frac{y}{n} \rfloor \right)$,loss 采用
    Smooth $L_1$ loss,且只在 GT 角点位置计算。
  \item Embedding 损失函数 $(\alpha L_{\mathrm{pull}} + \beta
    L_{\mathrm{push}})$:类似 triplet loss,同一 GT 的 embedding 尽可能小,不
    同 GT 的尽可能大,同样只在 GT 角点位置计算:
    \begin{align}
      \label{equ:cornernet-em-loss-pull}
      L_{\mathrm{pull}} & = \frac{1}{N} \sum_{k=1}^{N} \left[ {(e_{t_k} - e_k)}^2 + {(e_{b_k} - e_k)}^2 \right] \\
      \label{equ:cornernet-em-loss-push}
      L_{\mathrm{push}} & = \frac{1}{N(N-1)} \sum_{k=1}^{N} \sum_{\substack{j=1 \\ j \neq k}}^{N} \\max (0, 1-|e_k-e_j|)
    \end{align}
\end{itemize}

\paragraph{Inference 步骤}
\begin{itemize}
  \item 获得角点:先对 heatmap 做 $3 \times 3$ 的 max pooling,然后挑选所有类
    别 heatmap 中 top 100 的左上点和右下点。
  \item 精修角点:利用 offset 精修角点位置。
  \item 获得左上/右下点对:利用 embedding 计算点对,筛掉 embedding 相差大于 0.5
    或两个角点不属于同一个类别的点对,获得最终结果,最终分数为左上/右下角点
    heatmap 对应点的平均值。
\end{itemize}

\subsection{ExtremeNet}\label{sec:ExtremeNet}
ExtremeNet\citerb{2019-ExtremeNet}借鉴了 CornerNet 的思想,预测的是四个方向上的极值
点和中心点。

\paragraph{网络结构}
ExtremeNet 沿用了 CornerNet 中预测 heatmap 和 offset map 的结构,共预测上、下、
左、右及中心 5 个 heatmap,以及除中心外共 $4 \times 2$ 个 offset map。正负样本及
loss 也和 CornerNet 相同。

\paragraph{极值点的组合}
CenterNet 中利用 embedding 信息关联左上/右下角点,ExtremeNet 中获得极值点组合的
方法是:

\begin{itemize}
  \item 获得上/下/左/右 heatmap 局部极值点:局部极值点满足的条件是值大
    于 $\tau_p=0.1$,且在周围 $3 \times 3$ 的区域内为最大值。
  \item 计算中心点:枚举所有局部极值点组合,计算中心点坐标,其对应的 heatmap 值需
    要大于 $\tau_c=0.1$。
\end{itemize}

\paragraph{Ghost box 抑制}
Ghost box 是指包含多个较小检测框的大的检测框,文中提出一种抑制 ghost box 的方法,
如果一个检测框内部包含检测框的分数之和大于其自身分数的 3 倍,则将其分数除以 2,
这个做法类似 \hyperref[subsec:soft-nms]{Soft NMS}。

\paragraph{边缘增强}
找到局部极值点后,分别沿水平/竖直方向找到单调递减的最外点,然后更新局部极值点的
值,$\tilde{Y}_m = \hat{Y}_m + \lambda_{\mathrm{aggr}}\sum_{i=i_0}^{i_1}Y_i^m$,
其中 $\lambda_{\mathrm{aggr}}=0.1$。

\subsection{RepPoints}

\section{中心 + 变换方法}
\subsection{CenterNet}\label{sec:CenterNet}
CenterNet 和 ExtremeNet 类似,都出自 UT Austin 的 Zhou Xingyi,且同样借
鉴 CornerNet 的思路,但不再预测角点及其组合,而是直接预测中心点和物体的长和
宽\citerb{2019-CenterNet}。

\paragraph{网络结构和损失函数}
CenterNet 沿用了 CornerNet 中预测 heatmap 和 offset map 的结构,每个类别预测一个
中心点的 heatmap 及长和宽两个方向的 offset。此外还多预测长和宽方向检测框的尺寸,
除中心点 heatmap 外,offset 和尺寸所有类别共享同一个,loss 均为 Smooth $L_1$
loss,总的 loss 为:
\begin{equation}\label{equ:extreme-net-loss}
  L_{\det} = L_k + \lambda_{\mathrm{size}}L_{\mathrm{size}} + \lambda_{\mathrm{off}}L_{\mathrm{off}}
\end{equation}

其中 $\lambda_{\mathrm{size}}=0.1, \lambda_{\mathrm{off}}=1$。

\subsection{FCOS}\label{sec:FCOS}
FCOS 的整体结构和 RetinaNet 非常相似,但不再使用 bounding box anchor,而直接使用
feature map 上的点作为 anchor,网络的分类和回归也有一定区别\citerb{2019-FCOS}:

\begin{itemize}
  \item 分类:FCOS 中,正负样本对应 feature map 上的点,而 RetinaNet 的正负样本对应
    的是与点相关联的 anchor,同时 FCOS 还多预测一个中心度(Center-ness)。
  \item 回归:FCOS 直接回归 feature map 上的点到 GT box 上、下、左、右的距离,而
    非 RetinaNet 中 anchor 到 GT 的变换。
\end{itemize}

\paragraph{网络结构和损失函数}
FCOS 的网络结构参考 RetinaNet,由于使用了 FPN 结构,每个尺度都包括三部分输出:
\begin{itemize}
  \item 分类:输出 $H \times W \times C$ 个 map,其中 $C$ 为类别数,与RetinaNet 相
    同,均为二分类。由于每个点只对应一个值,与 RetinaNet 相比,减少了 $A$ 倍,其
    中 $A$ 为 RetinaNet 中每个点的 anchor 数。分类损失函数为 focal loss。
  \item 回归:输出 $H \times W \times 4$ 个 map,所有分类共享回归的距离值。损失
    函数为 IoU loss。
  \item 中心度:输出 $H \times W \times 1$ 个 map,所有分类共享中心度值。损 失函
    数为交叉熵 loss。
\end{itemize}

\paragraph{训练 label}
\begin{itemize}
  \item 分类 label:根据点与四条边距离的最大值分配到 FPN 的不同尺度,在对应尺度
    内的 GT 框内为正样本,否则为负样本。如果一个点对应多个 GT,选择面积最小的 GT
    作为该点对应的 GT。
  \item 回归 label:直接根据定义计算距离。
  \item 中心度 label:利用式 $\sqrt{\frac{\\min(l^*,
        r^*)}{\\max(l^*, r^*)} \times \frac{\\min(t^*,
        b^*)}{\\max(t^*, b^*)}}$ 计算,当点位于中心时取极大值 1,边缘时取极小
        值 0。
\end{itemize}

\paragraph{Inference}
Inference 时检测框的分数为分类分数和中心度的乘积,只取分类分数大于 0.05 的值生成
检测框,之后做 NMS 得到最终的检测结果。

\subsection{FoveaBox}
文献\citerb{2019-FoveaBox}同样以 RetinaNet 为基础,提出 FoveaBox 网络。

\paragraph{网络结构和损失函数}
\begin{itemize}
  \item 分类分支:最后一个卷积层的输出通道数为 $K$,对应的输出为 $W \times H
    \times K$的 heatmap,其中 $K$ 为类别数。损失函数为 focal loss。
  \item 回归分支:最后一个卷积层的输出通道数为 4,为该点坐标到 GT 的变换,具体参
    考训练 label 部分。损失函数为 Smooth $L_1$ loss。
\end{itemize}

\paragraph{训练 label}
FoveaBox 仍然借鉴 FPN 的思想,根据 GT 的尺寸分配 scale,不在 scale 负责范围内的
GT 为忽略样本。
\begin{itemize}
  \item 分类 label:GT 中心附近 $\sigma_1=0.3$ 倍长和宽内的区域为正样本,
    $\sigma_2=0.4$ 倍长和宽外的区域为负样本,中间为忽略样本。
  \item 回归 label:网络输出的变换值与 GT 的关系为:
    \begin{align}
      t_{x1} & = \\log \frac{2^l (x+0.5)-x_1}{z} \\
      t_{y1} & = \\log \frac{2^l (y+0.5)-y_1}{z} \\
      t_{x2} & = \\log \frac{x_2 - 2^l (x+0.5)}{z} \\
      t_{y2} & = \\log \frac{y_2 - 2^l (y+0.5)}{z}
    \end{align}
    其中 $z = \sqrt{S_l}$。根据变换关系可以计算对应的 label 值。
\end{itemize}

\subsection{FSAF}
文献\citerb{2019-FSAF}在 RetinaNet 基础上加入 anchor free 平行分支,训练时通过选
取不同 scale 中最小的 loss 进行反传,解决了人为给定 anchor 和 scale 对应关系并非
最优的问题。

\paragraph{网络结构}
FSAF 的网络结构如图~\ref{fig:FSAF}~所示:

\begin{figure}[ht]
  \centering
  \includegraphics[width=0.8\textwidth]{images/目标检测/FSAF.pdf}
  \caption{FSAF 网络结构}\label{fig:FSAF}
\end{figure}

在 RetinaNet 每个尺度分类和回归的最后一个 feature map 上,分别接对应
的 anchor free 分类/回归平行分支。

\begin{itemize}
  \item 分类分支:接 $3 \times 3$ 的卷积和 sigmoid,输出尺寸为 $W \times H \times
    K$ 的 heatmap,每个点取值区间为 [0, 1],值越大表示越可能是物体中心。
  \item 回归分支:接 $3 \times 3$ 的卷积和 ReLU,输出尺寸为 $W \times H \times
    4$ 的 heatmap,每个点均为正值,表示该点到上/下/左/右四条边的距离。
\end{itemize}

\paragraph{损失函数}
FSAF 的分类分支使用 focal loss,回归分支使用 IoU loss。

\paragraph{训练 label}
\begin{itemize}
\item 分类 label:与~\hyperref[subsubsec:GA-RPN]{GA-RPN}~相同。
  \item 回归 label:为像素点与四条边的距离值,只在 $b_e$ 范围内有值,其余均为忽略区域。
\end{itemize}

\paragraph{训练和测试}
\begin{itemize}
  \item 训练:多个 scale 同时计算每个 instance 的 loss,取最小的作为结果进行反传。
  \item 测试:由于最优 scale 输出的分数更高,相当于模型会自适应选择最合适
    的 scale。取所有 scale 中分类分数 > 0.05 的 top 1000 结果,进行阈值
    为 0.5 的 NMS,得到最终结果。
\end{itemize}

\subsection{SAPD}
文献\citerb{2019-SAPD}提出 SAPD,基础方法仍然采用 FSAF,但改进了训练策略。

\paragraph{Anchor points 权重}
FSAF 中,分类 heatmap 上所有点的权重均相同。SAPD 降低了分类 heatmap 上靠近物体边
缘点的权重,即离中心越近,权重越高,离边缘越近,权重越低,权重的具体计算方法为:
\begin{equation}
  w_{lij} = {\left( \frac{\\min \left( d_{lij}^l, d_{lij}^r \right) \cdot \\min \left( d_{lij}^t, d_{lij}^b \right)}{\\max \left( d_{lij}^l, d_{lij}^r \right) \cdot \\max \left( d_{lij}^t, d_{lij}^b \right)} \right)}^{\eta}, \; p_{lij} \in p^+
\end{equation}

\paragraph{Feature map 权重}
FSAF 中,每个 instance 的反传只选择 loss 最小的 scale。SAPD 同时选择 $k$ 个
scale 的 feature map 计算 loss,每个 scale 对应的权重由一个小的网络生成。该网络
的结构如下:
\begin{itemize}
  \item 对 $k$ 个 feature map 的 GT 做 RoI Align,得到 $C \times 7 \times 7$ 的 feature。
  \item 经过 3 个 $3 \times 3$ 卷积 + ReLU,得到 $C \times 1 \times 1$ 的 feature。
  \item 经过 FC + Softmax 得到 $k$ 个值,作为不同 scale feature map loss 的权重。
\end{itemize}

\chapter{专题}
\section{Scale 问题}
目标检测的 Scale 问题指的是单个模型需要同时检测不同尺寸的物体,一般而言,模型对小
物体的检出能力相对较差。

\subsection{Feature map 信息融合}
对于多尺度 feature map 同时预测的目标检测模型,小物体的检测一般在较浅层的
feature map 上进行。浅层 feature map 的一个问题是感受野较小,因此上下文语义信息
较弱。FPN、DSSD、RetinaNet 等模型都引入了 feature map 信息融合,即将深层 feature
map 上采样或反卷积,再与浅层 feature map 融合,得到尺寸较大同时上下文语义信息更
强的 feature map 用于最终预测。

\subsubsection{ASFF}
文献\citerb{2019-ASFF}中提出自适应空间融合(Adaptive Spatial Feature Fusion,
ASFF)方法,用于融合不同尺度的 feature map 信息。

\paragraph{Feature map resize}
\begin{itemize}
  \item 上采样:$1 \times 1$ 卷积,压缩通道,再进行上采样。
  \item 下采样:$3 \times 3, s = 2$ 卷积,压缩通道。
\end{itemize}

\paragraph{自适应融合}
三个尺寸的 feature map 融合后生成三个新的 feature map:
\begin{equation}
  y_{ij}^l = \alpha_{ij}^l \cdot x_{ij}^{1 \rightarrow l} + \beta_{ij}^l \cdot x_{ij}^{2 \rightarrow l}
  + \gamma_{ij}^l \cdot x_{ij}^{3 \rightarrow l}
\end{equation}

其中权重满足 $\alpha_{ij}^l + \beta_{ij}^l + \gamma_{ij}^l = 1$。

\subsection{图像金字塔}

\subsubsection{SNIP}
文献\citerb{2017-SNIP}中提出 SNIP(Scale Normalization for Image Pyramids)方法,
相当于融合改进的多尺度训练和图像金字塔方法:

\begin{itemize}
  \item 核心思想:训练和测试都使用图像金字塔,但不同分辨率的图像只负责对应
    scale 的物体。
  \item 训练:RPN 网络,不同分辨率图片,只有对应 scale 范围内的 GT 为有效 GT,
    忽略其他 GT。与忽略 GT IoU 大于 0.3 的 anchor 设为忽略样本,不参与当前 scale
    图片的训练;RCNN 网络类似,同样只选择 scale 范围内的 proposal 和 GT。
  \item 测试:同样使用不同分辨率图片经过 RPN 得到 proposal,但不对 proposal 做筛
    选,全部送到 RCNN 中做分类和回归,得到最终的 detection。最终的 detection 只
    有在与图片分辨率对应 scale 范围内的才认为有效,之后利用 Soft NMS 对不同分辨
    率图片生成的 detection 做后处理。
  \item 高分辨率图片裁剪:为解决分辨率为 1400 $\times$ 2000 的图片占用显存太大的
    问题,文中将其裁成 1000 $\times$ 1000 的图片,称为 chips。具体实现是每一张
    图片随机生成 50 个 chips,贪心挑选其中包含物体最多的 chip,不断进行这个过程
    直到挑出的 chips 包含图中所有物体。
\end{itemize}

\subsubsection{SNIPER}
文献\citerb{2018-SNIPER}中延续 SNIP 的思路,提出了图像金字塔的改进版 SNIPER,训
练时不再使用图像金字塔的原图,而是使用从原图中 crop 得到的 chips 作为训练图片,
提升模型性能。

\paragraph{Chips 的选取策略}
Chips 指的是图像金字塔的子图,文中给出了 chips 的选取策略。

\begin{itemize}
  \item 正样本较多的 chips:将原图缩放到三个尺度,再统一缩放到固定尺寸,然后每隔
    $d$ 个像素放置 $K \times K$ 的 chips,文中 $d = 32,\,K = 512$。不同尺度的图
    片有对应的 scale 范围,只有在范围内的 GT 才认为有效,贪心选取包含有效 GT 最
    多的前 $n$ 个 chip。
  \item 误检较多、正样本较少的 chips:利用未完全训练的 RPN 生成 proposal,去掉
    proposal 正样本,选择至少包含 $M$ 个 proposal 负样本的 chips。
\end{itemize}

\paragraph{SNIPER 的优势}
\begin{itemize}
  \item 只选择对应尺度的 GT 作为有效 GT 参与训练,部分解决不同尺度 GT 对模型性能
    的影响。
  \item 负样本相当于做了 HEM,保留了可能出现 FP 的 chip,去掉了简单 chip。
  \item 训练时可以同时使用 5 个不同尺度图片对应的 chips,batch size 更大,同时可
    以应用 BN。
\end{itemize}

\subsubsection{TridentNet}
文献\citerb{2019-TridentNet}提出双阶段目标检测模型 TridentNet,网络利用多个平行
且共享参数的空洞卷积 branch 代替原始的卷积,同时结合 SNIP 训练方法,不同 branch
只使用对应 scale 的 GT 进行训练。

\paragraph{网络结构}
TridentNet 的结构如图~\ref{fig:tridentnet}~所示:

\begin{figure}[ht]
  \centering
  \includegraphics[width=0.8\textwidth]{images/目标检测/TridentNet.pdf}
  \caption{TridentNet 结构}\label{fig:tridentnet}
\end{figure}

TridentNet 将 ResNet 最后一个 block 中的 $3 \times 3$ 卷积分别替换成 stride 为
1、2、3 的空洞卷积,共有三个平行的 branch,这三个 branch 共享参数,但感受野不同。

\paragraph{训练时与 SNIP 的结合}
训练时应用 SNIP,每个 branch 都有对应的有效 range,只有 GT 落在有效 range 内才认
为有效,否则为忽略。RPN 中,anchor 的正负样本根据有效 GT 确定,R-CNN 中,只选择
在有效 range 内的 proposal。

\subsubsection{CSN}
文献\citerb{2019-CSN}中提出 CSN 方法,改进了 SNIP scale 超参较多的问题,同时结合
FPN 多尺度预测,进一步提升小物体的检测性能。

\paragraph{SNIP 的问题}
SNIP 先将图片缩放到不同 scale,每个 scale 都有一个有效范围 $[l_s, u_s]$,只有 GT
落在该范围内才认为有效。有效范围的上下界为超参,需要仔细调整不同 scale 的有效范
围值才能获得较好的性能,调参工作量大。

\paragraph{CSN 的改进}
CSN 的有效范围只有两个参数 $[l, r]$,将所有图片按照预先设定的 scale 缩放,只有
GT 落在有效范围内才认为有效,测试时同样使用图像金字塔,只保留落在有效范围内的预
测结果,最后融合不同 scale 图片的输出作为最终结果。文中有效范围为 [16, 560],使
用 FPN 辅助解决尺度变化大的问题。

\section{损失函数}
\subsection{回归损失函数}
目标检测的代表性模型 Fast R-CNN 和 YOLO 分别使用了 Smooth $L_1$ 和 $L_2$ 形式的
损失函数作为回归 loss,但最终评测目标检测模型性能使用的指标一般是检测框与 GT 的
IoU,二者并不一致。

\subsubsection{IoU loss}
为统一训练 loss 和评测指标,文献\citerb{2016-IoU-loss}中提出 IoU loss 作为目标检
测模型回归部分的损失函数:
\begin{equation}
  L_{\mathrm{IoU}} = - \ln (\mathrm{IoU})
\end{equation}

\subsubsection{GIoU loss}
IoU loss 的一个致命问题是输出检测框与 GT IoU 为 0 时无法进行优化,为解决这一问题,
文献\citerb{2019-GIoU-loss} 提出了 GIoU loss:
\begin{equation}
  L_{\mathrm{GIoU}} = 1 - \left( \mathrm{IoU} - \frac{C - A \cup B}{C} \right)
\end{equation}

上式中 A,B 分别为检测框和 GT,C 为二者的最小包络矩形。

\subsubsection{Distance-IoU loss}
文献\citerb{2019-Distance-IoU-loss}中指出 GIoU loss 存在收敛慢和准确性低的问题,
提出了 Distance IoU 和 Complete IoU 两种 loss:
\begin{align}
  \mathcal{L} & = 1 - \mathrm{IoU} + \mathcal{R}(B, B^{gt}) \\
  \mathcal{R}_{\mathrm{DIoU}} & = \frac{\rho^2(b, b^{\mathrm{gt}})}{c^2} \\
  \mathcal{R}_{\mathrm{CIoU}} & = \frac{\rho^2(b, b^{\mathrm{gt}})}{c^2} + \alpha v \\
  v & = \frac{4}{\pi^2}\left( \arctan\frac{w^{\mathrm{gt}}}{h^{\mathrm{gt}}} - \arctan \frac{w}{h} \right) \\
  \alpha & = \frac{v}{1 - \mathrm{IoU} + v}
\end{align}

Distance IoU loss 可以有效改善 GIoU loss 在两个框水平或垂直对齐时,收敛速度慢导
致的精度较差问题。

\section{不平衡问题}
\subsection{正负样本不平衡}

传统的解决正负样本不平衡问题的方法包括 HEM 和 focal loss。

\subsubsection{Sampling-Free}
文献\citerb{2019-Sampling-Free}引入了三种手段,包括改进分类层初始化方法、自适应分
类/回归 loss 权重和自适应类别前景阈值,提出 Sampling-Free 的训练方法,可以达到甚
至超过原始检测器的精度。

\begin{itemize}
  \item 分类网络卷积层偏置初始值:RetinaNet 中,偏置的初始值 $b =
    -\\log\frac{1-\pi}{\pi}$,其中 $\pi=0.01$。如果不采用 focal loss,直
    接训练网络会出现 loss 爆炸,性能很差。文中将 $\pi$ 的值降为 $10^{-5}$,再计
    算对应的 $b$ 作为初始值。
  \item 自适应 loss 权重:总 loss 形式为:
    \begin{equation}
    L = w^{\mathrm{reg}} L^{reg} +
    w^{\mathrm{cls}} L^{\mathrm{cls}} \cdot r = \left( w^{\mathrm{cls}} + 1
    \right) w^{\mathrm{reg}} L^{\mathrm{reg}}
    \end{equation}
    令 $w^{\mathrm{reg}} = 1$,则只需要调整 $w^{\mathrm{cls}}$ 的值即可,相当于按
    照分类/回归 loss 比例将分类 loss 归一化。
  \item 自适应类别前景阈值:每个类别的前景阈值 $\theta$ 的计算方式为:
    \begin{equation}
      \theta_j = \frac{N_f}{N} \cdot \frac{N_j}{\sum_{j=1}^{C} N_j}
    \end{equation}
    其中 $N_f$ 为前景样本数,$N$ 为样本总数,$N_j$ 为第 $j$ 类的前景样本数。
\end{itemize}

\section{不一致问题}
\subsection{分类分数与位置精度不一致}
目标检测模型通常将检测框的分类分数作为 NMS 的依据,即优先保留分类分数高的检测框。
但检测框的分数与位置精度并非简单的正相关,有可能出现高分位置不准的检测框杀掉低分
位置准的检测框。

\subsubsection{LTR}
LTR 利用子网络预测每个 proposal 的排序分数,目标是排序分数与 proposal 和 GT 的
IoU 正相关\citerb{2019-LTR}。

\begin{itemize}
  \item 子网络:类似分类网络,RoI feature 接两个 fc 层得到排序分数。
  \item 正负样本:借鉴 HEM 思想,将 proposal 按照与 GT 的 IoU 划分成 $n$ 个区间,
    每个区间内的 proposal 作为正样本,负样本为低区间所有 proposal 中分数最高
    的 $h$ 个。
  \item 损失函数:Hinge loss 形式。
\end{itemize}

\subsubsection{Noisy Anchor}
Noisy Anchor 的改进包括以下两方面:

\begin{itemize}
  \item 自适应分类 soft label:正样本的分类 label 不再是 1,而是与分类和回归分支
    的输出相关
    \begin{equation}
      c = \alpha \cdot \mathtt{loc\_a} + (1 - \alpha) \cdot \mathtt{cls\_c}
    \end{equation}
    其中 $\mathtt{loc\_a}$ 为回归后的 bbox 与 GT 的 IoU。
  \item 自适应权重:正样本的权重 $r$ 同样与分类和回归分支的输出相关
    \begin{equation}
      r = {\left( \alpha \cdot \frac{1}{1-\mathtt{loc\_a}} + (1 - \alpha) \cdot \frac{1}{1-\mathtt{cls\_c}} \right )} ^ {\gamma}
    \end{equation}
    将所有正样本权重的平均值归一化为 1。
\end{itemize}

\subsection{分类分数与检测框对应关系不一致}
单阶段检测模型中,分类分支的 label 与原始 anchor 的位置相关,可能出现由于原
始 anchor为负样本,回归后的检测框位置较准但分数较低的情况。

\subsubsection{ConRetinaNet}

\subsubsection{HSD}
HSD 借鉴了 cascade 的思路,同时引入 reg-offset-cls(ROC) 模块,利用回归后 bbox 的
信息获得 offset 再结合 deformable 卷积作为分类分支\citerb{2019-HSD}。
图~\ref{fig:hsd}~为 HSD 的具体结构:

\begin{figure}[ht]
  \centering
  \includegraphics[width=0.9\textwidth]{images/目标检测/HSD.pdf}
  \caption{HSD 的结构}\label{fig:hsd}
\end{figure}

\begin{itemize}
  \item 检测分支与传统单阶段网络类似,在输出后再接两个 $1 \times 1$ 的卷积网络,
    输出 $18N$ 的 offset。
  \item 分类分支利用 deformable 卷积代替传统卷积,offset 为回归分支的输出。
  \item Cascade 两级之间的特征增强模块包括三个并列分支,分别为 $3 \times 3$ 卷积,
    $3 \times 3, stride=2$ 的卷积和 non-local 模块。
  \item 第一级类似 RPN,可以只做二分类,第二级再做 $C+1$ 分类。
\end{itemize}

%%% Local Variables:
%%% TeX-master: "../master"
%%% End:
