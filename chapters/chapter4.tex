\part{编程语言}
\chapter{C++}

\section{基础}
\subsection{基本类型}
\subsubsection{指针和引用的区别}
\begin{enumerate}
  \item 可否为空:有空指针(C++11 标准推荐 \texttt{nullptr}),但没有空引用,引用必须初始化。
  \item 可否改变指向的对象:指针可以改变,引用初始化之后不能改变。
\end{enumerate}

\subsection{关键字}
\subsubsection{static}
static 关键字主要用于以下四种情况:

\begin{itemize}
  \item 静态局部变量
  \item 静态全局变量
  \item 静态数据成员
  \item 静态成员函数
\end{itemize}

\paragraph{静态局部变量}
静态局部变量指在函数中利用 static 关键字声明的局部变量。

\begin{itemize}
  \item 初始化:程序首次执行到对象声明时进行初始化,后续不再初始化。如果没有显式
    初始化,将进行值初始化,内置类型变量初值为 0。
  \item 存储位置:存储在全局数据区,而非像函数内的局部变量存储在栈内。
  \item 作用域:虽然存储在全局数据区,但作用域仍然为局部作用域,只能在对应的函数
    内使用。
\end{itemize}

\paragraph{静态全局变量}
静态全局变量指在函数外利用 static 关键字声明的变量。

\begin{itemize}
  \item 初始化:程序执行前初始化。如果没有显式初始化,将进行值初始化,内置类型变量初值为 0。
  \item 存储位置:存储在全局数据区。
  \item 作用域:只能在声明静态全局变量的文件中使用,其他文件中不能使用,但可以定
    义相同名称的变量。
\end{itemize}

C++ 中,通常通过在匿名命名空间中声明变量实现静态全局变量,不再使用 static 关键字。

\paragraph{静态数据成员}
静态数据成员指类中利用 static 关键字声明的数据成员。

\begin{itemize}
  \item 初始化:只能在类内声明,类外定义并初始化。如果没有显式初始化,将进行值初始
    化,内置类型变量初值为 0。
  \item 存储位置:存储在全局数据区,类的所有对象共享同一个静态数据成员。
  \item 作用域:程序执行的整个过程。
\end{itemize}

\paragraph{静态成员函数}
静态成员函数指类中利用 static 关键字声明的成员函数。静态成员函数只能访问类的静态
数据成员和调用其他静态成员函数,不能访问非静态成员和调用非静态成员函数,也没有
this 指针。

\subsection{变量}
\subsubsection{作用域}
C++ 中,根据作用域的不同,可以将对象分为局部对象、全局对象和动态对象。

\begin{table}[htbp]
  \centering
  \caption{C++ 的变量作用域}
  \label{tab:cpp-variable-scope}
  \begin{tabular}{cccc}
    \specialrule{0em}{10pt}{1pt}
    \toprule[1.5pt]
    {\heiti 名称} & {\heiti 作用域} & {\heiti 相关关键字} & {\heiti 存储区域} \\
    \midrule[1pt]
    局部对象       & 函数/块作用域内    & 无         & 栈 \\
    静态全局对象   & 所在文件           & static     & 静态存储区 \\
    全局对象       & 整个程序           & 无         & 静态存储区 \\
    动态对象       & 申请到释放内存期间 & new/delete & 堆 \\
    \bottomrule[1.5pt]
  \end{tabular}
\end{table}

\subsubsection{初始化}
C++ 中定义变量时如果没有指定初值,则会根据变量类型和位置进行对应的初始化。

\begin{itemize}
  \item 内置类型:非静态局部变量将不被初始化,拥有未定义的值;静态局部变量和全局
    变量进行值初始化,初值为 0。
  \item 类类型:与位置无关,统一调用默认构造函数生成对象,如果类不支持默认初始化,
    要求显式提供初值,将会出现编译错误。
\end{itemize}

\section{面向对象编程}
\subsection{面向对象编程的三大基本特征}

\subsubsection{封装}
封装是指将对象抽象成具体的类,同时进行接口和实现的分离,外界只能通过接口与类的对
象交互,将内部的信息隐藏。

\subsubsection{继承}
继承是指子类继承父类的数据成员和函数,并根据需要进行扩展或重写,无需完全重写新类,
实现代码的复用。

\subsubsection{多态}
多态是指通过将父类的指针或引用指向子类对象,在运行时根据所指对象类型实现对应功能,
即同一个父类指针通过指向不同的对象,可以表现出不同形态。


\chapter{Python}


%%% Local Variables:
%%% TeX-master: "../master"
%%% End:
