\part{数据结构和算法}

\chapter{数据结构}

\section{栈}
栈是先进后出的基本数据结构,实际上程序在递归调用时都需要使用栈,但这种情况一般由
编译器隐式完成。

\subsection{括号题目}
括号是典型的满足先进后出特点的一种结构,所以和括号相关的题目一般都可以利用栈来解
决,例如以下题目:

% LC 301. Remove Invalid Parentheses(Stack + BFS)
% LC 678. Valid Parenthesis String(min/max count)

\begin{itemize}
  \item
    \href{https://leetcode.com/problems/longest-valid-parentheses/}{LeetCode 32. Longest Valid Parentheses}
  \item
    \href{https://leetcode.com/problems/minimum-add-to-make-parentheses-valid/}{LeetCode
      921. Minimum Add to Make Parentheses Valid}
  \item
    \href{https://leetcode.com/problems/minimum-remove-to-make-valid-parentheses/}{LeetCode
      1249. Minimum Remove to Make Valid Parentheses}
\end{itemize}

以下题目需要利用哈希表存储不同层级括号间的信息:

\begin{itemize}
\item
  \href{https://leetcode.com/problems/score-of-parentheses/}{LeetCode 856. Score of Parentheses}
\item
  \href{https://leetcode.com/problems/reverse-substrings-between-each-pair-of-parentheses/}{LeetCode
    1190. Reverse Substrings Between Each Pair of Parentheses}
\end{itemize}

\subsection{计算器题目}

\subsection{文本解析题目}

\section{队列}

\section{优先队列}

\section{链表}

\section{哈希表}

\section{树}

\section{图}

\subsection{拓扑排序}
拓扑排序是指将一个有向无环图(Directed Acyclic Graph,DAG)G 中的所有顶点排成一个
线性序列,使得对任意一对顶点 $u, v$,若边 $(u, v) \in G$,则在序列中 $u$ 出现
在 $v$ 之前。

拓扑排序相关题目:
\begin{itemize}
  \item
    \href{https://leetcode.com/problems/course-schedule/}{LeetCode 207. Course Schedule}
  \item
    \href{https://leetcode.com/problems/course-schedule-ii/}{LeetCode 210. Course Schedule II}
  \item
    \href{https://leetcode.com/problems/alien-dictionary}{LeetCode 269. Alien Dictionary}
  \item
    \href{https://leetcode.com/problems/sequence-reconstruction}{LeetCode 444.
      Sequence Reconstruction}
  \item
    \href{https://leetcode.com/problems/parallel-courses}{LeetCode 1136. Parallel Courses}
  \item
    \href{https://leetcode.com/problems/sort-items-by-groups-respecting-dependencies/}{LeetCode
      1203. Sort Items by Groups Respecting Dependencies}
\end{itemize}

\section{前缀树}
\section{线段树}

\chapter{算法}
\section{遍历}
\subsection{深度优先搜索}

\subsection{广度优先搜索}

\subsection{最优优先搜索}

\section{并查集}

\section{动态规划}

\subsection{一维动态规划}

\subsection{二维动态规划}
二维动态规划的一个典型场景是字符串的公共或回文子串问题:
\begin{itemize}
  \item
    \href{https://leetcode.com/problems/longest-palindromic-subsequence/}{LeetCode
      516. Longest Palindromic Subsequence}:可转化为求字符串与其自身倒序的 LCS,
    即转化为 LeetCode 1143。
  \item
    \href{https://leetcode.com/problems/longest-common-subsequence/}{LeetCode 1143. Longest Common Subsequence}
  \item
    \href{https://leetcode.com/problems/palindrome-removal/}{LeetCode 1246. Palindrome Removal}
\end{itemize}

\section{贪心}

\section{二分查找}
二分查找的典型应用场景包括:

\begin{itemize}
  \item 在有序序列(数组、矩阵等)中查找某个值。
  \item 在一定范围内查找满足某个条件的最大/最小值。
\end{itemize}

\subsection{有序序列查找}

\subsection{范围查找}
以下题目可以直接写出二分的判断条件:

\begin{itemize}
  \label{lc:bs-range-general}
  \item
    \href{https://leetcode.com/problems/split-array-largest-sum/}{LeetCode 410. Split Array Largest Sum}
  \item
    \href{https://leetcode.com/problems/koko-eating-bananas/}{LeetCode 875. Koko Eating Bananas}
  \item
    \href{https://leetcode.com/problems/capacity-to-ship-packages-within-d-days/}{LeetCode
      1011. Capacity To Ship Packages Within D Days}
\end{itemize}

以下题目需要结合一定的数学推导写判断条件:

\begin{itemize}
  \label{lc:bs-range-math}
  \item
    \href{https://leetcode.com/problems/nth-magical-number/}{LeetCode 878. Nth
      Magical Number}:容斥原理。
  \item
    \href{https://leetcode.com/problems/ugly-number-iii/}{LeetCode 1201. Ugly
      Number III}:LeetCode 878 扩展版,两个数变为三个数。

\end{itemize}

\section{滑动窗口}

\section{数学}

这部分题目一般都是考察数学功底,需要一定的数学基础和推导计算才能完成:

\begin{itemize}
  \item
    \href{https://leetcode.com/problems/check-if-it-is-a-good-array/}{LeetCode.
      1250. Check If It Is a Good Array}:裴蜀定理。
\end{itemize}


%%% Local Variables:
%%% TeX-master: "../master"
%%% End:
