\part{深度学习基础}

\chapter{卷积神经网络}

\section{感受野}

感受野(Receptive Field,RF) 是指卷积神经网络中某一层的一个像素,会受到多少个输入数据像素的影响。直观上理解,就是该像素能看到多大范围的输入层像素 \cite{2017-Guide-to-RF-cal}。具体的计算方法为:
\begin{align}
j_{\mathrm {out}} & = j_{\mathrm{in}} \times s \\
r_{\mathrm {out}} & = r_{\mathrm{in}} + (k-1) \times j_{\mathrm{in}}
\end{align}

其中 $j$ 为 jump,即等效的 stride;s 为卷积的 stride,k 为卷积的 kernel size,r 为感受野。

经典网络 VGG-16 各层感受野的详细计算可以参考 \cite{2018-VGG-16-RF-cal}。