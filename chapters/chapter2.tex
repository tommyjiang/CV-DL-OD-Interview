\part{数据结构和算法}

\chapter{数据结构}

\section{栈}
栈是先进后出的基本数据结构,实际上程序在递归调用时都需要使用栈,但这种情况一般由
编译器隐式完成。

\subsection{括号题目}
括号是典型的满足先进后出特点的一种结构,所以和括号相关的题目一般都可以利用栈来解
决,例如以下题目:

% LC 678. Valid Parenthesis String(min/max count)

\begin{itemize}
  \item
    \href{https://leetcode.com/problems/minimum-add-to-make-parentheses-valid/}{LeetCode
      921. Minimum Add to Make Parentheses Valid}
  \item
    \href{https://leetcode.com/problems/minimum-remove-to-make-valid-parentheses/}{LeetCode
      1249. Minimum Remove to Make Valid Parentheses}
  \item
    \href{https://leetcode.com/problems/longest-valid-parentheses/}{LeetCode 32. Longest Valid Parentheses}
\end{itemize}

以下题目需要利用哈希表存储不同层级括号间的信息:

\begin{itemize}
  \item
    \href{https://leetcode.com/problems/score-of-parentheses/}{LeetCode 856. Score of Parentheses}
  \item
    \href{https://leetcode.com/problems/reverse-substrings-between-each-pair-of-parentheses/}{LeetCode
    1190. Reverse Substrings Between Each Pair of Parentheses}
\end{itemize}

\subsection{计算器题目}

\subsection{文本解析题目}

相关题目:

\begin{itemize}
  \item 
    \href{https://leetcode.com/problems/decode-string/}{LeetCode 394. Decode
      String}:数字、字母、左右括号分别处理,利用栈记录当前和之前的 string。
\end{itemize}

\section{队列}

\section{优先队列(堆)}
优先队列的适用场合是在插入/删除频繁的数组中获得最小值,一个典型的实现是堆。优先
队列也是在难题中应用广泛的一种数据结构。

\begin{itemize}
  \item
    \href{https://leetcode.com/problems/design-a-leaderboard}{LeetCode 1244.
      Design A Leader board}:堆的直接应用,也可以使用快排获得前 k 大的数。
  \item
    \href{https://leetcode.com/problems/minimum-cost-to-connect-sticks}{LeetCode
      1167. Minimum Cost to Connect Sticks}:维护小顶堆。
\end{itemize}

\section{链表}

\section{哈希表}

\section{树}

\section{图}

\subsection{拓扑排序}
拓扑排序是指将一个有向无环图(Directed Acyclic Graph,DAG)G 中的所有顶点排成一个
线性序列,使得对任意一对顶点 $u, v$,若边 $(u, v) \in G$,则在序列中 $u$ 出现
在 $v$ 之前。

拓扑排序相关题目:
\begin{itemize}
  \item
    \href{https://leetcode.com/problems/course-schedule/}{LeetCode 207. Course Schedule}
  \item
    \href{https://leetcode.com/problems/course-schedule-ii/}{LeetCode 210. Course Schedule II}
  \item
    \href{https://leetcode.com/problems/parallel-courses}{LeetCode 1136. Parallel Courses(Premium)}
  \item
    \href{https://leetcode.com/problems/sequence-reconstruction}{LeetCode 444.
      Sequence Reconstruction}
  \item
    \href{https://leetcode.com/problems/alien-dictionary}{LeetCode 269. Alien Dictionary}
  \item
    \href{https://leetcode.com/problems/sort-items-by-groups-respecting-dependencies/}{LeetCode
      1203. Sort Items by Groups Respecting Dependencies}:双层拓扑排序
\end{itemize}

\section{前缀树}

\section{线段树}

\chapter{算法}
\section{遍历}
\subsection{深度优先搜索}
% LC 39. Combination Sum
深度优先搜索(Depth First Search,DFS)的应用非常广泛,基于递归的方法相当于程序
实现 DFS + 回溯,此外在树和图的遍历中也有普遍应用。

深度优先搜索 + 回溯题目:

\subsection{广度优先搜索}
广度优先搜索(Breath First Search,BFS)一般用在求最少次数/最短路径等问题中。

最少次数:
\begin{itemize}
  \item
    \href{https://leetcode.com/problems/remove-invalid-parentheses/}{LeetCode 301. Remove Invalid Parentheses}
\end{itemize}

最短路径:
\begin{itemize}
  \item
    \href{https://leetcode.com/problems/shortest-path-in-binary-matrix/}{LeetCode
      1091. Shortest Path in Binary Matrix}:八方向连通最短路径。
\end{itemize}

\subsection{最优优先搜索}
最优优先搜索是 DFS 和 BFS 的推广,一般通过维护最优堆进行搜索,在搜索时更新最优堆。
DFS 和 BFS 都可以看成是最优优先搜索的特例。相关问题:

\begin{itemize}
  \item
    \href{https://leetcode.com/problems/swim-in-rising-water/}{LeetCode 778. Swim in Rising Water}
  \item
    \href{https://leetcode.com/problems/path-with-maximum-minimum-value}{LeetCode
      1102. Path With Maximum Minimum Value}:与 LeetCode 778 相同。
\end{itemize}


\section{并查集}

\section{动态规划}

\subsection{一维动态规划}
% LC 91. Decode ways

\subsection{二维动态规划}
二维动态规划的一个典型场景是字符串的公共/回文子串和编辑距离问题:
\begin{itemize}
  \item
    \href{https://leetcode.com/problems/longest-common-subsequence/}{LeetCode 1143. Longest Common Subsequence}
  \item
    \href{https://leetcode.com/problems/longest-palindromic-subsequence/}{LeetCode
      516. Longest Palindromic Subsequence}:可转化为求字符串与其自身倒序的 LCS,
    即转化为 LeetCode 1143。
  \item
    \href{https://leetcode.com/problems/palindrome-removal/}{LeetCode 1246. Palindrome Removal(Premium)}
  \item
    \href{https://leetcode.com/problems/edit-distance/}{LeetCode 72. Edit Distance}
  \item
    \href{https://leetcode.com/problems/one-edit-distance}{LeetCode 161. One
      Edit Distance}:与 LeetCode 72 类似,只需要判断是否为 1。
  \item
    \href{https://leetcode.com/problems/minimum-ascii-delete-sum-for-two-strings/}{LeetCode
      712. Minimum ASCII Delete Sum for Two Strings}:LeetCode 72 升级版,只允许
    删除操作,同时代价不再为 1,而是字符的 ASCII 值。
\end{itemize}

\section{贪心}
贪心可以看做一种特殊的动态规划,即局部最优解一定是全局最优解,相当于动态规
划时每次无需考虑其他分支,只选择局部最优解即可。

最小生成树 Kruskal 算法:
\begin{itemize}
  \item 
    \href{https://leetcode.com/problems/optimize-water-distribution-in-a-village}{LeetCode
      1168. Optimize Water Distribution in a Village}:well 可转化为 0 号节点到
    所有节点的边,然后应用 Kruskal 算法,利用并查集判断是否连通。
\end{itemize}

\section{二分查找}
二分查找的典型应用场景包括:

\begin{itemize}
  \item 在有序序列(数组、矩阵等)中查找某个值。
  \item 在一定范围内查找满足某个条件的最大/最小值。
\end{itemize}

\subsection{有序序列查找}
二分查找的基本题型:

二分查找的变形题型:
\begin{itemize}
  \item
    \href{https://leetcode.com/problems/search-in-rotated-sorted-array/}{LeetCode
      33. Search in Rotated Sorted Array}:判断 left/right 和 mid 值的关系,得到
    有序部分,判断 target 是否在有序范围,重新查找。
\end{itemize}

\subsection{范围查找}
以下题目可以直接写出二分的判断条件:

\begin{itemize}
  \label{lc:bs-range-general}
  \item
    \href{https://leetcode.com/problems/split-array-largest-sum/}{LeetCode 410. Split Array Largest Sum}
  \item
    \href{https://leetcode.com/problems/koko-eating-bananas/}{LeetCode 875. Koko Eating Bananas}
  \item
    \href{https://leetcode.com/problems/capacity-to-ship-packages-within-d-days/}{LeetCode
      1011. Capacity To Ship Packages Within D Days}
  \item
    \href{https://leetcode.com/problems/divide-chocolate}{LeetCode 1231. Divide Chocolate}
\end{itemize}

以下题目需要结合一定的数学推导写判断条件:

\begin{itemize}
  \label{lc:bs-range-math}
  \item
    \href{https://leetcode.com/problems/nth-magical-number/}{LeetCode 878. Nth
      Magical Number}:容斥原理。
  \item
    \href{https://leetcode.com/problems/ugly-number-iii/}{LeetCode 1201. Ugly
      Number III}:LeetCode 878 升级版,两个数变为三个数。

\end{itemize}

\section{滑动窗口}

滑动窗口有时也叫双指针,一般是用 l 和 r 两个变量记录起始、结束位置,根据条件更新
结果或调整 l 和 r 的值。

两指针异向滑动,一般用于解决回文问题:
\begin{itemize}
  \item
    \href{https://leetcode.com/problems/longest-palindromic-substring/}{LeetCode
    5. Longest Palindromic Substring}
\end{itemize}

两指针同向滑动,典型问题是求满足条件的最长子串:
\begin{itemize}
  \item
    \href{https://leetcode.com/problems/max-consecutive-ones-iii/}{LeetCode 1004. Max Consecutive Ones III}
  \item
    \href{https://leetcode.com/problems/longest-repeating-character-replacement/}{LeetCode
      424. Longest Repeating Character Replacement}:LeetCode 1004 升级版,0/1 变为 26 个大写字母。
  \item
    \href{https://leetcode.com/problems/longest-substring-with-at-most-two-distinct-characters}{LeetCode
      159. Longest Substring with At Most Two Distinct Characters(Premium)}
  \item
    \href{https://leetcode.com/problems/longest-substring-with-at-most-k-distinct-characters}{LeetCode
      340. Longest Substring with At Most K Distinct Characters(Premium)}:LeetCode 159
    升级版,2 个不同字母变为 k 个不同字母。
  \item
    \href{https://leetcode.com/problems/subarrays-with-k-different-integers}{LeetCode
    992. Subarrays with K Different Integers}:要求的是正好 k 个,可以转化为最多
  k 个 - 最多 k-1 个,类似 LeetCode 340。
  \item
    \href{https://leetcode.com/problems/get-equal-substrings-within-budget/}{LeetCode
      1208. Get Equal Substrings Within Budget}:转化为和不超过 k 的最长子数组。
  \item
    \href{https://leetcode.com/problems/minimum-size-subarray-sum/}{LeetCode
      209. Minimum Size Subarray Sum}:和不小于 k 的最短子数组。
\end{itemize}

\section{前缀和}
前缀和(Prefix Sum)可以认为是线段树的简化版,在数组不频繁更新的情况下,可以利用
前缀和在 O(1) 时间,O(n) 空间的条件下获得数组在某一范围内的和。

前缀和 + 哈希相关题目:
\begin{itemize}
  \item
    \href{https://leetcode.com/problems/subarray-sum-equals-k/}{LeetCode 560. Subarray Sum Equals K}
  \item
    \href{https://leetcode.com/problems/count-number-of-nice-subarrays/}{LeetCode
      1248. Count Number of Nice Subarrays}
\end{itemize}

\section{数学}

这部分题目一般都是考察数学功底,需要一定的数学基础和推导计算才能完成。

需要一定推导:
\begin{itemize}
  \item 
    \href{https://leetcode.com/problems/rotate-function/}{LeetCode 396. Rotate
      Function}:前后项相减写递推关系。
\end{itemize}

对数学基础有要求的题目:

\begin{itemize}
  \item
    \href{https://leetcode.com/problems/check-if-it-is-a-good-array/}{LeetCode
      1250. Check If It Is a Good Array}:裴蜀定理。
\end{itemize}

% 待分类
% 排列
% 31. Next Permutation
%
% DFS/BFS
% 1245. Tree Diameter
% 动态规划
% 1239. Maximum Length of a Concatenated String with Unique Characters

%%% Local Variables:
%%% TeX-master: "../master"
%%% End:
