\chapter{GAN 应用于图像生成}

\section{评价指标}

\subsection{Inception Score(IS)}
文献\citerb{2016-IS}中提出用 Inception Score 衡量 GAN 生成图片的质量,其形式为:
\begin{equation}
  \label{equ:IS-score}
  \mathrm{IS} = \mathrm{KL}(p(y|x) || p(y))
\end{equation}

其中 $x$ 为输入图片,$y$ 为网络输出,IS 的具体含义包括:

\begin{itemize}
  \item 使 $p(y|x)$ 的熵尽可能小,即输入图片的类别置信度高。
  \item 使 $p(y)$ 的熵尽可能大,即尽可能生成所有类别的图片。
\end{itemize}

两个分布的散度即两种分布之间的差异,这个差异越大越好,因此 IS 越高,GAN 生成图片
的质量越好。

\subsection{Fréchet Inception Distance(FID)}
文献\citerb{2017-TURR}中提出 FID 指标衡量 GAN 生成图片的质量,其形式为:
\begin{equation}
  \label{equ:FID}
  \mathrm{FID} = \left\Vert m-m_w \right\Vert _{2}^2 + \mathrm{Tr} \left( C + C_w - 2 \left( C C_w \right)^{1/2} \right)
\end{equation}
上式中的所有值均取自 Inception 网络中的 feature,其中 $m, C$ 和 $m_w, C_w$ 分别
为真实数据和生成数据对应的均值和协方差矩阵。

\section{直接生成}
本节中的文献是将隐空间向量 $z$ 和类别等信息作为生成器的输入,直接生成图像。

\subsection{cGAN 系列}
cGAN 系列的生成器 G 都能利用输入 label 信息控制生成图像的类别。

\subsubsection{cGAN}
文献\citerb{2014-cGAN}提出 cGAN(Conditional GAN),通过在输入中加入 label 信息,
可以控制生成器生成的图片类别。cGAN 的生成器 G 和判别器 D 的结构为:

\begin{itemize}
  \item 生成器 G:与原始 GAN 相比,输入多了一个类别 label,先将 label 转换
    为对应的$1 \times n$ embedding,与隐空间向量 $z$ concat 后作为 G 的输入。
  \item 判别器 D:与生成器 G 类似,输入多了一个类别 label,转换为对应的$1 \times
    n$ embedding 后与真实/生成图片 concat 后作为 D 的输入。
\end{itemize}

\subsubsection{ACGAN}
文献\citerb{2016-ACGAN}提出 ACGAN,与 cGAN 相比,改进是在判别器 D 中加入了类别分类器。

\paragraph{ACGAN 的生成器和判别器}
\begin{itemize}
  \item 生成器 G:与 cGAN 类似,只是将 embedding 和 $z$ 之间的运算关系由 concat 改
    为相乘。
  \item 判别器 D:与原始 GAN 相比,多了一个类别输出的 branch,与 cGAN 不
    同,ACGAN 中 D 的输入只有图像,不包括类别 label 的 embedding,类别 label 只
    在计算分类 branch 的 loss 时使用。
\end{itemize}

\paragraph{Title}{ACGAN 的损失函数}
\begin{itemize}
  \item 判别器 D 的损失函数:包括原始 GAN 中图片来源的交叉熵损失函数 $L_{s}$,以及
    图片类别的 Softmax 损失函数 $L_{c}$。判别器 D 试图最小化 $L_{s} + L_{c}$。
  \item 生成器 G 的损失函数:生成器 G 试图最小化 $-L_{s} + L_{c}$,注意 $L_{c}$ 的
    符号为正,即 G 也试图使判别器 D 正确给出生成图片的类别,同时想让 D 将生成图
    片误认为是真实图片。
\end{itemize}

\subsubsection{Projection cGAN}
文献\citerb{2018-Projection-cGAN}提出 Projection cGAN 改进判别器的结构,该结构
与 cGAN 和 ACGAN 的对比如图~\ref{fig:projection-gan}~所示。

\begin{figure}[ht]
  \centering
  \includegraphics[width=0.8\textwidth]{images/GAN/Projection-cGAN.pdf}
  \caption{cGAN、ACGAN 和 Projection GAN 判别器的结构对比}
  \label{fig:projection-gan}
\end{figure}

\begin{enumerate}
  \item (a) 为 cGAN,将隐空间向量 $x$ 和类别 label $y$ concat 后作为判别器的输入。
  \item (b) 与原始 cGAN 类似,只是将类别 label $y$ 的位置由输入挪到网络中间。
  \item (c) 为 ACGAN,类别 label $y$ 只在计算分类 branch loss 时使用,并不作为判
    别器的输入。
  \item (d) 为 Projection cGAN,与 (b) 类似,类别 label $y$ 出现在网络中间层,但
    并非简单 concat,而是与 $\phi(x)$ 做内积。
\end{enumerate}

\subsubsection{SAGAN}
文献\citerb{2018-SAGAN}在 GAN 中加入 self-attention 模块,以改善卷积只能利用局部
信息的问题。

\paragraph{Title}{SAGAN 中的 self-attention 模块}
SAGAN 中的 self-attention 模块的结构如图~\ref{fig:self-attention}~所示。

\begin{figure}[ht]
  \centering
  \includegraphics[width=\textwidth]{images/GAN/Self-attention.png}
  \caption{SAGAN 中的 self-attention 模块}
  \label{fig:self-attention}
\end{figure}

Self-attention 模块将卷积后的 feature map 相乘经过 softmax 得到 attention map,
引入了全局信息,其中 $f(x), g(x), h(x)$ 的通道数都是输入 feature map 通道数的
1/8,模块输出 $y$ 为:
\begin{equation}
  \label{equ:self-attention}
  y = \gamma \cdot o + x
\end{equation}

其中 $\gamma$ 为网络参数,初值为 0,这样在网络训练初期相当于没有 self-attention
模块参与,只利用原始卷积 feature map。

\paragraph{Title}{SAGAN 的损失函数}
SAGAN 采用 hinge loss 形式的损失函数:
\begin{align}
  L_{\mathrm{D}} & = -\mathrm{min} \left( 0, -1 + D(x) \right) - \mathrm{min} \left( 0, -1-D(G(z)) \right) \\
  L_{\mathrm{G}} & = -D(G(z))
\end{align}

\paragraph{SAGAN 的训练技巧}
\begin{itemize}
  \item 谱归一化(Spectral Normalization,SN):原始论文只在判别器 D 中加入了 SN
    模块,SAGAN 在 G 中也加入了 SN 模块,提高模型训练的稳定性。
  \item 生成器和判别器采用不同学习率:借鉴文献\citerb{2017-TURR}中的思路,判别器
    D 的学习率是生成器 G 的四倍,提高了训练速度。
\end{itemize}

\subsubsection{BigGAN}
文献\citerb{2018-BigGAN}提出 BigGAN,可以生成 ImageNet 对应的 1000 类,分辨率从
$128 \times 128$ 到 $512 \times 512$ 的图片。

\paragraph{BigGAN 对 SAGAN 的改进}

\begin{itemize}
  \item 更大的 batch size:batch size 变为原来的 8 倍,从 256 增加到 2048,IS 提高 46\%。
  \item 更宽的网络:channel 数变为原来的 1.5 倍,从 64 增加到 96,IS 提高 21\%。
  \item 共享类别 embedding:BN 层使用 class conditional BN,对不同的 class,生成
    维度为 $g_e$ 的 embedding,将其和 $z$ 的对应部分拼接后,经过 FC 层得到 BN 层
    的参数 $\gamma$ 和 $\beta$。
  \item Skip-z 连接:隐空间变量 $z$ 除了加入输入层,还会加入生成器 G 的中间层。生
    成器 G 中包含 $n$ 个 block 时,将 $z$ 的维度变
    为 $\mathrm{int}({z}_{\mathrm{dim}} / (n+1)) \times (n+1)$ 份,第一份作为输入,其
    他 $n$ 份分别送入对应的 $n$ 个 block。
  \item 截断分布:在 inference 时将 $z$ 进行截断,超出范围的全部取为截断值,这样
    可以提高 IS,但会降低 FID,需要做 trade-off。
  \item 正交正则化:对生成器的卷积层权重加入正交正则化(Orthogonal
    Regularization),原始形式为
    \begin{equation}
      R_{\beta}(W) = \beta \left\Vert W^{\mathrm{T}}W-I\right\Vert _{\mathrm{F}}^2
    \end{equation}
    文中采用的形式为
    \begin{equation}
      R_{\beta}(W) = \beta \left\Vert W^{\mathrm{T}}W \odot (1-I)\right\Vert _{\mathrm{F}}^2
    \end{equation}
    其中 $\odot$ 表示内积。
\end{itemize}

\subsection{ProGAN 系列}
\subsubsection{ProGAN}
文献\citerb{2017-ProGAN}中提出 ProGAN 结构,从 $4 \times 4$ 的低分辨率开始学习 G
和 D,再逐渐提高到原始图像分辨率(例如 $1024 \times 1024$)。

\paragraph{ProGAN 的生成器和判别器}
\begin{itemize}
  \item 主体结构:以 $3 \times 3$ 卷积为主,生成器 G 第一个卷积层和判别器 D 最后一
    个卷积层使用 $4 \times 4$ 的卷积,生成器 G 的最后一个卷积层和判别器 D 的第一
    个卷积层使用 $1 \times 1$ 的卷积。
  \item 激活函数:生成器 G 和判别器 D 都采用 Leaky ReLU,但最后一层都不加激活函
    数,通常 G 的最后一层会使用 tanh 激活函数。
  \item 改变分辨率模块:直接使用线性上采样和 pooling,没有使用转置卷积和 stride
    大于 1 的普通卷积。
  \item 从低分辨率改为高分辨率时进行过渡训练。
\end{itemize}

\paragraph{ProGAN 中避免生成器 G 模式坍缩}
为避免生成器 G 出现模式坍缩,在 D 的最后一层之前加入一层额外的 feature map,计算
最后一层 feature map 所有 batch 所有点的标准差的平均值,再用这个值扩展成 $B
\times 1 \times H \times W$ 的 feature map,与最后一层 feature map concat。这样可
以使生成图像的统计量尽可能接近真实图片,否则很容易被判别器 D 根据统计量特征识别为
生成的图片。

\paragraph{ProGAN 中的归一化}
\begin{itemize}
  \item 各层学习率归一化:采用 He 初始化,每层权重除以 $\sqrt{2/n_{\mathrm{in}}}$,
    其中 $n_{\mathrm{in}}$ 为输入的神经元个数。
  \item 特征向量归一化:每个卷积层后将所有 feature map 同一位置的值进行归一化,
    即 $b_{x, y} = a_{x, y}/\sqrt{\frac{1}{N} \sum_{j=0}^{N-1} \left( a_{x,
          y}^{j} \right) ^ 2 + \epsilon }$
\end{itemize}

\subsubsection{StyleGAN}
文献\citerb{2018-StyleGAN}提出 StyleGAN,在 ProGAN 的基础上进行改进,
图~\ref{fig:stylegan}~所示。

\begin{figure}[ht]
  \centering
  \includegraphics[width=0.8\textwidth]{images/GAN/StyleGAN.pdf}
  \caption{StyleGAN 和 ProGAN 结构对比}
  \label{fig:stylegan}
\end{figure}

\paragraph{StyleGAN 对 ProGAN 的改进}
\begin{itemize}
  \item 隐向量 $z$ 到 $w$ 的映射:利用多层感知机将 $z$ 映射为 $w$,再经过统一的
    仿射变换得到 AdaIN 的变换系数 $(y_s, y_b)$,维度为 $2 \times
    C_{\mathrm{feature\ map}}$。
  \item 高斯噪声:每个卷积层之后加入线性变换后的噪声,每个 feature map 有单独的
    缩放系数,之后再经过 AdaIN。
  \item 生成器 G 使用常数输入:隐向量 $z$ 映射为 $w$ 后作用于 AdaIN,不再加入 G
    的输入层,直接使用常数输入。
\end{itemize}

\paragraph{StyleGAN 中 style 和细节的调整}
\begin{itemize}
  \item Style Mixing:利用 $z_1$ 和 $z_2$ 生成 $w_1$ 和 $w_2$,生成器 G 中前半部
    分和后半部分分别用 $w_1$ 和 $w_2$ 变换后进行 AdaIN。这个技巧也被用于网络正则
    化,在训练时使用。
  \item Style 和细节调整:分别依靠 $w$ 和高斯噪声实现,因为 $w$ 影响整个 feature
    map,噪声只影响 feature map 上的单个点,所以从生成图像结果看,不影响身份外貌
    等整体特征,只影响局部特征。
  \item 不同 scale 噪声的影响:stride 低的层上加噪声会使背景特征更多,stride 高
    的层上加噪声会使前景特征更多,图像更细腻。
\end{itemize}

\subsubsection{StyleGAN2}
文献\citerb{2019-StyleGAN2}提出 StyleGAN2,对 StyleGAN 进行一系列改进。

\begin{itemize}
  \item 去掉 StyleGAN 生成图像中的“水滴”:水滴是 StyleGAN 中的 AdaIN 造成的,。
\end{itemize}

\section{迁移生成}
本节中的文献是将隐空间向量 $z$ 和待转换的图像作为生成器的输入,生成待转换图像迁移
后的图像,一些典型示例包括分割 label 与原始图像的转换,季节/日夜的转换等。

\subsection{pix2pix}
文献\citerb{2016-pix2pix}提出 pix2pix 方法,利用成对图像训练 GAN,是利用 GAN 实
现图像迁移生成的开山之作。

\subsubsection{pix2pix 的生成器 G 和判别器 D}
\begin{itemize}
  \item 生成器 G:生成器 G 试图最小化 $-L_{s} + L_{c}$,注意 $L_{c}$ 的
    符号为正,即 G 也试图使判别器 D 正确给出生成图片的类别,同时想让 D 将生成图
    片误认为是真实图片。
  \item 判别器 D:包括原始 GAN 中图片来源的交叉熵损失函数 $L_{s}$,以及
    图片类别的 Softmax 损失函数 $L_{c}$。判别器 D 试图最小化 $L_{s} + L_{c}$。
\end{itemize}

\subsubsection{pix2pix 的损失函数}
\begin{itemize}
  \item 生成器 G 的损失函数:包括 GAN 的损失函数和 L1 loss 两部分:
    \begin{equation}
      L_{\mathrm{pix2pix\_G}} = \mathrm{log}(1 - D(x, G(x, z))) + \lambda \left\| y - G(x, z) \right\|_1
    \end{equation}
  \item 判别器 D 的损失函数:使用原始 GAN 的损失函数:
    \begin{equation}
      L_{\mathrm{pix2pix\_D}} = -(\mathrm{log}D(x, y) + \mathrm{log}(1 - D(x, G(x, z))))
    \end{equation}
\end{itemize}

\subsection{CycleGAN}

%%% Local Variables:
%%% TeX-master: "../master"
%%% End: