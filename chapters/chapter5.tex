\part{目标检测}
\chapter{基本概念}

\section{指标}

\subsection{交并比}
交并比(Intersection over Union,IoU)

\section{非极大值抑制(NMS)}

非极大值抑制(Non maximum suppression,NMS)是一种后处理方法,其作用是保证一个待
检测物体只有一个检测框与之对应,更直观地理解其实是极大值保留,即只保留(置信度)
是极大值的检测框。图~\ref{fig:nms}~给出了一个具体示例\citerb{2018-NMS}:

\begin{figure}[ht]
  \centering
  \includegraphics[width=0.8\textwidth]{images/目标检测/NMS.png}
  \caption{NMS 示例}
  \label{fig:nms}
\end{figure}

\subsection{传统 NMS}

传统 NMS 算法的具体流程是,先根据分数对所有检测框进行排序,然后从分高到分低遍历所
有检测框,如果高分检测框与低分检测框的 IoU 大于一定阈值,则将低分检测框删掉,后续
遍历时不再处理,如图~\ref{fig:nms-algo}~中的红框。

\begin{figure}[ht]
  \centering
  \includegraphics[width=0.5\textwidth]{images/目标检测/Soft-NMS.pdf}
  \caption{NMS 和 Soft NMS 算法流程}
  \label{fig:nms-algo}
\end{figure}

\subsection{Soft NMS}

传统 NMS 算法需要选择合适的两个检测框的 IoU 阈值,否则阈值太低会导致误检,而阈值
太高又会导致漏检。Soft NMS 的基本思想是根据两个检测框的 IoU,降低低分检测框的分
数,但并不直接将其删除,具体形式如图~\ref{fig:nms-algo}~中的绿框~\citerb{2017-Soft-NMS}。

\subsection{Softer NMS}
文献~\citerb{2018-Softer-NMS}~中在 Soft NMS 的基础上提出了 Softer NMS,具体流程
如图~\ref{fig:softer-nms}~所示:

\begin{figure}[ht]
  \centering
  \includegraphics[width=0.5\textwidth]{images/目标检测/Softer-NMS.pdf}
  \caption{Softer NMS 算法流程}
  \label{fig:softer-nms}
\end{figure}

由图~\ref{fig:softer-nms}~可知,Softer NMS 在 Soft NMS 基础上,加入了 var voting
的部分,其具体步骤是:

\begin{enumerate}
  \item 对任一检测框,计算所有分数低于该框的检测框与其的 IoU。
  \item 根据 IoU 和每个检测框的不确定度 $\sigma$,修正该检测框的位置,其中 IoU
    越大,$ \sigma $ 越小,则相应的权重越高。
\end{enumerate}

\chapter{两阶段方法}
\section{R-CNN 系列}
\label{sec:R-CNN}

\subsection{Fast R-CNN}
\label{subsec:Fast-R-CNN}

Fast R-CNN 在训练和测试时,只需要利用 CNN 将整张图片前传一次,不需要像 R-CNN 中
将所有 proposal 对应的图片区域先 resize 再进行前传,因此可以大大提升训练和测试速
度\citerb{2015-Fast-RCNN}。

\paragraph{网络结构} 

Fast R-CNN 的网络结构如~\ref{fig:Fast-RCNN}~所示,前面是统一的 CNN backbone,对于
每个 proposal,将其投射到 feature map 再经过 RoI pooling 层得到尺寸为 $h \times
w$ 的 RoI,再经过两个全连接层得到 RoI feature,再经过全连接层 + Softmax 层的分类
器得到分类分数,同时 RoI feature 也会经过全连接层得到回归变换值。

\begin{figure}[ht]
  \centering
  \includegraphics[width=0.8\textwidth]{images/目标检测/Fast-RCNN.pdf}
  \caption{Fast R-CNN 网络结构}
  \label{fig:Fast-RCNN}
\end{figure}

\paragraph{RoI Pooling 层}

RoI 即 Region of Interest,中文为感兴趣区域。

\paragraph{Loss 函数}

\paragraph{正负样本}

\paragraph{训练}

\subsection{Faster R-CNN}
\label{subsec:Faster-R-CNN}

\chapter{单阶段方法}

\section{YOLO 系列}
\label{sec:YOLO}

\subsection{YOLO v1}
\label{subsec:YOLOv1}
YOLO v1 是目标检测单阶段方法的开创性工作之一,文中将检测问题刻画为回归问题,没有
任何分类的部分\citerb{2015-YOLO-v1}。

\paragraph{检测方法}

\begin{enumerate}
  \item 将整张图片划分为 $S \times S$ 个网格,每个网格预测 $ B $ 个检测框和 $ C $
  个条件概率 $ \mathrm{Pr}(\mathrm{Class_i}|\mathrm{Object}) $。
  \item 每个检测框共包括 5 个值 $x, y, w, h$ 和 confidence。其中 $x, y$为中心位
  置,$w, h$ 为长和宽(相对图片而言),confidence 为$\mathrm{Pr}(\mathrm{Object})
  \times \mathrm{IoU}^{\mathrm{truth}}_{\mathrm{pred}}$,inference 时直接
  将 confidence乘以该网格对应的条件概率得到最终的分数。网络的输出维度为 $ S
  \times S \times (B \times 5 + C) $。
\end{enumerate}

\paragraph{Backbone}

YOLO v1 的 backbone 没有采用改造后的经典分类网络,而是采用结构为 24 个卷积层 + 2
个全连接层的自定义网络。训练时,先将前 20 个卷积层 + 平均池化层组成的网络
在 ImageNet 数据集上进行预训练,输入尺寸为 $224 \times 224$,在检测时将输入尺寸扩
大为 $448 \times 448$。与此同时,为避免过拟合,第一个全连接层后加了一个 $p=0.5$
的 dropout 层。

\paragraph{Loss 函数}

\begin{align}
  \label{equ:yolo-v1-loss}
  \begin{split}
    L = & \, \lambda_{\mathrm{coord}} \sum_{i=0}^{S^2} \sum_{j=0}^{B} \mathds{1}_{ij}^{\mathrm{obj}} \left [ \left (x_i - \hat{x}_i \right )^2 + \left (y_i - \hat{y}_i \right )^2 \right ] \\
    & \, + \lambda_{\mathrm{coord}} \sum_{i=0}^{S^2} \sum_{j=0}^{B} \mathds{1}_{ij}^{\mathrm{obj}} \left [ \left(\sqrt{w_i} - \sqrt{\hat{w}_i} \right)^2 + \left (\sqrt{h_i} - \sqrt{\hat{h}_i} \right )^2 \right ]  \\
    & \, + \sum_{i=0}^{S^2} \sum_{j=0}^{B} \mathds{1}_{ij}^{\mathrm{obj}} \left( C_i - \hat{C}_i \right)^2  \\
    & \, + \lambda_{\mathrm{noobj}} \sum_{i=0}^{S^2} \sum_{j=0}^{B} \mathds{1}_{ij}^{\mathrm{noobj}} \left( C_i - \hat{C}_i \right)^2  \\
    & \, + \sum_{i=0}^{S^2} \mathds{1}_{i}^{\mathrm{obj}} \sum_{c \in \mathrm{classes}} \left( p_i(c) - \hat{p}_i(c) \right)^2
  \end{split}
\end{align}

\begin{enumerate}
  \item Loss 函数共包括三部分:前两项是第一部分,是检测框位置的 loss;中间两项是
    第二部分,是检测框分数的 loss;最后一项是第三部分,是网格类别概率的 loss。所
    有的 loss 均为 $L_2$ 形式。
  \item 检测框位置 loss 权重 $\lambda_{\mathrm{coord}}$ 提高到 5,同时将不含 GT 的
    网格中检测框分数 loss 的权重 $\lambda_{\mathrm{noobj}}$ 降为 0.5。
  \item 只有与 GT IoU 最大的检测框才会产生检测框位置 loss,只有网格中包含 GT 中
    心才产生网格类别概率 loss。
\end{enumerate}

\subsection{YOLO v2}
\label{subsec:YOLOv2}
YOLO v2 在 YOLO v1 的基础上做了一系列改进\citerb{2016-YOLO-v2}:

\begin{enumerate}
  \item 引入 BN:所有卷积层都加入 BN 层,由于 BN 层有正则化作用,因此可以去掉
    dropout 层。
  \item 改变预训练尺寸:YOLO v1 中预训练时图片输入尺寸为 $224 \times 224$,而检
    测时图片的输入尺寸为 $448 \times 448$,v2 中将预训练时的输入尺寸也改为 $448
    \times 448$,但网络实际的输入尺寸为 $416 \times 416$。
  \item 引入 anchor:参考 Faster R-CNN \cite{2015-Faster-RCNN},YOLO v2 中同样引
    入 anchor 的概念,即网络的输出为 anchor 到 GT 的变换,同时每个 anchor 分别预
    测类别概率和前景分数,而不是一个网格只预测一个类别概率,这样就可以避免 v1 中
    每个网格只能预测一个类别的问题。
  \item Anchor 尺寸聚类:利用 k-means 算法将 GT 聚类,得到 k 个 anchor 尺寸的先
    验。聚类时距离定义为 $d = 1 - \mathrm{IoU}(\mathrm{box}, \mathrm{centroid})$。
  \item 预测相对位置:YOLO v2 输出的检测框的坐标是相对网格的偏差,而非与
    Faster R-CNN 中相同的变换,这样可以将 anchor 的中心限制在该网格内。在网络输
    出后加入 sigmoid 函数即可实现将任意输入压缩至 0 到 1 之间。
  \item 特征融合:将尺寸为 $26 \times 26$ 的 feature map 与尺寸为 $13 \times 13$
    的 feature map 进行融合,具体方法是将 $26 \times 26 \times 512$ 的 feature
    map 变换为 $13 \times 13 \times 2048$,再和 $13 \times 13$ 的 feature map 拼
    接。
  \item 多尺度训练:训练时采用多尺度训练,即图片输入尺寸为 320 到 608 之间,步长
    为 32 的随机数。
  \item 新的 backbone:采用新的 backbone Darknet-19,包含 19 个卷积层和 5 个最大
    池化层。
\end{enumerate}

\subsection{YOLO v3}
\label{subsec:YOLOv3}
YOLO v3 在 YOLO v2 的基础上做了部分改进\citerb{2018-YOLO-v3}:

\begin{enumerate}
  \item Anchor loss 的计算:每一个 GT 分配与其 IoU 最大的 anchor,该 anchor 对应的
    前景分数为 1,如果不是 IoU 最大的 anchor 且与某个 GT 的 IoU 大于 0.5,则
    该 anchor 会被忽略,即不产生前景分数 loss。同时,没有 GT 匹配的 anchor 也不产生
    检测框位置和类别 loss,只计算前景分数 loss。
  \item Loss 类型:将 Softmax loss 改为多个 sigmoid loss,一个 anchor 可以同时对
    应多个类别。
  \item 多尺度特征融合:采用类似 FPN\cite{2016-FPN} 的结构,将深层小尺寸 feature
    map 进行上采样后,与浅层大尺寸 feature map 相加,作为新的 feature map,文
    中共有 3 个尺度。
  \item Anchor 按尺度分配:YOLO v3 中共有 9 种尺寸的 anchor,根据大小分配到不同的
    尺度。
  \item 新的 backbone:采用新的 backbone Darknet-53,包含 53 个卷积层,top-5 精
    度与 ResNet-152 相当,速度快 1 倍。
\end{enumerate}


\section{SSD 系列}
\label{sec:SSD}

\subsection{SSD}
\label{subsec:SSD}

\subsection{DSSD}
\label{subsec:DSSD}

\chapter{Anchor Free 方法}


%%% Local Variables:
%%% TeX-master: "../master"
%%% End:
