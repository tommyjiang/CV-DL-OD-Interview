\part{目标检测}
\chapter{基本概念}

\section{指标}

\subsection{交并比}
交并比(Intersection over Union,IoU)

\section{非极大值抑制}

非极大值抑制(Non maximum suppression,NMS)是一种后处理方法,其作用是保证一个待
检测物体只有一个检测框与之对应,更直观地理解其实是极大值保留,即只保留(置信度)
是极大值的检测框。图~\ref{fig:nms}~给出了一个具体示例\rc{2018-NMS}:

\begin{figure}[ht]
  \centering
  \includegraphics[width=0.8\textwidth]{images/目标检测/NMS.png}
  \caption{NMS 示例}
  \label{fig:nms}
\end{figure}

\subsection{传统 NMS}

传统 NMS 算法的具体流程是,先根据分数对所有检测框进行排序,然后从分高到分低遍历所有检
测框,如果高分检测框与低分检测框的 IoU 大于一定阈值,则将低分检测框删掉,后续遍
历时不再处理,如图~\ref{fig:nms-algo}~中的红框。

\begin{figure}[ht]
  \centering
  \includegraphics[width=0.5\textwidth]{images/目标检测/Soft-NMS.pdf}
  \caption{NMS 和 Soft NMS 算法流程}
  \label{fig:nms-algo}
\end{figure}

\subsection{Soft NMS}

传统 NMS 算法需要选择合适的两个检测框的 IoU 阈值,否则阈值太低会导致误检,而阈值
太高又会导致漏检。Soft NMS 的基本思想是根据两个检测框的 IoU,降低低分检测框的分
数,但并不直接将其删除,具体形式如图~\ref{fig:nms-algo}~中的绿框~\rc{2017-Soft-NMS}。

\subsection{Softer NMS}
文献~\rc{2018-Softer-NMS}~中在 Soft NMS 的基础上提出了 Softer NMS,具体流程
如图~\ref{fig:softer-nms}~所示:

\begin{figure}[ht]
  \centering
  \includegraphics[width=0.5\textwidth]{images/目标检测/Softer-NMS.pdf}
  \caption{Softer NMS 算法流程}
  \label{fig:softer-nms}
\end{figure}

由图~\ref{fig:softer-nms}~可知,Softer NMS 在 Soft NMS 基础上,加入了 var voting
的部分,其具体步骤是:

\begin{enumerate}
  \item 对任一检测框,计算所有分数低于该框的检测框与其的 IoU。
  \item 根据 IoU 和每个检测框的不确定度 $\sigma$,修正该检测框的位置,其中 IoU
    越大,$ \sigma $ 越小,则相应的权重越高。
\end{enumerate}

\chapter{两阶段方法}

\chapter{单阶段方法}

\chapter{Anchor Free 方法}


%%% Local Variables:
%%% TeX-master: "../master"
%%% End:
