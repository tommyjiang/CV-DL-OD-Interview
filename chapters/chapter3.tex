\part{机器学习和深度学习基础}

\chapter{基本概念}

\section{激活函数}

激活函数是神经网络中最典型的非线性层。如果没有激活函数,无论神经网络有多少层,输
出总为输入的线性函数,即可以将多层等效为单个线性层,将无法解决异或这样简单的非线
性问题。

\subsection{Sigmoid 函数}\label{subsec:Sigmoid}

Sigmoid 函数及其导数的形式为:
\begin{align}
  \label{equ:sigmoid}
  \sigma(x) & = \frac{1}{1 + e^{-x}} \\
  \label{equ:sigmoid-d}
  \sigma'(x) & = \sigma(x) (1-\sigma(x))
\end{align}

\begin{figure}[ht]
  \centering
  \includegraphics[width=0.8\textwidth]{images/机器学习和深度学习基础/基本概念/Sigmoid.png}
  \caption{Sigmoid 函数及其导数的图像}\label{fig:sigmoid}
\end{figure}

\begin{itemize}
  \item 图~\ref{fig:sigmoid}~中蓝色曲线为 sigmoid 函数,其形状与字母 S 类似,可以
    将 $-\infty$ 到 $\infty$ 的输入单调映射到 0 到 1 之间。
  \item 图~\ref{fig:sigmoid}~中红色曲线为 sigmoid 函数的导数,在输入值较大或较小时,
    该值趋于 0,因此 sigmoid 函数作为激活函数时,可能出现梯度消失,网络无法正常训
    练。
\end{itemize}

\subsection{线性整流函数(ReLU)}

线性整流函数(Rectifier Linear Unit,ReLU)是目前应用最广泛的激活函数,其形式为:
\begin{equation}
  \label{equ:ReLU}
  \mathrm{ReLU}(x) = \left\{
    \begin{array}{lr}
      x & x > 0 \\
      0 & x \leq 0 \\
    \end{array}
  \right.
\end{equation}

ReLU 的图像如图~\ref{fig:ReLU}~所示。当 $x > 0$ 时,梯度始终为 1;当 $x \leq 0$
时梯度始终为 0,同样会导致梯度消失。

\begin{figure}[ht]
  \centering
  \includegraphics[width=0.5\textwidth]{images/机器学习和深度学习基础/基本概念/ReLU.png}
  \caption{ReLU 的图像}\label{fig:ReLU}
\end{figure}

\section{损失函数}

\subsection{交叉熵损失函数}\label{subsec:CELoss}

交叉熵损失函数(Cross Entropy Loss,CE Loss)的形式为:
\begin{equation}
  \label{equ:CELoss}
  \mathrm{L}_{\mathrm{CE}} = - \left( y \, \log \, \hat{y} + (1-y) \log (1-\hat{y}) \right )
\end{equation}

其中 $y$ 为真实 label,正样本为 1,负样本为 0,$\hat{y}$ 为预测值,介于 0 和 1
之间,预测值越接近真实 label,则 loss 越小。

\subsection{Softmax 损失函数}

\subsection{$L_1$ 损失函数}

\subsection{$L_2$ 损失函数}

\subsection{Smooth $L_1$ 损失函数}
Smooth $L_1$ 损失函数是 $L_1$ 和 $L_2$ 损失函数的组合,其形式为:
\begin{equation}
  \label{equ:SmoothL1}
  \mathrm{SL}_1(x) = \left\{
    \begin{array}{lr}
      |x| & |x| > \alpha \\
      \frac{1}{\alpha} x^2 & |x| \leq \alpha \\
    \end{array}
  \right.
\end{equation}

由上式可知,Smooth $L_1$ 损失函数在 $x$ 绝对值较小时为 $L_2$ 函数,而在 $x$ 绝对
值较大时为 $L_1$ 函数。在目标检测的两阶段方法中,回归 loss 一般使用 Smooth $L_1$
损失函数。

\subsection{Focal loss}

文献\citerb{2017-RetinaNet}提出 focal loss,用于解决目标检测中正负样本不平衡的问
题。Focal loss 降低了简单样本的权重,与难样本挖掘(Hard Negative Mining,HEM)类
似。Focal loss 的形式和图像如公式~\ref{equ:focal-loss}~和图~\ref{fig:focal-loss}~所
示:
\begin{equation}
  \label{equ:focal-loss}
  \mathrm{FL}(p_{\mathrm{t}}) = - \alpha_t {(1-p_{\mathrm{t}})}^{\gamma} \log (p_{\mathrm{t}})
\end{equation}

\begin{figure}[ht]
  \centering
  \includegraphics[width=0.8\textwidth]{images/机器学习和深度学习基础/基本概念/Focal-Loss.pdf}
  \caption{Focal loss 的图像($\alpha_t = 1$}\label{fig:focal-loss}
\end{figure}

图~\ref{fig:focal-loss}~中,$\gamma = 0$ 的蓝色曲线即传统的交叉熵 loss。其他颜色
的曲线分别表示 $\gamma$ 取不同数值时的 focal loss。Focal loss 中两个参数的作用
是:

\begin{itemize}
  \item $\alpha_t$:平衡正负样本的权重,一般正负样本对应的 $\alpha_t$ 之和为 1。
  \item $\gamma$:平衡难易样本的权重,$\gamma$ 越大,简单样本的权重越小。例
    如 $\gamma = 2$ 时,对于预测值 $p = 0.9$ 的样本,focal loss 的值要比交叉
    熵 loss 的值小 100 倍。
\end{itemize}

\section{优化方法}\label{sec:opt}

\subsection{梯度下降(GD)和随机梯度下降(SGD)}
梯度下降方法属于一阶优化算法,即利用函数的一阶导数。由于梯度是函数值上升最快的方
向,因此在最小化 loss 时,选择负梯度方向更新参数:
\begin{equation}
  \boldsymbol{\theta}_t = \boldsymbol{\theta}_{t-1} - \eta \cdot \nabla_{\boldsymbol{\theta}}J(\boldsymbol{\theta})
\end{equation}

根据上式中梯度计算时使用样本数的不同,梯度下降法又可分为:
\begin{itemize}
  \item 批梯度下降:使用全部样本计算梯度。
  \item 随机梯度下降:使用单个样本计算梯度。
  \item Mini-batch 梯度下降:使用部分样本(一般为 2 的整数次幂,例如 16、32、64
    等)计算梯度。目前应用最为广泛,有时也称为随机梯度下降,注意与单个样本的随机
    梯度下降做区分。
\end{itemize}

\subsection{带动量的随机梯度下降}
带动量的随机梯度下降可以使迭代过程中满足梯度方向相同时加速,梯度改变方向时减速:
\begin{align}
  v_t & = \gamma v_{t-1} + \eta \cdot \nabla_{\boldsymbol{\theta}}J(\boldsymbol{\theta}) \\
  \boldsymbol{\theta}_{t} & = \boldsymbol{\theta}_{t-1} - v_t
\end{align}

\subsection{Nesterov 方法}
Nesterov 方法是对带动量的随机梯度下降的改进,计算梯度时不用 $\theta$ 而用
$\theta - \gamma_{t-1}$:
\begin{align}
  v_t & = \gamma v_{t-1} + \eta \cdot \nabla_{\boldsymbol{\theta}}J(\boldsymbol{\theta}-\gamma v_{t-1}) \\
  \boldsymbol{\theta}_{t} & = \boldsymbol{\theta}_{t-1} - v_t
\end{align}

\subsection{Adagrad 方法}
Adagrad 是对随机梯度下降的改进,将参数对应梯度除以历史梯度累积和的开方:
\begin{equation}
  \boldsymbol{\theta}_{t} = \boldsymbol{\theta}_{t-1} - \frac{\eta \cdot \nabla_{\boldsymbol{\theta}}J(\boldsymbol{\theta})}{G_t + \epsilon}
\end{equation}

\subsection{Adadelta 方法}
Adagrad 方法中,分母上的历史梯度累计和会越来越大,导致当前更新值很小。Adadelta
方法将历史梯度值的计算改为类似 Nesterov 的形式:
\begin{align}
  g_t & = \nabla_{\boldsymbol{\theta}}J(\boldsymbol{\theta_t}) \\
  {E[g^2]}_t & = \gamma {E[g^2]} {t-1} + (1-\gamma)g_t^2 \\
  \boldsymbol{\theta}_{t} & = \boldsymbol{\theta}_{t-1} - \frac{\eta \cdot g_t}{{E[g^2]}_t+\epsilon}
\end{align}

RMSProp 方法和 Adadelta 方法类似。

\subsection{Adam 方法}
Adam 方法在 Adagrad 的基础上,同时计算梯度的动量:
\begin{align}
  m_t & = \beta_1 m_{t-1} + (1-\beta_1) g_t \\
  v_t & = \beta_2 v_{t-1} + (1-\beta_2) g_t^2 \\
  \hat{m}_t & = \frac{m_t}{1-\beta_1^t} \\
  \hat{v}_t & = \frac{v_t}{1-\beta_2^t} \\
  \boldsymbol{\theta}_{t} & = \boldsymbol{\theta_{t-1}} - \frac{\eta}{\sqrt{\hat{v}_t + \epsilon}}\hat{m}_t
\end{align}

\section{反向传播算法}
神经网络的训练是通过不断调整网络参数从而减小损失函数的过程。对网络参数的优化一般
采用基于梯度的方法,因此需要先计算损失函数 $L$ 对神经网络中各参数 $\omega$ 的导数。
反向传播算法利用链式法则,由后向前逐层计算 $L$ 对各层参数 $W$ 的导数,是目前应用
最多的导数计算方法。

考虑一个神经网络中的某个全连接层 $Y = XW + b$,设网络的 loss 为 $L$,根据链式法
则,有如下关系~\citerb{2017-FC-BP}:

\begin{align}
  \label{equ:bp-fc}
  \frac{\partial L}{\partial X} & = \frac{\partial L}{\partial Y} \frac{\partial Y}{\partial X} = \frac{\partial L}{\partial Y} W^{\mathrm{T}} \\
  \frac{\partial L}{\partial W} & = \frac{\partial L}{\partial Y} \frac{\partial Y}{\partial W} = X^{\mathrm{T}} \frac{\partial L}{\partial Y}\\
  \frac{\partial L}{\partial b} & = \frac{\partial L}{\partial Y} \frac{\partial Y}{\partial b} = \frac{\partial L}{\partial Y}
\end{align}

上面三个式子中,$\frac{\partial L}{\partial X}$ 是为了继续进行反向传播计算 loss
对上一层参数的导数;$\frac{\partial L}{\partial W}$ 和 $\frac{\partial
  L}{\partial b}$ 即更新参数时需要的梯度。类似以上过程,从最后的 loss 开始逐层向
后计算 loss 对每一层输入 $X$ 和参数 $W, \, b$ 的梯度。

\subsection{梯度消失和梯度爆炸}\label{subsec:gradient-vanish-explosion}
根据链式法则,对于深层神经网络,浅层参数的梯度与之后所有层输出对输入梯度的乘积
相关。直观理解,如果梯度都小于 1,相乘后的值很小,会导致梯度消失;反之如果梯度都大
于 1,相乘后的值很大,会导致梯度爆炸。

梯度消失的一个典型原因是激活函数使用了 sigmoid 函数。梯度爆炸的一个简单的解决方法
是梯度裁剪(Gradient Clipping),即设定一个阈值,当梯度值大于阈值时,使其等于阈值,
这样即可避免梯度过大。

\section{归一化}

\subsection{批归一化(BN)}\label{sub:BN}

批归一化(Batch Normalization,BN)是文献\citerb{2015-BN}中提出的一种归一化方法,
BN 的作用包括:

\begin{enumerate}
  \item 降低深层神经网络的训练难度。加入 BN 后,训练时可以使用更大的学习率。
  \item 防止过拟合。BN 一定程度上相当于引入了正则项,加入 BN 后可以去掉 dropout
    层,\hyperref[subsec:YOLOv2]{YOLOv2} 就是一个典型的例子。
\end{enumerate}

文献\citerb{2015-BN}中认为 BN 有效的原因是其减少了神经网络的内部协同变化
(Internal Covariate Shift,ICS),但文献\citerb{2018-BN-help-opt}论证 BN 并不能
减少 ICS,其提升性能的原因是使网络的 loss 变得更平滑。

\paragraph{BN 的形式}
BN 包括归一化和线性变换两部分。其中归一化是为了减少上层网络参数变化对输入的影响;
线性变换部分是为了增加 BN 层的拟合能力,如果没有线性变换部分,BN 层的输出将始终为
均值为 0,方差为 1 的分布。BN 的具体形式为:
\begin{align}
  \label{equ:BN}
  \mu_{\mathcal{B}} & = \frac{1}{m} \sum_{i=1}^{m} x_i \\
  \sigma_{\mathcal{B}}^2 & = \frac{1}{m} \sum_{i=1}^{m} {(x_i-\mu_{\mathcal{B}})}^2 \\
  \hat{x}_i & = \frac{x_i - \mu_{\mathcal{B}}}{\sqrt{\sigma_{\mathcal{B}}^2 + \epsilon}} \\
  y_i & = \gamma \hat{x}_i + \beta
\end{align}

\begin{itemize}
\item 上式中 $m$ 为一个 batch 的样本数,即期望和方差计算的都是一个 batch 的统计量。
  训练时采用滑动平均法计算均值和方差,即同时考虑历史 batch 和当前 batch 的统计量,
  例如 Caffe 中的~\href{https://bit.ly/2JBY7aw}{BN层},测试时\textbf{使用训练时保
    存的均值和方差},不需要都重新计算每个 batch 的统计量。
  \item BN 层中有两个需要学习的参数,即线性变换的参数 $\gamma$ 和 $\beta$,这两
    个参数的初值分别为 1 和 0。
\end{itemize}

\paragraph{CNN 中的 BN}
\begin{itemize}
  \item 卷积层之后 BN 均值和方差的计算方法:计算的是一个 batch 中 $m$ 个样本
    同一个 feature map 上所有点的均值和方差。如果一个 batch 对应的卷积层的输出维度
    为 $m \times C \times H \times W$,则只计算通道数 $C$ 个均值和方差,不同
    feature map 利用与其对应的均值和方差进行归一化。
  \item BN 的位置:文献~\citerb{2015-BN}~中给出的结构是\textbf{卷积层-BN-ReLU},
    即 BN 层加在卷积层之后,ReLU 之前。
  \item 卷积层的偏置 $b$:由于 BN 中包括减均值,因此可以省略卷积层的偏置。
  \item 测试时将卷积层和 BN 层进行合并:由于卷积层和 BN 层都是线性变换,且测试
    时 BN 层使用的均值和方差均为训练时保存的结果,因此可以将二者合并以提升计算速
    度,合并后参数的计算方法可以参考文献\citerb{2017-DSSD}。
\end{itemize}

\subsection{组归一化(GN)}\label{sub:GN}

文献\citerb{2018-GN}中提出了组归一化(Group Normalization,GN)方法,主要解决 BN
在 batch 较小时性能不佳的问题。

\paragraph{GN 的形式}
不同归一化方法的对比如图~\ref{fig:all-norm}~所示:
\begin{figure}[ht]
  \centering
  \includegraphics[width=0.8\textwidth]{images/机器学习和深度学习基础/基本概念/GN.pdf}
  \caption{不同归一化方法对比}\label{fig:all-norm}
\end{figure}

\begin{itemize}
  \item BN:对一个 batch 内不同样本的同一个 feature map 做归一化,共有通道
    数 $C$ 个均值和方差。
  \item LN:同一样本,对所有通道的 feature map 做归一化,共有 batch 样本数 $N$
    个均值和方差。
  \item GN:同一样本,对按组划分通道的 feature map 做归一化,共有 batch 样本数乘
    以组数 $N \times G$ 个均值和方差。一般 $G$ 取 32,LN 可看做 GN 中 $G=1$ 的特
    例。
\end{itemize}

\chapter{卷积神经网络}

\section{卷积类型}

\subsection{分组卷积}

分组卷积(Group Convolution,GC)在 AlexNet 中就已经应用,但当时的原因是 GPU 计
算能力不足,只能将标准卷积拆成 2 个分组卷积~\citerb{2012-AlexNet}。分组卷积和标准
卷积的对比如图~\ref{fig:normal-conv} 和图~\ref{fig:group-conv} 所
示~\citerb{2019-Diff-Conv}:

\begin{figure}[ht]
  \centering
  \includegraphics[width=0.8\textwidth]{images/机器学习和深度学习基础/卷积神经网络/标准卷积.png}
  \caption{标准卷积}\label{fig:normal-conv}
\end{figure}

\begin{figure}[ht]
  \centering
  \includegraphics[width=0.8\textwidth]{images/机器学习和深度学习基础/卷积神经网络/分组卷积.png}
  \caption{分组卷积}\label{fig:group-conv}
\end{figure}

分组卷积的具体步骤是:

\begin{enumerate}
  \item 将输入按通道拆分成 $g$ 组,每组数据的尺寸与输入相同,均为 $H_{\mathrm{in}}
    \times W_{\mathrm{in}}$,通道数均为输入的 $ 1/g $,即 $
    C_{\mathrm{g}} = C_{\mathrm{in}} / g $。
  \item 分别对拆分后的 $g$ 组数据进行标准卷积,每一组数据对应的输出的通道数
    均为 $ C_{\mathrm{out}}/g $。
  \item 将 $g$ 组数据的输出结果进行拼接,得到通道数为 $C_{\mathrm{out}}$
    的输出。
\end{enumerate}

分组卷积可以节省计算量,因此在轻量型网络中得到广泛应用。参数相同的情况下,分组卷
积的计算量约为标准卷积的 $ 1 / g^2 $,组数越多,计算量越小。

分层卷积(Depthwise Convolution,DW)是分组卷积的一种特例,是指组数 $g$ 与输入数
据通道数 $C_{\mathrm{in}}$ 相同的卷积,即每组卷积的输入只有 1 个通道。

\subsection{深度可分离卷积}

深度可分离卷积(Depthwise Separable Convolution,DSC)如图~\ref{fig:ds-conv}~所
示~\citerb{2019-Diff-Conv}:

\begin{figure}[ht]
  \centering
  \includegraphics[width=0.8\textwidth]{images/机器学习和深度学习基础/卷积神经网络/深度可分离卷积.png}
  \caption{深度可分离卷积}\label{fig:ds-conv}
\end{figure}

深度可分离卷积的步骤包括分层卷积和逐点卷积两部分,计算量为标准卷积计算量的 $ 1 /
C_{\mathrm{in}} + 1/k^2 $,$k$ 为分层卷积的卷积核大小。
当通道数 $C_{\mathrm{in}}$ 较大时,上式可近似为 $1 / k^2$~\citerb{2017-MobileNet-v1}。

\subsection{反卷积}\label{subsec:deconv}
反卷积(Deconvolution),有时也称转置卷积(Transposed Convolution),是一种特殊的卷积
形式。反卷积的具体步骤是,先通过补 0 扩大输入图像的尺寸,再进行标准卷积。反卷积的
作用是将卷积后尺寸较小的 feature map 恢复到与输入大小相同的尺寸。 需要特别说明的
是,反卷积虽然可以恢复尺寸,\textbf{但不能保证恢复后的数值与输入相同}~\citerb{2016-Guide-Conv}。

\section{感受野}

感受野(Receptive Field,RF) 是指卷积神经网络中某一层的一个像素,会受到多少个输
入数据像素的影响。直观上理解,就是该像素能看到多大范围的输入层像
素~\citerb{2017-Guide-to-RF-cal}。具体的计算方法为:
\begin{align}
\label{equ:rf-cal}
j_{\mathrm {out}} & = j_{\mathrm{in}} \times s \\
r_{\mathrm {out}} & = r_{\mathrm{in}} + (k-1) \times j_{\mathrm{in}}
\end{align}

其中 $j$ 为 jump,即每一层的 stride;s 为该层运算的 stride,k 为卷积的 kernel
size,r 即感受野。利用公式~\eqref{equ:rf-cal}~即可计算各层的感受野。经典网络 VGG-16
各层感受野的详细计算可以参考文献~\citerb{2018-VGG-16-RF-cal}。

%%% Local Variables:
%%% TeX-master: "../master"
%%% End:
