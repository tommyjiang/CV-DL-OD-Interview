\chapter{生成对抗网络(GAN)}

\section{GAN}

文献\citerb{2014-GAN}中第一次提出了生成对抗网络,同时训练生成器和判别器完成图像生
成任务。

\subsection{GAN 的生成器和判别器}

\begin{itemize}
  \item 生成器 G:输入为隐空间向量 $z$,输出为与真实数据维度相同的生成数据 $G(z)$。对于
    图像生成任务,生成器 G 输出 3 通道 RGB 维度为 $3 \times H \times W$ 的图片。
  \item 判别器 D:输入为真实数据 $x_{\mathrm{real}}$ 或生成器 G 的输出
    $x_{\mathrm{fake}} = G(z)$,输出为输入数据是真实数据的概率 $D(x)$。
\end{itemize}

\subsection{GAN 的损失函数}
\begin{itemize}
  \item 判别器 D 的损失函数为\hyperref[subsec:CELoss]{交叉熵损失函数}:
  \begin{equation}
    \label{equ:GAN-D}
    \mathrm{L}_{\mathrm{GAN-D}} = - \left( \mathrm{log} D(x_i) + \mathrm{log} (1-D(G(z))) \right )
  \end{equation}

  \item 生成器 G 的损失函数为判别器 D 损失函数的后一项取相反数,即:
  \begin{equation}
    \label{equ:GAN-G}
    \mathrm{L}_{\mathrm{GAN-G}} =  \mathrm{log} (1-D(G(z)))
  \end{equation}

\end{itemize}

\subsection{GAN 的训练}

GAN 采用判别器和生成器交替训练的方式。

\begin{itemize}
  \item 判别器的训练:输入为 $x$ 和 $G(z)$ 组成的 batch,输出为 $D(x)$ 和
    $D(G(z))$,计算 loss 反传更新 D 的参数。
  \item 生成器的训练:输入为 $z$,输出为 $D(G(z))$,计算 loss 反传更新 G 的参数。
\end{itemize}

\section{CGAN}
文献\citerb{2014-CGAN}提出 CGAN(Conditional GAN),通过加入额外的 label 信息,
可以控制 GAN 中生成器生成的图片类别。

\subsection{CGAN 的生成器和判别器}
\begin{itemize}
  \item 生成器 G:与原始 GAN 相比,输入多了一个类别 label(0 到 $n-1$ 之间),先转换
    为对应的$1 \times n$ embedding,与隐空间向量 $z$ concat 后作为 G 的输入。
  \item 判别器 D:与生成器 G 类似,输入同样多了一个类别 label(0 到 $n-1$ 之间),
    转换为 embedding 后与真实/生成图片 concat 后作为 D 的输入。
\end{itemize}

\section{ACGAN}
文献\citerb{2016-ACGAN}提出 ACGAN,与 CGAN 相比,在判别器 D 结构中加入类别分类器
和分类 loss,提升生成图像的质量。

\subsection{ACGAN 的生成器和判别器}
\begin{itemize}
  \item 生成器 G:与原始 GAN 相比,输入多了一个类别 label(0 到 $n-1$ 之间),先转换
    为对应的$1 \times \mathrm{len}(z)$ 的 embedding,与隐空间向量 $z$ 相乘后作为 G
    的输入。
  \item 判别器 D:与原始 GAN 相比,多了一个类别输出的 branch,注意与 CGAN 不同,
    ACGAN 中 D 的输入只有图像,不包括类别。
\end{itemize}

\subsection{ACGAN 的损失函数}
\begin{itemize}
  \item 判别器 D 的损失函数:包括原始 GAN 中图片来源的交叉熵损失函数 $L_{s}$,以及
    图片类别的 Softmax 损失函数 $L_{c}$。判别器 D 试图最小化 $L_{s} + L_{c}$。
  \item 生成器 G 的损失函数:生成器 G 试图最大化 $L_{s} - L_{c}$,注意 $L_{c}$的
    符号为负,即 G 也试图使判别器 D 正确给出生成图片的类别,但错误地将生成图片认
    为是真实图片。
\end{itemize}


%%% Local Variables:
%%% TeX-master: "../master"
%%% End:
