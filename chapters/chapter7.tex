\part{生成对抗网络(GAN)}

\chapter{GAN 基础}
\section{GAN}

文献\citerb{2014-GAN}中第一次提出了生成对抗网络(Generative Adversarial
Network,GAN),通过交替训练生成器 G 和判别器 D,生成器 G 可以将输入的噪声 $z$ 转
化为类似训练样本的图片。

\subsection{GAN 的生成器 G 和判别器 D}

\begin{itemize}
  \item 生成器 G:输入为随机的隐空间向量 $z$,输出为与真实数据维度相同的数
    据 $G(z)$。对于图像生成任务,如果输入为 3 通道 RGB 图片,那么生成器 G 同样输
    出 3 通道 RGB 图片(维度为 $3 \times H \times W$)。
  \item 判别器 D:输入为真实数据 $x_{\mathrm{real}}$ 或生成器 G 的输出
    $x_{\mathrm{fake}} = G(z)$,输出为输入是真实数据的概率 $D(x)$。
\end{itemize}

\subsection{GAN 的损失函数}

\begin{itemize}
  \item 判别器 D 的损失函数为\hyperref[subsec:CELoss]{交叉熵损失函数}:
  \begin{equation}
    \label{equ:GAN-D}
    \mathrm{L}_{\mathrm{GAN-D}} = - \left( \mathrm{log} D(x_i) + \mathrm{log} (1-D(G(z))) \right )
  \end{equation}

  \item 生成器 G 的损失函数为判别器 D 损失函数的后一项取相反数,即:
  \begin{equation}
    \label{equ:GAN-G}
    \mathrm{L}_{\mathrm{GAN-G}} =  \mathrm{log} (1-D(G(z)))
  \end{equation}
\end{itemize}

\subsection{GAN 的训练}

GAN 的训练过程为交替训练判别器 D 和生成器 G。

\begin{itemize}
  \item 判别器 D 的训练:输入为 $x$ 和 $G(z)$ 组成的 batch,输出为 $D(x)$ 和
    $D(G(z))$,计算 loss 反传更新 D 的参数。
  \item 生成器 G 的训练:输入为 $z$,输出为 $D(G(z))$,计算 loss 反传更新 G 的参
    数。
\end{itemize}

理想情况下,判别器 D 和生成器 G 收敛到纳什均衡解,即生成器 G 完美还原原始数据,判
别器 D 对任何输入始终输出 0.5。

\section{GAN 的训练技巧}
\subsection{Spectral Normalization}
文献\citerb{2018-SN}提出 Spectral Normalization,将其加入 GAN 的判别器可以提
升 GAN 训练的稳定性。

假设神经网络中的卷积或全连接层为 $y = Wx$,其中 $W$ 为权重矩阵,SN 方法是将权重
$W$ 归一化为 $W/\sigma(W)$,其中 $\sigma(W)$ 为 $W$ 的谱范数,即:
\begin{equation}
  \sigma(W) = \sqrt{\mathrm{max} \, \left | \mathrm{eig} (W^{\mathrm{T}} W) \right |}
\end{equation}

SN 在实际应用时,采用迭代求解的方式以加快计算速度,具体算法如
图~\ref{fig:sn-algo}~所示。

\begin{figure}[ht]
  \centering
  \includegraphics[width=\textwidth]{images/GAN/SN.pdf}
  \caption{Spectral Normalization 迭代算法流程}
  \label{fig:sn-algo}
\end{figure}

%%% Local Variables:
%%% TeX-master: "../master"
%%% End:
