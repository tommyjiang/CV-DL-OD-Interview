\chapter{生成对抗网络(GAN)}

\section{GAN}

文献\citerb{2014-GAN}中第一次提出了生成对抗网络,同时训练生成器和判别器完成图像生
成任务。

\subsection{生成器和判别器}

\begin{itemize}
  \item 生成器 G:输入为隐空间向量 $z$,输出为与真实数据维度相同的生成数据 $G(z)$。对于
    图像生成任务,生成器 G 输出 3 通道 RGB 维度为 $3 \times H \times W$ 的图片。
  \item 判别器 D:输入为真实数据 $x_{\mathrm{real}}$ 或生成器 G 的输出
    $x_{\mathrm{fake}} = G(z)$,输出为输入数据是真实数据的概率 $D(x)$。
\end{itemize}

\subsection{GAN 的损失函数}
\begin{itemize}
  \item 判别器 D 的损失函数为\hyperref[subsec:CELoss]{交叉熵损失函数}:
  \begin{equation}
    \label{equ:GAN-D}
    \mathrm{L}_{\mathrm{GAN-D}} = - \left( \mathrm{log} D(x_i) + \mathrm{log} (1-D(G(z))) \right )
  \end{equation}

  \item 生成器 G 的损失函数为判别器 D 损失函数的后一项取相反数,即:
  \begin{equation}
    \label{equ:GAN-G}
    \mathrm{L}_{\mathrm{GAN-G}} =  \mathrm{log} (1-D(G(z)))
  \end{equation}

\end{itemize}

\subsection{GAN 的训练}

GAN 采用判别器和生成器交替训练的方式。

\begin{itemize}
  \item 判别器的训练:输入为 $x$ 和 $G(z)$ 组成的 batch,输出为 loss,进行反向传播。
    此时 $G(z)$ 为定值,因此训练只更新 D 的参数,不会更新 G 的参数。
  \item 生成器的训练:输入为 $z$,输出为 $D(G(z))$,会同时更新 G 和 D 的参数。
\end{itemize}

\section{CGAN}

\section{ACGAN}

%%% Local Variables:
%%% TeX-master: "../master"
%%% End:
