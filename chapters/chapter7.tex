\chapter{生成对抗网络(GAN)}

\section{GAN}

文献\citerb{2014-GAN}中第一次提出了生成对抗网络(Generative Adversarial
Network,GAN),通过交替训练生成器 G 和判别器 D,生成器 G 可以将输入的噪声 $z$ 转
化为类似训练样本的图片。

\subsection{GAN 的生成器 G 和判别器 D}

\begin{itemize}
  \item 生成器 G:输入为随机的隐空间向量 $z$,输出为与真实数据维度相同的数
    据 $G(z)$。对于图像生成任务,如果输入为 3 通道 RGB 图片,那么生成器 G 同样输
    出 3 通道 RGB 图片(维度为 $3 \times H \times W$)。
  \item 判别器 D:输入为真实数据 $x_{\mathrm{real}}$ 或生成器 G 的输出
    $x_{\mathrm{fake}} = G(z)$,输出为输入是真实数据的概率 $D(x)$。
\end{itemize}

\subsection{GAN 的损失函数}

\begin{itemize}
  \item 判别器 D 的损失函数为\hyperref[subsec:CELoss]{交叉熵损失函数}:
  \begin{equation}
    \label{equ:GAN-D}
    \mathrm{L}_{\mathrm{GAN-D}} = - \left( \mathrm{log} D(x_i) + \mathrm{log} (1-D(G(z))) \right )
  \end{equation}

  \item 生成器 G 的损失函数为判别器 D 损失函数的后一项取相反数,即:
  \begin{equation}
    \label{equ:GAN-G}
    \mathrm{L}_{\mathrm{GAN-G}} =  \mathrm{log} (1-D(G(z)))
  \end{equation}
\end{itemize}

\subsection{GAN 的训练}

GAN 的训练过程为交替训练判别器 D 和生成器 G。

\begin{itemize}
  \item 判别器 D 的训练:输入为 $x$ 和 $G(z)$ 组成的 batch,输出为 $D(x)$ 和
    $D(G(z))$,计算 loss 反传更新 D 的参数。
  \item 生成器 G 的训练:输入为 $z$,输出为 $D(G(z))$,计算 loss 反传更新 G 的参
    数。
\end{itemize}

理想情况下,判别器 D 和生成器 G 收敛到纳什均衡解,即生成器 G 完美还原原始数据,判
别器 D 对任何输入始终输出 0.5。

\section{CGAN}
文献\citerb{2014-CGAN}提出 CGAN(Conditional GAN),通过在输入中加入 label 信息,
可以控制生成器生成的图片类别。

\subsection{CGAN 的生成器和判别器}
\begin{itemize}
  \item 生成器 G:与原始 GAN 相比,输入多了一个类别 label,先将 label 转换
    为对应的$1 \times n$ embedding,与隐空间向量 $z$ concat 后作为 G 的输入。
  \item 判别器 D:与生成器 G 类似,输入多了一个类别 label,转换为对应的$1 \times
    n$ embedding 后与真实/生成图片 concat 后作为 D 的输入。
\end{itemize}

\section{ACGAN}
文献\citerb{2016-ACGAN}提出 ACGAN,与 CGAN 相比,改进是在判别器 D 中加入了类别分类器。

\subsection{ACGAN 的生成器和判别器}
\begin{itemize}
  \item 生成器 G:与 CGAN 类似,只是将 embedding 和 $z$ 之间的运算关系由 concat 改
    为相乘。
  \item 判别器 D:与原始 GAN 相比,多了一个类别输出的 branch,与 CGAN 不
    同,ACGAN 中 D 的输入只有图像,不包括类别 label 的 embedding。
\end{itemize}

\subsection{ACGAN 的损失函数}
\begin{itemize}
  \item 判别器 D 的损失函数:包括原始 GAN 中图片来源的交叉熵损失函数 $L_{s}$,以及
    图片类别的 Softmax 损失函数 $L_{c}$。判别器 D 试图最小化 $L_{s} + L_{c}$。
  \item 生成器 G 的损失函数:生成器 G 试图最小化 $-L_{s} + L_{c}$,注意 $L_{c}$ 的
    符号为正,即 G 也试图使判别器 D 正确给出生成图片的类别,同时想让 D 将生成图
    片误认为是真实图片。
\end{itemize}

\section{SAGAN}
文献\citerb{2018-SAGAN}在 GAN 中加入 self-attention 模块,以改善卷积只能利用局部
信息的问题。

\subsection{SAGAN 中的 self-attention 模块}
SAGAN 中的 self-attention 模块的结构如图~\ref{fig:self-attention}~所示。

\begin{figure}[ht]
  \centering
  \includegraphics[width=\textwidth]{images/GAN/Self-attention.png}
  \caption{SAGAN 中的 self-attention 模块}
  \label{fig:self-attention}
\end{figure}

Self-attention 模块将卷积后的 feature map 相乘经过 softmax 得到 attention map,
引入了全局信息,其中 $f(x), g(x), h(x)$ 的通道数都是输入 feature map 通道数的
1/8,模块输出 $y$ 为:
\begin{equation}
  \label{equ:self-attention}
  y = \gamma \cdot o + x
\end{equation}

其中 $\gamma$ 为网络参数,初值为 0,这样在网络训练初期相当于没有 self-attention
模块参与,只利用原始卷积 feature map。

\subsection{SAGAN 的损失函数}
SAGAN 采用 hinge loss 形式的损失函数:
\begin{align}
  L_{\mathrm{D}} & = -\mathrm{min} \left( 0, -1 + D(x) \right) - \mathrm{min} \left( 0, -1-D(G(z)) \right) \\
  L_{\mathrm{G}} & = -D(G(z))
\end{align}

\subsection{SAGAN 的训练技巧}
\begin{itemize}
  \item 谱归一化(Spectral Normalization,SN):原始论文只在判别器 D 中加入了 SN
    模块,SAGAN 在 G 中也加入了 SN 模块,提高模型训练的稳定性。
  \item 生成器和判别器采用不同学习率:借鉴文献\citerb{2017-TURR}中的思路,判别器
    D 的学习率是生成器 G 的四倍,提高了训练速度。
\end{itemize}

\section{Spectral Normalization}
文献\citerb{2018-SN}提出 Spectral Normalization,用于 GAN 中的判别器,以提升 GAN
训练的稳定性。

假设神经网络中的卷积或全连接层为 $y = Wx$,其中 $W$ 为权重矩阵,SN 方法是将权重
$W$ 归一化为 $W/\sigma(W)$,其中 $\sigma(W)$ 为 $W$ 的谱范数,即:
\begin{equation}
  \sigma(W) = \sqrt{\mathrm{max} \, \left | \mathrm{eig} (W^{\mathrm{T}} W) \right |}
\end{equation}

SN 在实际应用时,采用迭代求解的方式以加快计算速度,具体算法如
图~\ref{fig:sn-algo}~所示。

\begin{figure}[ht]
  \centering
  \includegraphics[width=0.8\textwidth]{images/目标检测/Soft-NMS.pdf}
  \caption{Spectral Normalization 迭代算法流程}
  \label{fig:sn-algo}
\end{figure}


%%% Local Variables:
%%% TeX-master: "../master"
%%% End:
