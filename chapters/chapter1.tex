\part{数学}
\chapter{概率}

%\section{基础}
%
%\section{几何概率}
%
%\subsection{单位圆圆周任取两点的弦长}
%
%问题:单位圆的圆周上,任取两点,求两点之间弦长的期望。

% 模型性能即可达到全部图片训练模型性能的 \textbf{95\%},见表~\ref{tab:AL-Normal}。

%\begin{table}[htb]
%  \centering
%  \caption{主动学习(AL)模型性能与正常训练(Normal)模型性能对比}
%  \label{tab:AL-Normal}
%    \begin{tabular}{cccc}
%    \specialrule{0em}{10pt}{1pt}
%      \toprule[1.5pt]
%      {\heiti 数据集} & {\heiti AL 模型 mAP(35\%)} & {\heiti Normal 模型 mAP(100\%)} & AL/Normal \\ \midrule[1pt]
%      PASCOL VOC & 73.46 & 77.20 & 95.15\% \\
%      KITTI & 54.94 & 57.67 & 95.27\% \\
%      \bottomrule[1.5pt]
%    \end{tabular}
%\end{table}

%%% Local Variables:
%%% TeX-master: "../master"
%%% End: